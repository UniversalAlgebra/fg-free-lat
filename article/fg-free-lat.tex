%% FILE: diffTerm.tex
%% AUTHOR: William DeMeo, Peter Mayr, Nik Ruskuc
%% DATE: 14 May 2018
%% COPYRIGHT: (C) 2018 DeMeo, Mayr, Ruskuk
%%%%%%%%%%%%%%%%%%%%%%%%%%%%%%%%%%%%%%%%%%%%%%%%%%%%%%%%%%
%%                         BIBLIOGRAPHY FILE            %%
%%%%%%%%%%%%%%%%%%%%%%%%%%%%%%%%%%%%%%%%%%%%%%%%%%%%%%%%%%
%% The `filecontents` command will crete a file in the inputs directory called
%% refs.bib containing the references in the document, in case this file does
%% not exist already.
%% If you want to add a BibTeX entry, please don't add it directly to the
%% refs.bib file.  Instead, add it in this file between the
%% \begin{filecontents*}{refs.bib} and \end{filecontents*} lines
%% then delete the existing refs.bib file so it will be automatically generated
%% again with your new entry the next time you run pdfaltex.
\begin{filecontents*}{inputs/refs.bib}
@book {MR1319815,
    AUTHOR = {Freese, Ralph and Je{\v{z}}ek, Jaroslav and Nation, J. B.},
     TITLE = {Free lattices},
    SERIES = {Mathematical Surveys and Monographs},
    VOLUME = {42},
 PUBLISHER = {American Mathematical Society, Providence, RI},
      YEAR = {1995},
     PAGES = {viii+293},
      ISBN = {0-8218-0389-1},
   MRCLASS = {06B25 (06-02 06-04 06B20 68Q25)},
  MRNUMBER = {1319815 (96c:06013)},
MRREVIEWER = {T. S. Blyth},
       DOI = {10.1090/surv/042},
       URL = {http://dx.doi.org/10.1090/surv/042},
}
@article {MR3239624,
    AUTHOR = {Valeriote, M. and Willard, R.},
     TITLE = {Idempotent {$n$}-permutable varieties},
   JOURNAL = {Bull. Lond. Math. Soc.},
  FJOURNAL = {Bulletin of the London Mathematical Society},
    VOLUME = {46},
      YEAR = {2014},
    NUMBER = {4},
     PAGES = {870--880},
      ISSN = {0024-6093},
   MRCLASS = {08A05 (06F99 68Q25)},
  MRNUMBER = {3239624},
       DOI = {10.1112/blms/bdu044},
       URL = {http://dx.doi.org/10.1112/blms/bdu044},
}
@book {MR2839398,
    AUTHOR = {Bergman, Clifford},
     TITLE = {Universal algebra},
    SERIES = {Pure and Applied Mathematics (Boca Raton)},
    VOLUME = {301},
      NOTE = {Fundamentals and selected topics},
 PUBLISHER = {CRC Press, Boca Raton, FL},
      YEAR = {2012},
     PAGES = {xii+308},
      ISBN = {978-1-4398-5129-6},
   MRCLASS = {08-02 (06-02 08A40 08B05 08B10 08B26)},
  MRNUMBER = {2839398 (2012k:08001)},
MRREVIEWER = {Konrad P. Pi{\'o}ro},
}
@article{Freese:2009,
    AUTHOR = {Freese, Ralph and Valeriote, Matthew A.},
    TITLE = {On the complexity of some {M}altsev conditions},
    JOURNAL = {Internat. J. Algebra Comput.},
    FJOURNAL = {International Journal of Algebra and Computation},
    VOLUME = {19},
    YEAR = {2009},
    NUMBER = {1},
    PAGES = {41--77},
    ISSN = {0218-1967},
    MRCLASS = {08B05 (03C05 08B10 68Q25)},
    MRNUMBER = {2494469 (2010a:08008)},
    MRREVIEWER = {Clifford H. Bergman},
    DOI = {10.1142/S0218196709004956},
    URL = {http://dx.doi.org/10.1142/S0218196709004956}
  }
@article {MR3076179,
    AUTHOR = {Kearnes, Keith A. and Kiss, Emil W.},
     TITLE = {The shape of congruence lattices},
   JOURNAL = {Mem. Amer. Math. Soc.},
  FJOURNAL = {Memoirs of the American Mathematical Society},
    VOLUME = {222},
      YEAR = {2013},
    NUMBER = {1046},
     PAGES = {viii+169},
      ISSN = {0065-9266},
      ISBN = {978-0-8218-8323-5},
   MRCLASS = {08B05 (08B10)},
  MRNUMBER = {3076179},
MRREVIEWER = {James B. Nation},
       DOI = {10.1090/S0065-9266-2012-00667-8},
       URL = {http://dx.doi.org/10.1090/S0065-9266-2012-00667-8},
}
@misc{william_demeo_2016_53936,
  author       = {DeMeo, William and Freese, Ralph},
  title        = {AlgebraFiles v1.0.1},
  month        = May,
  year         = 2016,
  doi          = {10.5281/zenodo.53936},
  url          = {http://dx.doi.org/10.5281/zenodo.53936}
}
@article{FreeseMcKenzie2016,
	Author = {Freese, Ralph and McKenzie, Ralph},
	Date-Added = {2016-08-22 19:43:56 +0000},
	Date-Modified = {2016-08-22 19:45:50 +0000},
	Journal = {Algebra Universalis},
	Title = {Mal'tsev families of varieties closed under join or Mal'tsev product},
	Year = {to appear}
}
@misc{UACalc,
	Author = {Ralph Freese and Emil Kiss and Matthew Valeriote},
	Date-Added = {2014-11-20 01:52:20 +0000},
	Date-Modified = {2014-11-20 01:52:20 +0000},
	Note = {Available at: {\verb+www.uacalc.org+}},
	Title = {Universal {A}lgebra {C}alculator},
	Year = {2011}
}
@incollection {MR0472614,
    AUTHOR = {J{\'o}nsson, B. and Nation, J. B.},
     TITLE = {A report on sublattices of a free lattice},
 BOOKTITLE = {Contributions to universal algebra ({C}olloq., {J}\'ozsef
              {A}ttila {U}niv., {S}zeged, 1975)},
     PAGES = {223--257. Colloq. Math. Soc. J\'anos Bolyai, Vol. 17},
 PUBLISHER = {North-Holland, Amsterdam},
      YEAR = {1977},
   MRCLASS = {06A20},
  MRNUMBER = {0472614},
MRREVIEWER = {Ralph Freese},
}
@article {MR0313141,
    AUTHOR = {McKenzie, Ralph},
     TITLE = {Equational bases and nonmodular lattice varieties},
   JOURNAL = {Trans. Amer. Math. Soc.},
  FJOURNAL = {Transactions of the American Mathematical Society},
    VOLUME = {174},
      YEAR = {1972},
     PAGES = {1--43},
      ISSN = {0002-9947},
   MRCLASS = {06A20 (08A15)},
  MRNUMBER = {0313141},
MRREVIEWER = {G. Gratzer},
}
\end{filecontents*}
%:biblio
%\documentclass[12pt]{amsart}
\documentclass[12pt,reqno]{amsart}

%%%%%%% wjd: added these packages vvvvvvvvvvvvvvvvvvvvvvvvv
% PAGE GEOMETRY
% These settings are for letter format
\def\OPTpagesize{8.5in,11in}     % Page size
\def\OPTtopmargin{1in}     % Margin at the top of the page
\def\OPTbottommargin{1in}  % Margin at the bottom of the page
%% \def\OPTinnermargin{0.5in}    % Margin on the inner side of the page
\def\OPTinnermargin{1.5in}    % Margin on the inner side of the page
\def\OPTbindingoffset{0in} % Extra offset on the inner side
%% \def\OPToutermargin{0.75in}   % Margin on the outer side of the page
\def\OPToutermargin{1.5in}   % Margin on the outer side of the page
\usepackage[papersize={\OPTpagesize},
             twoside,
             includehead,
             top=\OPTtopmargin,
             bottom=\OPTbottommargin,
             inner=\OPTinnermargin,
             outer=\OPToutermargin,
             bindingoffset=\OPTbindingoffset]{geometry}
\usepackage{url,amssymb,enumerate,tikz,scalefnt}
\usepackage[normalem]{ulem} % for \sout (strikeout)   wjd: could remove this in final draft
\usepackage[colorlinks=true,urlcolor=blue,linkcolor=blue,citecolor=blue]{hyperref}
\usepackage{algorithm2e}
\usepackage{stmaryrd}

\newcommand\alg[1]{\ensuremath{\mathbf{#1}}}
%% uncomment the next line if we want to revert to the "set" minus notation
%% \renewcommand{\mysetminus}{\ensuremath{\setminus}} 

\usepackage[yyyymmdd,hhmmss]{datetime}
\usepackage{background}
\backgroundsetup{
  position=current page.east,
  angle=-90,
  nodeanchor=east,
  vshift=-1cm,
  hshift=8cm,
  opacity=1,
  scale=1,
  contents={\textcolor{gray!80}{WORK IN PROGRESS.  DO NOT DISTRIBUTE. (compiled on \today\ at \currenttime)}}
}
%%%%%%  (end wjd addition of packages)


\usepackage{pdfcomment}
\usepackage{color}
\usepackage{amsmath}
\usepackage{amssymb}
\usepackage{amsfonts}
\usepackage{mathtools}
\usepackage{amscd}
%% \usepackage{exers}
\usepackage{inputs/wjdlatexmacs}

\usepackage[mathcal]{euscript}
\usepackage{comment}
\usepackage{tikz}
\usetikzlibrary{math} %needed tikz library

\renewcommand{\th}[2]{#1\mathrel{\theta}#2}
\newcommand{\infixrel}[3]{#2\mathrel{#1}#3}
\newcommand\llb{\ensuremath{\llbracket}}
\newcommand\rrb{\ensuremath{\rrbracket}}

%%////////////////////////////////////////////////////////////////////////////////
%% Theorem styles
\numberwithin{equation}{section}
\theoremstyle{plain}
\newtheorem{theorem}{Theorem}[section]
\newtheorem{lemma}[theorem]{Lemma}
\newtheorem{proposition}[theorem]{Proposition}
\newtheorem{prop}[theorem]{Prop.}
\theoremstyle{definition}
\newtheorem{conjecture}{Conjecture}
\newtheorem{claim}[theorem]{Claim}
\newtheorem{subclaim}{Subclaim}
\newtheorem{corollary}[theorem]{Corollary}
\newtheorem{definition}[theorem]{Definition}
\newtheorem{notation}[theorem]{Notation}
\newtheorem{Fact}[theorem]{Fact}
\newtheorem*{fact}{Fact}
\newtheorem{example}[theorem]{Example}
\newtheorem{examples}[theorem]{Examples}
\newtheorem{exercise}{Exercise}
\newtheorem*{lem}{Lemma}
\newtheorem*{cor}{Corollary}
\newtheorem*{remark}{Remark}
\newtheorem*{remarks}{Remarks}
\newtheorem*{obs}{Observation}

\title[Characterization of Bounded Lattices]{A characterization of boundedness\\ for finite lattices}
\author[W.~DeMeo]{William DeMeo}
\email{williamdemeo@gmail.com}
%% \urladdr{http://williamdemeo.github.io}
%% \address{University of Colorado\\Mathematics Dept\\Boulder 80309\\USA}

\author[P.~Mayr]{Peter Mayr}
%% \email{}\urladdr{}
%% \address{University of Colorado\\Mathematics Dept\\Boulder 80309\\USA}

\author[N.~Ru\v{s}kuc]{Nik Ru\v{s}kuc}
%% \email{}\urladdr{}
%% \address{University of St. Andrews\\Mathematics Dept\\St. Andrews, Scottland}

%% \thanks{The first and second authors were supported by the National
%% Science Foundation under Grant No...}

\date{\today}

\begin{document}
\begin{abstract}
Let $X$ be a finite set, $\alg F (X)$ the free lattice 
generated by $X$, and $h \colon \alg F(X) \to \alg L$ a lattice 
epimorphism.
Then $\alg L$ is bounded if and only if the 
kernel of every such $h$ is a finitely generated sublattice of 
$\alg F(X) \times \alg F(X)$. 
% In the present note we prove this new characterization of boundedness.  
\end{abstract}

\maketitle
\section{Introduction}
\label{sec:introduction}

\wjd{TODO. Insert background on the problem, why it's interesting, 
applications, etc.}

\section{Notation and Definitions}

We present some standard notation and definitions needed for 
the statement and proof of our main result (Theorem~\ref{thm:main}).
Many of these are found in the ``Free Lattices'' book by Freese, 
Jezek, Nation~\cite{MR1319815}, although the authors of that book
point out that these concepts were first introduced in Ralph 
McKenzie's work on nonmodular lattice varieties~\cite{MR0313141}, 
and Bjarni J\'onsson's work on sublattices of free lattices~\cite{MR0472614}.

\subsection{Bounded Homomorphisms}
% We follow~\cite{MR1319815} very closely,
% although the authors of that book indicate that the ideas in this 
% subsection have their roots in Ralph McKenzie's work on nonmodular 
% lattice varieties~\cite{MR0313141}, and Bjarni J\'onsson's work on 
% sublattices of free lattices~\cite{MR0472614}.

Let $\alg L = \< L, \wedge, \vee\>$ be a lattice and suppose $x, y \in L$
and $x \leq y$. Then we write $\llbracket x, y \rrbracket$ to denote the 
sublattice of elements between $x$ and $y$; that is, 
\[
\llb x, y \rrb := \{z \in L \mid x \leq z \leq y \}.
\]

Let $\mathbf K$ and $\mathbf L$ be lattices and suppose $\mathbf L$ has bottom 
and top elements, $0$ and $1$, resp.  
If $h \colon \mathbf K \to \mathbf L$ is a lattice homomorphism, then for 
each $a \in L$ consider the sets 
$h^{-1}\llb a, 1 \rrb = \{x \in K \mid h(x) \geq a\}$ and
$h^{-1}\llb 0, a \rrb = \{x \in K \mid h(x) \leq a \}$. 
If the domain $\alg K$ has a top element, $1_{\alg K}$, 
then $h^{-1}\llb a, 1 \rrb$ is nonempty if and only if 
$a \leq h(1_{\mathbf K})$.  
If $\alg K$ has a bottom element, $0_{\alg K}$, 
then $h^{-1}\llb 0, a \rrb$ is nonempty 
if and only if $a \geq h(0_{\mathbf K})$.
When $h^{-1}\llb a, 1 \rrb$ is nonempty,
it is a filter of $\mathbf K$; dually a nonempty $h^{-1}\llb 0, a \rrb$ is an ideal. 
If $\alg K$ is infinite, then 
$h^{-1}\llb a, 1 \rrb$ need not have a least element, nor 
$h^{-1}\llb 0, a \rrb$ a greatest element. However, 
considering when such extrema exist leads to the notion of
bounded homomorphism, which is useful for studying the structure of free lattices.


A lattice homomorphism $h \colon \mathbf K \to \mathbf L$ is called 
\textbf{lower bounded} if for every $a \in L$, the set $h^{-1} \llb a, 1 \rrb$ 
is either empty or has a least element. 
The least element of a nonempty $h^{-1}\llb a, 1\rrb$ is denoted by $\beta(a)$.
Thus, if $h$ is a lower bounded homomorphism, then 
$\beta \colon \mathbf L \rightharpoonup \mathbf K$ 
is a partial function whose domain is an ideal of $\mathbf L$. 

Dually, a homomorphism $h$ is called \textbf{upper bounded} if every 
nonempty $h^{-1}\llb 0, a \rrb$ has a greatest element, called $\alpha (a)$.
If $h$ is an upper bounded homomorphism, then the domain of 
$\alpha \colon \mathbf L \rightharpoonup \mathbf K$ 
is a filter of $\mathbf L$.  
A \textbf{bounded} homomorphism is one that is both upper and lower bounded.

These definitions simplify when $h$ is an epimorphism.
In that case $h$ is lower (upper) bounded if and only if each preimage 
$h^{-1}\{ a \}$ has a least (greatest) element.
Likewise, if $L$ is finite, then $h \colon \mathbf K \to \mathbf L$ is lower (upper)
bounded if and only if each $h^{-1}\{a\}$ has a least (greatest) element whenever 
it is nonempty. Of course, every lattice homomorphism on a finite domain is
bounded.

Note that $\beta$  is monotonic and a left adjoint for $h$; that is, 
$a \leq h(x)$ iff $\beta (a) \leq x$. Therefore, $\beta$ is a join preserving 
map on its domain: 
% if $h^{-1}\llb a, 1 \rrb \neq \emptyset$ and
% $h^{-1}\llb b, 1 \rrb \neq \emptyset$, then
if $h^{-1}\llb a, 1 \rrb \neq \emptyset \neq h^{-1}\llb b, 1 \rrb$, then
$\beta (a \vee b) = \beta (a) \vee \beta (b)$.
Similarly, $\alpha$  is a right adjoint---$h(y) \leq a$ iff
$y \leq \alpha (a)$)---so  $\alpha (a \wedge b) = \alpha (a) \wedge \alpha (b)$, for all 
$a, b$ in the domain of $\alpha$.
In particular, if $h$ is an epimorphism, then $\alpha$ is 
a meet homomorphism, and $\beta$ a join homomorphism, 
from $\mathbf L$ to $\mathbf K$.


\subsection{Unbounded Homomorphisms.}
\newcommand\finsub{\ensuremath{\subseteq_{\omega}}}
If $A$ and $B$ are sets, then by $A \finsub B$ we mean
that $A$ is a finite subset of $B$.
Let $X$ be a set, and $\alg{F}:=\alg{F}(X)$ the free lattice 
generated by $X$. Define a pair of closure operators, denoted by
$^\wedge$ and $^\vee$, on the powerset $\mathcal P (\alg{F})$ as follows: 
for $A\in \mathcal P (\alg{F})$, 
\[
A^\wedge = \{\bigwedge B \mid B \finsub A\} \quad \text{ and }\quad
A^\vee = \{\bigvee B \mid B \finsub A\}.
\]
Define also
% \begin{align*}
%   H_0 &:= X^\wedge = \{\bigwedge B \mid B \finsub X\}, \\
%   H_{k+1} &:= (H_k)^{\vee \wedge} = \{\bigwedge B \mid B \finsub H_k^{\vee}\}, \text{ for all $k\geq 0$,} \\
% & \text{ where } H_k^{\vee} := \{\bigvee A \mid A \finsub H_k\}.
% \end{align*}
$H_0 = X^\wedge = \{\bigwedge B \mid B \finsub X\}$, and for all $k\geq 0$, let
\[ H_{k+1} = (H_k)^{\vee \wedge} = \{\bigwedge B \mid B \finsub H_k^{\vee}\}, \]
where $H_k^{\vee} = \{\bigvee A \mid A \finsub H_k\}$.
Evidentally, $\alg{F}$ is the union of the chain of subsets 
$H_0\subseteq H_1 \subseteq \cdots$.
Each $H_n = X^{\wedge(\vee\wedge)^n}$ is a finite meet-closed 
subset of $\alg{F}$ and, since $X$  generates $\alg{F}$, 
it should be apparent that $\bigcup H_n = \alg{F}$. 


To gain some intuition about $H_k$ $(k\geq 0)$, let's consider the complexity 
of the terms in $H_1 = (H_0)^{\vee \wedge}$. 
\begin{align*}
  H_1 &= \{\bigwedge B \mid B \finsub (H_0)^{\vee}\}= \{\bigwedge B \mid B \finsub X^{\wedge \vee}\}\\
      &= \left\{\bigwedge B \mid B \finsub \bigl\{\bigvee A \mid A \finsub X^\wedge \bigr\}\right\}\\
      &= \left\{\bigwedge B \mid B \finsub \bigl\{\bigvee A \mid A \finsub \{\bigwedge B \mid B \finsub X\} \bigr\}\right\}.
\end{align*}
For a concrete example, if $x_i \in X$, then the following four
terms belong to $H_0$,
\[
t_1 = x_1 \wedge x_2 \wedge x_3, \quad t_2 = x_2 \wedge x_4, \quad t_3 = x_3 \wedge x_5, \quad t_4 = x_1,
\]
and here's a term that belongs to $H_1 \setminus H_0$,
\[
(t_1 \vee t_2) \wedge (t_3 \vee t_4) = \bigl[(x_1 \wedge x_2 \wedge x_3) \vee (x_2 \wedge x_4)\bigr] \wedge \bigl[(x_1 \wedge x_5)\vee x_2\bigr].
\]

Let $h \colon  \alg{F}(X) \to \alg{L}$ be an epimorphism and, 
for each $y \in L$ and $k< \omega$, define
\[
\beta_k(y) = \bigwedge \{w \in H_k : h(w) \geqslant y\}.
\]

\begin{theorem}[\protect{\cite[Thm.~2.2]{MR1319815}}]\label{thm:2.2}
   Let $\mathbf K$ be finitely generated, and let $h \colon \mathbf K \to \mathbf L$ be a lattice
  homomorphism. If $a \leq h(1_{\mathbf K})$, then
\begin{enumerate}
\item $j \leq k$ implies $\beta_j(a) \geq \beta_k(a)$,
\item  $\beta_k(a)$ is the least element of $H_k \cap h^{-1}\llb a, 1 \rrb$,
\item  $h^{-1}\llb a, 1 \rrb = \bigcup_{k\in \omega}\llb \beta_k(a), 1 \rrb$.
\end{enumerate}  
\end{theorem}

\section{Main Theorem}

\begin{theorem}
\label{thm:main}
Let $X$ be a finite set, $\alg F (X)$ the free lattice 
generated by $X$, $\mathbf L = \langle L, \wedge, \vee\rangle$ a finite lattice, and 
$h\colon \alg{F}(X) \twoheadrightarrow \mathbf{L}$ a lattice epimorphism.
Then $h$ is bounded if and only if the kernel of $h$ is a finitely generated 
sublattice of $\alg F (X) \times \alg F (X)$.
\end{theorem}

A key ingredient in our proof of Theorem~\ref{thm:main} is Lemma~\ref{lem:main},
for which we require a practical definition of the length of a term.
If $t \in \alg F(X)$, we define 
the {\bf length} of $t$ as follows:
\begin{itemize}
\item $t$ has \emph{length} $\ell$ if $t \in H_\ell \setminus H_{\ell-1}$
\item $t$ has \emph{length at most} $\ell$ if $t \in H_\ell$.
\end{itemize} 
Observe that if $t$ has length at most $\ell$, then $t\geq \beta_\ell h (t)$.


\begin{lemma}\label{lem:main}
Let $h$ be an epimorphism from $\alg F(X)$ onto a finite lattice $\alg L$.
Let $K$ be a finite subset of the kernel of $h$.
Then there exists $n>0$ such that for all $(p,q) \in \<K\>$, if $p$ has length 
$\ell$, then $q \geq \beta_{\ell+n}h(q)$.
\end{lemma}

\begin{remark}
If $K \subseteq \ker h$, then $(p,q) \in \<K\>$ implies $h(p) = h(q)$. 
\end{remark}
  
\begin{proof}
Let $\alg F := \alg F(X)$, let $K$ be a finite subset of $\ker h$, and
define the following, where $S$ is an arbitrary subset of $\alg F^2$:
 \begin{itemize}
\item $S_1 = \{s_1 \in \alg F \mid \exists s_2, \; (s_1,s_2) \in S \}$,
\item $m = \min \{\ell \mid K_1 \cap H_\ell \neq \emptyset\}$,
\item $M = \min \{\ell \mid K_1 \subseteq H_\ell\}$.
\end{itemize}
In other terms, $m$ and $M$ are the minimum and maximum lengths of 
terms $p$ such that $(p,q) \in K$.
Clearly $m\leq M$, so for all $a\in L$ we have $\beta_m(a) \geq \beta_M(a)$. 
Also, $p\geq \beta_M h(p)$ for all pairs $(p, q)\in K$, since $K_1 \subseteq H_M$.
Moreover, $\exists p \in K_1 \cap H_m$ and, by minimality of $m$, the length of 
this $p$ is exactly $m$ and $p \geq \beta_{m} h(p)$.
Finally, since $K$ is a finite set, we can find $n>0$ such that 
$q \geq \beta_{m+n} h(p)$ for all $(p, q) \in K$.

Let $(p,q) \in \<K\>$ and let $\ell$ be the length of $p$, so 
$p\geq \beta_{\ell}h(p)$.  We want to show $q \geq \beta_{\ell + n}h(p)$.

\noindent {\bf Case 1.} Suppose $(p, q) = (p_1, q_1) \vee (p_2, q_2)$,
and assume the conclusion of Lemma~\ref{lem:main} holds for 
$(p_1, q_1)$ and $(p_2, q_2)$.
Let $\ell_i$ be the length of $p_i$.  Then $p$ has length 
$\ell = \max\{\ell_1, \ell_2\} + 1$. For each $i\in \{1,2\}$,
we have $p_i\geq \beta_{\ell_i}h(p_i)$ and, by the induction hypothesis,
$q_i\geq \beta_{\ell_i+n}h(p_i)$.  Observe also that 
$\ell -1 \geq \ell_i$, so 
$\beta_{\ell_i+n}h(p_i) \geq \beta_{\ell+n-1}h(p_i)$ for each $i\in \{1,2\}$.
Therefore, 
\[
q = q_1\vee q_2\geq \beta_{\ell_1+n}h(p_1) \vee \beta_{\ell_1+n}h(p_2)
\geq \beta_{\ell+n-1}h(p_1) \vee \beta_{\ell+n-1}h(p_2).
\]
To complete the proof in this case, then, it suffices to show,
\begin{equation}
\label{eq:goal}  
\beta_{\ell+n-1}h(p_1) \vee \beta_{\ell+n-1}h(p_2) \geq 
\beta_{\ell+n}h(p).
\end{equation}

Let $t_i = \beta_{\ell+n-1}h(p_i)$; that is, $t_i$ denotes the
least term of length at most 
$\ell + n -1$ that is mapped by $h$ to $h(p_i)$.
Now $t_1 \vee t_2$ is mapped by $h$ to $h(p)$. Indeed,
$h(t_1\vee t_2) = h(t_1)\vee h(t_2) = h(p_1)\vee h(p_2) = h(p_1\vee p_2) = 
h(p)$. Also, the length of $t_1 \vee t_2$ is at most $\ell +n$.
Therefore, as $\beta_{\ell+n}h(p)$ denotes the least term of length at most
$\ell +n$ that is mapped to $h(p)$,~(\ref{eq:goal}) holds.

\medskip

\noindent {\bf Case 2.} Suppose $(p, q) = (p_1, q_1) \wedge (p_2, q_2)$,
and assume the conclusion of the Lemma~\ref{lem:main} holds 
for $(p_1, q_1)$ and $(p_2, q_2)$.
Let $\ell_i$ be the length of $p_i$.  Then
$p_i\geq \beta_{\ell_i}h(p_i)$ and, by the induction hypothesis,
$q_i\geq \beta_{\ell_i+n}h(p_i)$. Our goal once again is to prove
$q\geq \beta_{\ell+n}h(p)$.

For all $k$, $H_k$ is closed under meet, so 
the length of $p$ is 
$\ell = \max\{\ell_1, \ell_2\}$ (in contrast to Case 1).
Still, 
$\beta_{\ell_i+n}h(p_i) \geq \beta_{\ell+n}h(p_i)$, so
\[
q = q_1\wedge q_2\geq \beta_{\ell_1+n}h(p_1) \wedge \beta_{\ell_1+n}h(p_2)
\geq \beta_{\ell+n}h(p_1) \wedge \beta_{\ell+n}h(p_2).
\]
Therefore, to complete the proof it this case, it suffices to show
\[
\beta_{\ell+n}h(p_1) \wedge \beta_{\ell+n}h(p_2) \geq 
\beta_{\ell+n}h(p).
\]
Recall that $\beta_{\ell+n}h(p_i)$ denotes the least term $t_i$ 
of length at most $\ell +n$ such that $h(t_i) = h(p_i)$. 
Now, $t_1 \wedge t_2$ has length at most $\ell+n$, since $H_{\ell+n}$ is 
closed under meet. Also, 
\[
h(t_1\wedge t_2) = 
h(t_1) \wedge h(t_2) = 
h(p_1) \wedge h(p_2) = 
h(p_1 \wedge p_2) = h(p),
\] 
so the image of $t_1\wedge t_2$ under $h$ is $h(p)$.
Since $\beta_{\ell+n}h(p)$ is the least term of length at 
most $\ell+n$ that is mapped to $h(p)$, we have 
$t_1 \wedge t_2 \geq \beta_{\ell+n}h(p)$, as desired.

\section{Proof of the Main Theorem}
With Lemma~\ref{lem:main} established, we are now in a position to prove 
Theorem~\ref{thm:main}.  We begin with the converse direction.  
In fact, we will prove the contrapositive of the converse.
Specifically, if $h: \alg F \to \alg L$ is not lower bounded,
then $\ker h$ is not a finitely generated sublattice of 
$\alg F \times \alg F$.  After proving the converse, we will
assume $h$ is (both upper- and lower-) bounded, and then
prove that $\ker h$ is a finitely generated sublattice of 
$\alg F \times \alg F$. 

\medskip

\noindent {\it Proof of Theorem~\ref{thm:main}.}

  \noindent ($\Leftarrow$)
  Suppose $h: \alg F \to \alg L$ is not lower bounded, 
  and let $K$ be an arbitrary finite subset of $\ker h$.  We prove that the 
  subalgebra $\<K\>\leq \alg F \times \alg F$ generated by $K$ is not 
  all of $\ker h$.
  Since $K$ is an arbitrary finite subset of $\ker h$, this will prove that 
  $\ker h$ is not finitely generated.

  Let $h^{-1}(a)$ be a class of $\ker h$ that is not lower bounded.
  The the sequence $\beta_k(a)$ defined prior to Lemma~\ref{lem:main}
  is an infinite decending chain.  Of course, for every $k\in \N$, 
  we have $h \beta_0(a) = h \beta_k(a)$, so 
  $(\beta_0(a), \beta_k(a)) \in \ker h$. 
  On the other hand, since $K$ is a finite subset of $\ker h$, 
  Lemma~\ref{lem:main} asserts the existence of 
  an $n\in \N$ such that, if $(p,q) \in \<K\>$, then
  $p\geq \beta_\ell h(p)$ implies $p\geq \beta_{\ell +n} h(p)$.
  Therefore, $(\beta_0(a), \beta_{n+1}(a)) \in \ker h \setminus \<K\>$. 
  Thus $\<K\>\neq \ker h$, as we set out to prove.


\medskip

  \noindent ($\Rightarrow$)
  Assume $h$ is a bounded epimorphism so that the preimage of each 
  $y\in L$ under $h$ is bounded.  For each $y\in L$, let 
  $\alpha y= \bigvee h^{-1}\{y\}$ and $\beta y = \bigwedge h^{-1}\{y\}$ 
  denote the greatest and least elements of $h^{-1}\{y\}$, respectively 
  (both of which exist by the assumed boundedness and surjectivity of $h$).  
  Observe that $h \alpha h = h$, and $h \beta h = h$. In fact, $\alpha$ 
  and $\beta$ are adjoint to $h$. Indeed, it is easy to see that
  \[
  h x \leqslant y \quad \Leftrightarrow \quad x \leqslant \alpha y,
  \]
  \[
  y \leqslant h x \quad \Leftrightarrow \quad \beta y \leqslant x.
  \]

For each $y \in L$, let $X_y := X\cap h^{-1}\{y\}$, the set of generators 
that lie in the inverse image of $y$ under $h$.
Let $G$ be the (finite) set of pairs in $\alg{F} \times \alg{F}$ defined 
as follows:
\[
G = \bigcup_{y \in L}\{(x, \alpha y), (\alpha y, x), (x, \beta y), 
                       (\beta y, x), (\alpha y, \beta y), 
                       (\beta y, \alpha y) : x \in X_y\}.
\]
We claim that $G$ generates $\ker h$.  To prove this, we first show, by 
induction on term complexity, that for every $y \in L$, for every 
$r \in h^{-1}\{y\}$, the pairs $(r,\alpha y)$ and $(r,\beta y)$ belong 
to the sublattice $\langle G \rangle \leqslant \alg{F} \times \alg{F}$ 
generated by $G$.

\begin{itemize}
  \item {\bf Case 0.} Suppose $r \in X$. Then $(r,\alpha y)$ and $(r,\beta y)$ 
  belong to $G$ itself, so there's nothing to prove.  

  \item {\bf Case 1.} Suppose $r = s \vee t$. Assume (the induction hypothesis) 
  that $(s, \alpha {h(s)})$, $(s, \beta{h(s)})$, $(t, \alpha {h(t)})$, and 
  $(t, \beta{h(t)})$ belong to $\langle G \rangle$. Then 
  $y = h (r) = h(s\vee t) = h (s)\vee h(t)$, so 
  \[
  h(\alpha {h(s)} \vee \alpha {h(t)})= h\alpha h(s) \vee h\alpha h(t)=
  h(s) \vee h(t) = y.
  \]
  Likewise, $h(\beta{h(s)} \vee \beta {h(t)})= h(s) \vee h(t) = y$.
  Therefore, 
  \[
  \beta y \leqslant \beta h(s) \vee \beta h(t) \leqslant \alpha {h(s)} 
  \vee \alpha {h(t)} \leqslant \alpha y.
  \]
  Also, $r \leqslant \alpha y$, so $r = \alpha y \wedge (s\vee t)$.  Taken 
  together, these observations yield
  \begin{align}
    \left(\begin{array}{c} r \\ \beta y\end{array}\right) 
    &=\left(\begin{array}{c} 
              \alpha y \wedge (s\vee t) \\ 
              \beta y
            \end{array}
      \right) = 
      \left(\begin{array}{c} 
            \alpha y \wedge (s\vee t) \\ 
            \beta y \wedge (\beta {h(s)} \vee \beta {h(t)}) 
            \end{array}
      \right)\nonumber\\
    &=\left(\begin{array}{c} 
            \alpha y\\ 
            \beta y
            \end{array}
      \right) \wedge 
      \left[\left(\begin{array}{c}
                  s \\ 
                  \beta {h(s)}
                  \end{array}
            \right) \vee 
            \left(\begin{array}{c}
                  t \\ 
                  \beta {h(t)} 
                  \end{array}
            \right)
      \right], \nonumber
  \end{align}
  and each term in the last expression belongs to $\langle G \rangle$, 
  so $(r, \beta y) \in \langle G \rangle$, as desired.

Similarly, $(r, \alpha y) \in \langle G \rangle$.  Indeed, $\beta y \leqslant r$ 
implies $r = \beta y \vee s\vee t$, and 
$\beta {h(s)} \vee \beta {h(t)} \leqslant \alpha y$ implies 
$\alpha y = \alpha y \vee \beta {h(s)} \vee \beta {h(t)}$. Therefore,
\begin{align*}
  \left(\begin{array}{c} r \\ \alpha y\end{array}\right) 
  &=\left(\begin{array}{c} 
          \beta y \vee s\vee t \\ 
          \alpha y \vee \beta {h(s)} \vee \beta {h(t)} 
          \end{array}
    \right)\\
  &=\left(\begin{array}{c} 
          \beta y\\ 
          \alpha y
          \end{array}
    \right) \vee 
    \left(\begin{array}{c}
          s \\ 
          \beta {h(s)}
          \end{array}
    \right) \vee 
    \left(\begin{array}{c}
          t \\ 
          \beta {h(t)} 
          \end{array}
    \right).
\end{align*}

\item {\bf Case 2.} Suppose $r = s \wedge t$. Assume $(s, \alpha {h(s)})$, 
$(s, \beta{h(s)})$, $(t, \alpha {h(t)})$, and $(t, \beta{h(t)})$ belong 
to $\langle G \rangle$. Then $h(s\wedge t) = h(r) = y$, so 
$h(\alpha {h(s)} \wedge\alpha {h(t)}) = y = h(\beta {h(s)} \wedge\beta {h(t)})$, 
so $\beta y \leqslant \beta h(s) \wedge \beta h(t) \leqslant \alpha {h(s)} \wedge \alpha {h(t)} \leqslant \alpha y$.
Also, $\beta y \leqslant r \leqslant \alpha y$ so $r = \alpha y \wedge s\wedge t$
and $r = \beta y \vee (s\wedge t)$. Altogether, we have
\begin{align*}
\left(\begin{array}{c} r \\ \alpha a\end{array}\right) &= 
\left(\begin{array}{c} \beta y \vee (s\wedge t) \\ \alpha y \vee (\alpha {h(s)} \wedge \alpha {h(t)}) \end{array}\right)\\
&= \left(\begin{array}{c} \beta y\\ \alpha y\end{array}\right) \vee
\left[\left(\begin{array}{c}s \\ \alpha {h(s)}\end{array}\right) \wedge \left(\begin{array}{c}t \\ \alpha {h(t)} \end{array}\right)\right],
\end{align*}
and each term in the last expression belongs to $\langle Y \rangle$, as desired.
Similarly,
\begin{align*}
\left(\begin{array}{c} r \\ \beta y\end{array}\right) &= 
\left(\begin{array}{c} \alpha y \wedge s\wedge t \\ \beta y \wedge \alpha {h(s)} \wedge \alpha {h(t)} \end{array}\right)\\
&= \left(\begin{array}{c} \alpha y\\ \beta y\end{array}\right) \wedge 
\left(\begin{array}{c}s \\ \alpha {h(s)}\end{array}\right) \wedge \left(\begin{array}{c}t \\ \alpha {h(t)} \end{array}\right).
\end{align*}
Indicentally, in each of the last two derivations, we could have used $\beta$'s 
instead of $\alpha$'s; in both cases the last meet could be replaced with
\[
\left(\begin{array}{c}s \\ \beta {h(s)}\end{array}\right) \wedge \left(\begin{array}{c}t \\ \beta {h(t)} \end{array}\right).
\]
In each case, we end up with an expression involving terms from $\langle G \rangle$, 
and this proves that $(r, \alpha y)$ and $(r, \beta y)$ belong to $\langle G \rangle$, as desired.
\end{itemize}
\end{proof}







\bibliographystyle{alphaurl}
\bibliography{inputs/refs}


%%%%%%%%%%%%%%%%%%%%%%%%%%%%%%%%%%%%%%%%%%%%%%%%%%%%%%%%%%%%%%%%%%%%%%%%%%
%%%%%%%%%%%%%%%%%%%%%%%%%%%%%%%%%%%%%%%%%%%%%%%%%%%%%%%%%%%%%%%%%%%%%%%%%%
%%%%%%%%%%%%%%%%%%%  END OF DOCUMENT %%%%%%%%%%%%%%%%%%%%%%%%%%%%%%%%%%%%%
%%%%%%%%%%%%%%%%%%%%%%%%%%%%%%%%%%%%%%%%%%%%%%%%%%%%%%%%%%%%%%%%%%%%%%%%%%
%%%%%%%%%%%%%%%%%%%%%%%%%%%%%%%%%%%%%%%%%%%%%%%%%%%%%%%%%%%%%%%%%%%%%%%%%%

\end{document}

%%%%%%%%%%%%%%%%%%%%%%%%%%%%%%%%%%%%%%%%%%%%%%%%%%%%%%%%%%%%%%%%%%%%%%%%%%
%%%%%%%%%%%%%%%%%%%%%%%%%%%%%%%%%%%%%%%%%%%%%%%%%%%%%%%%%%%%%%%%%%%%%%%%%%
%%%%%%%%%%%%%%%%%%%  END OF DOCUMENT %%%%%%%%%%%%%%%%%%%%%%%%%%%%%%%%%%%%%
%%%%%%%%%%%%%%%%%%%%%%%%%%%%%%%%%%%%%%%%%%%%%%%%%%%%%%%%%%%%%%%%%%%%%%%%%%
%%%%%%%%%%%%%%%%%%%%%%%%%%%%%%%%%%%%%%%%%%%%%%%%%%%%%%%%%%%%%%%%%%%%%%%%%%





























  \appendix

\section{Examples}

\subsection{$\mathbf{M_3}$}

Let $\mathbf{M_3} = \langle \{0, a, b, c, 1\}, \wedge, \vee\rangle$, where $a \wedge b = a \wedge c = b \wedge c = 0$ and $a \vee b = a \vee c = b \vee c = 1.$ Let $\alg{F} := \alg{F}(x, y, z)$ denote the free lattice generated by $\{x, y, z\}$.

\begin{proposition}
Let $h\colon \mathbf{F} \twoheadrightarrow \mathbf{M_3}$ be the epimorphism that acts on the generators as follows: $x\mapsto a$, $y\mapsto b$, $z\mapsto c.$ Then $\operatorname{ker} h$ is not finitely generated.
%% If $K = \operatorname{ker}\left(\mathbf{F}\{x, y, z\} \twoheadrightarrow \mathbf{M_3}\right)$, then $K$ is not finitely generated.
\end{proposition}
\begin{proof}
Let $K := \operatorname{ker} h$, and for $u \in \{x, y, z\}$ let $C_u := u/K := \{v \in F : h(v) = h(u)\}$.
Define sequences of elements in these classes as follows:
let 
\[x_0 := x, \quad y_0 := y, \quad z_0 := z, \quad \text{ and for $i< \omega$, }\]
\[
x_{i+1} := x\vee (y_i \wedge z_i), \quad y_{i+1} := y\vee (x_i \wedge z_i), \quad z_{i+1} := z\vee (x_i \wedge y_i),
\]
\[
m_i := (x_i \wedge y_i) \vee (x_i \wedge z_i)\vee (y_i \wedge z_i).
\]
Summarizing these observations,
\begin{align*}
m_0 &= (x\wedge y) \vee (x\wedge z)\vee (y\wedge z),\\
x_1 &= x \vee (y \wedge z), \quad y_1 = y \vee (x \wedge z),  \quad z_1 = z \vee (x \wedge y),\\  
m_1 &= 
 (x_1 \wedge y_1) \vee (x_1 \wedge z_1)\vee (y_1 \wedge z_1),\\
x_2 &= x \vee \bigl\{\bigl[y \vee (x \wedge z)\bigr] \wedge \bigl[z \vee (x \wedge y)\bigr]\bigr\},\\  
y_2 &= y \vee \bigl\{\bigl[x \vee (y \wedge z)\bigr] \wedge \bigl[z \vee (x \wedge y)\bigr]\bigr\},\\  
z_3 &= z \vee (x_2 \wedge y_2)\\ 
& \vdots
\end{align*}

Let $X$ be a finite subset of $K$.  We will prove there exists $(p,q) \in K \setminus \langle X \rangle$.  
Fix $u\in \{x, y, z, m\}$ and let $\{u_i\}$ be the corresponding sequence defined above. 
Since $X$ is finite, Lemma~\ref{lem:2} implies that there exists 
$M \in \mathbb{N}$ such that for every $(p, q) \in X$ with $p, q \in C_u$, we have 
$p, q \leqslant u_M$.

\begin{subclaim}\label{claim:2.1}
   For $(p,q) \in \langle X \rangle$ and $u \in \{x, y, z\}$, the following implication holds:
\begin{equation}
  \label{eq:star}
q \leqslant u \quad \Longrightarrow \quad p\leqslant u_M.
\end{equation}
\end{subclaim}
We prove the subclaim by induction on the complexity of terms.
Fix $(p,q) \in \langle X \rangle$. Then $p, q \in C_u$ for some $u\in \{x, y, z\}$.
\begin{itemize}
\item {\bf Case 0.} Suppose $(p, q) \in X$. Then by definition of $M$ we have $p, q \leqslant u_M$.

\item {\bf Case 1.} Suppose $(p, q) = (p_1, q_1) \wedge (p_2, q_2)$, where 
  $(p_i, q_i)$ satisfies~(\ref{eq:star}) for $i = 1, 2$.    
  If $q= q_1 \wedge q_2 \leqslant u$, then, since generators in the free lattice are meet-prime (see Theorem~\ref{thm:wordprob} below), we have $q_1\leqslant u$ or $q_2\leqslant u$. Assume $q_1\leqslant u$.  Then, by the induction hypothesis, $p_1\leqslant u_M$.  Therefore, $p = p_1\wedge p_2 \leqslant u_M$, as desired.

\item {\bf Case 2.} Suppose $(p, q) = (p_1, q_1) \vee (p_2, q_2)$, where 
  $(p_i, q_i)$ satisfies~(\ref{eq:star}) for $i = 1, 2$.    
  If $q= q_1 \vee q_2 \leqslant u$, then 
  $q_i \leqslant u$ for $i = 1, 2$.
  It now follows from the induction hypothesis that $p_i\leqslant u_M$ for $i = 1, 2$, so 
  $p = p_1 \vee p_2 \leqslant u_M$, as desired.
\end{itemize}
This completes the proof of Subclaim~\ref{claim:2.1}. It now follows from Lemma~\ref{lem:1} that 
$(x, x_{M+1})\in K \setminus \langle X \rangle$, so proposition is proved.
\end{proof}

\begin{lemma}%##### Lemma 1
  \label{lem:1}
For each $u \in \{x, y, z, m\}$, the sequence $\{u_i\}$ is a strictly ascending chain; that is, $u_0 < u_1 < u_2 < \cdots$.  
\end{lemma}
\begin{proof}  We split the proof up into cases: either $u \in \{x, y, z\}$, or $u = m$.
\begin{itemize}
  \item  \textbf{Case 1.} $u \in \{x, y, z\}$.\\
  For simplicity, assume $u = x$ for the remainder of the proof of this case.  
  (Of course, the same argument applies to the case when $u$ is $y$ or $z$.) 
  Fix $n <\omega$.  We prove $x_n < x_{n+1}$.

  % \noindent {\it Subclaim.} For all $n\in \mathbb N$,
  \begin{subclaim}\label{claim:3.2.1}For all $n< \omega$,
    \begin{enumerate}
      \item 
      $x_n\in C_x$, 
      \item $x_n\ngeq y$, and $x_n\ngeq z$.
    \end{enumerate}
  \end{subclaim}
  {\it Proof of Subclaim~\ref{claim:3.2.1}.}    The first item is obvious; 
  for the second, if $x_n\geqslant y$, then $x_n\wedge y = y$, and 
    then $0 = h(x_n\wedge y) = h(y) = b$.  A similar contradiction is reached 
    if we assume $x_n\geqslant z$, so the subclaim is proved.
  
    Recall, $x_{n+1} = x_n \vee (y_n \wedge z_n)$, so $x_{n+1} > x_n$
    holds as long as $x_n \ngeqslant y_n \wedge z_n$.
    So, by way of contradiction, suppose 
    \begin{equation}\label{eq:doubledaggar}
      x_n \geqslant y_n \wedge z_n.
    \end{equation}

    Now, $y_n = y \vee (x \wedge z) \vee \cdots$, so clearly $y_n \geqslant y$.  
    Similarly, $z_n \geqslant z$.  This, together with~(\ref{eq:doubledaggar}), 
    implies $x_n \geqslant y_n \wedge z_n \geqslant y \wedge z$.  
    But then Theorem~\ref{thm:wordprob} below implies that either 
    $x_n \geqslant y$ or $x_n \geqslant z$, which contradicts Subclaim~\ref{claim:3.2.1}.

\item \textbf{Case 2.} $u = m$.\\
  We first prove that $m_0 = (x\wedge y) \vee (x\wedge z)\vee (y\wedge z)$ is strictly below 
  $m_1 = (x_1\wedge y_1) \vee (x_1\wedge z_1) \vee (y_1\wedge z_1)$.
  By symmetry, it suffices to show $x\wedge y < x_1\wedge y_1$;
  that is, $x\wedge y < \bigl[x \vee (y\wedge z)\bigr]\wedge \bigl[y \vee (x\wedge z)\bigr]$.

  Clearly 
  $x\wedge y \leqslant \bigl[x \vee (y\wedge z)\bigr]\wedge \bigl[y \vee (x\wedge z)\bigr]$. 
  Suppose 
  $x\wedge y = \bigl[x \vee (y\wedge z)\bigr]\wedge \bigl[y \vee (x\wedge z)\bigr]$. 
  Then $\bigl[x \vee (y\wedge z)\bigr]\wedge \bigl[y \vee (x\wedge z)\bigr]\leqslant x$.
  By Theorem~\ref{thm:wordprob}, the latter holds iff
  $x \vee (y\wedge z)\leqslant x$ or
  $y \vee (x\wedge z)\leqslant x$
  The first of these inequalities is clearly false, so it must be the case that 
  $y \vee (x\wedge z)\leqslant x$.  But then $y \leqslant x$, which is obviously false.  
  We conclude that
  $x\wedge y < \bigl[x \vee (y\wedge z)\bigr]\wedge \bigl[y \vee (x\wedge z)\bigr]$.
  This proves $m_0 < m_1$.

  Now fix $n < \omega$ and assume $m_n < m_{n+1}$.
  We show $m_{n+1} < m_{n+2}$.
\wjd{Complete the proof of this case.}

\end{itemize}
\end{proof}

\begin{comment}
  $\downarrow$ \textit{begin scratch work} $\downarrow$

  $m_{n} := (x_n \wedge y_n) \vee (x_n \wedge z_n)\vee (y_n \wedge z_n)$,

  $m_{n+1} := (x_{n+1} \wedge y_{n+1}) \vee (x_{n+1} \wedge z_{n+1})\vee (y_{n+1} \wedge z_{n+1})$,

  By the first Case above, $u_n < u_{n+1}$.
  
  $\uparrow$ \textit{end scratch work} $\uparrow$
\end{comment}


\begin{lemma}\label{lem:2} %##### Lemma 2 
For all $u \in \{x, y, z\}$ and $p \in C_u \cup C_0$ there exists $n \in \mathbb N$ such that $p\leqslant m_{u,n}$.  
\end{lemma}

\begin{proof} We prove this by induction on the complexity of $p$. 

\begin{itemize}
  \item {\bf Case 0.} $p\in \{x, y, z\}$. Then $u = p = m_{p,0}$.\\
  For the remaining cases assume $u = x$, without loss of generality.

  \item {\bf Case 1.} $p = p_1 \vee p_2$.  \\
  If $p \in C_x\cup C_0$, then $p_i \in C_x \cup C_0$ for $i = 1, 2$, and 
  the induction hypothesis yields $i$ and $j$ for which $p_1 \leqslant x_i$ 
  and $p_2 \leqslant x_j$. Letting $n = \max\{i, j\}$, we have 
  $p_1, p_2 \leqslant x_n$, from which $p = p_1 \vee p_2 \leqslant x_n$, as desired.

  \item {\bf Case 2.} $p = p_1 \wedge p_2$.  \\
  If $p \in C_x$, then we may assume $p_1 \in C_x$ and $p_2 \in C_x \cup C_0$. 
  By the induction hypothesis, there exists $n\in \mathbb N$ such that 
  $p_1 \leqslant x_n$, whence $p \leqslant p_1 \leqslant x_n$.
  If $p \in C_0$, then each $p_i$ belongs to $C_u \cup C_0$ for some 
  $u\in \{x, y, z\}$.  If $p_1 \in C_x \cup C_0$, then $p_1 \leqslant x_n$, as 
  above and we're done.  Similarly, if $p_2 \in C_x \cup C_0$.  So assume 
  $p_1 \in C_y \cup C_0$ and $p_2 \in C_z \cup C_0$. Then the induction 
  hypothesis implies that there exist $i$ and $j$ such that 
  $p_1 \leqslant y_i$ and $p_2 \leqslant z_j$. If $n = \max\{i, j\}$, then
  $p_1 \leqslant y_n$ and $p_2 \leqslant z_n$.  Then, by the above definition 
  of the sequences, we have $p_1 \wedge p_2 \leqslant y_n \wedge z_n \leqslant 
  m_n \leqslant x_{n+1}$.
\end{itemize}
\end{proof}



\subsection{J.B.'s Example}
An example in which the unbounded
classes of $\ker h$ do not contain generators of $\alg{F}$ we brought to our attention by J.B.~Nation.  Despite the fact that the unbounded kernel classes do not contain generators, we are still able to prove that the kernel is not a finitely generated subalgebra of $\alg{F} \times \alg{F}$.
\begin{proposition}
  \label{prop:2.2}
Let $\mathbf{F} = \mathbf{F}(x,y,z)$, and let $\mathbf{L} = \mathbf{F}_{\mathbf{M}_3}(0,1,2)$ (see Figure~\ref{fig:1}).  
Let $h\colon \mathbf{F} \twoheadrightarrow \mathbf{L}$ be the epimorphism induced by $x\mapsto 0$, $y\mapsto 1$, $z\mapsto 2$. Then $\operatorname{ker}h$ is not finitely generated.  
\end{proposition}

\newcommand{\dotsize}{1.5pt}
\tikzstyle{lat} = [circle,draw,inner sep=\dotsize]
\newcommand{\figscale}{1}
\begin{figure}
\begin{tikzpicture}[scale=\figscale]
  \foreach \j in {0,...,8} {
    \node[lat] (0\j) at (0,\j) {};
  }
  \foreach \j in {1,2,4,6,7} {
    \node[lat] (n1\j) at (-1,\j) {};
    \node[lat] (1\j) at (1,\j) {};
  }
  \foreach \j in {3,5} {
    \node[lat] (n2\j) at (-2,\j) {};
    \node[lat] (2\j) at (2,\j) {};
  }
  \node[lat] (n34) at (-3,4) {};
  \node[lat] (34) at (3,4) {};

  \draw[semithick] (00) -- (01);
  \draw[semithick] (02) -- (03) -- (04) -- (05) -- (06);
  \draw[semithick] (07) -- (08);
  \draw[semithick] (n16) -- (n17);
  \draw[semithick] (16) -- (17);
  \draw[semithick] (n11) -- (n12);
  \draw[semithick] (11) -- (12);

  \draw[semithick] (00) -- (n11) -- (02) -- (11) -- (00);
  \draw[semithick] (06) -- (n17) -- (08) -- (17) -- (06);

  \draw[semithick] (01) -- (n12) -- (n23) -- (n34);
  \draw[semithick] (01) -- (12) -- (23) -- (34);
  \draw[semithick] (12) -- (03) -- (n14) -- (n25);
  \draw[semithick] (n12) -- (03) -- (14) -- (25);
  \draw[semithick] (23) -- (14) -- (05) -- (n16);
  \draw[semithick] (n23) -- (n14) -- (05) -- (16);
  \draw[semithick] (34) -- (25) -- (16) -- (07);
  \draw[semithick] (n34) -- (n25) -- (n16) -- (07);
  % \node (n13) at (-1,3) [circle,fill,inner sep=\dotsize]{};
  % \node (n15) at (-1,5) [circle,fill,inner sep=\dotsize]{};
  % \node (n24) at (-2,4) [circle,fill,inner sep=\dotsize]{};
  \node[lat] (n13) at (-1,3) {};
  \node[lat] (n15) at (-1,5) {};
  \node[lat] (n24) at (-2,4) {};
  \draw[thick,blue] (02) -- (n13) -- (n24) -- (n15) -- (06);
  \draw[thick,blue] (n13) -- (04) -- (n15);

  \node (01) at (0,1) [green,circle,fill,inner sep=\dotsize]{};
  \node (n11) at (-1,1) [green,circle,fill,inner sep=\dotsize]{};
  \node (11) at (1,1) [green,circle,fill,inner sep=\dotsize]{};
  \node (n25) at (-2,5) [red,circle,fill,inner sep=\dotsize]{};
  \node (x) at (-3,4) [circle,fill,inner sep=\dotsize]{};
  \node (y) at (3,4) [circle,fill,inner sep=\dotsize]{};
  \node (25) at (2,5) [red,circle,fill,inner sep=\dotsize]{};
  \node (z) at (-2,4) [circle,fill,inner sep=\dotsize]{};
  \node (n15) at (-1,5) [red,circle,fill,inner sep=\dotsize]{};
  \node (03) at (0,3) [orange,circle,fill,inner sep=\dotsize]{};
  \draw (x) node[left] {$0$};
  \draw (n11) node[left] {$0 \wedge 2$};
  \draw (25) node[right] {$1 \vee (0 \wedge 2)$};
  \draw (y) node[right] {$1$};
  \draw (z) node[left] {$2$};
  \end{tikzpicture}
  \caption{The free lattice over $M_3$ generated by $\{0, 1, 2\}$. 
  Green dots identify elements $0 \wedge 1$, 
  $0 \wedge 2$, and $1 \wedge 2$; red dots identify $0 \vee (1 \wedge 2)$, 
  $1 \vee (0 \wedge 2)$, and $2 \vee (0 \wedge 1)$.}
  \label{fig:1}
\end{figure}

To prove this we will need a technical lemma about the behavior of 
certain sequences of elements of $\alg{F}$.

\begin{lemma}\label{claim:2.3}
For each $s \in \{x, y, z, m\}$, define the sequence $\{s_i : i < \omega\}$
of elements of $\alg{F}$ as follows: 
$x_0 = x$, $y_0 = y$, $z_0 = z$, and for $i\geqslant 0$, 
\begin{gather*}
  x_{i+1} = x\vee (y_i \wedge z_i), \quad  y_{i+1} = y\vee (x_i \wedge z_i), \quad  z_{i+1} = z\vee (x_i \wedge y_i),\\
  m_i = (x_i \wedge y_i) \vee (x_i \wedge z_i) \vee (y_i \wedge z_i).
\end{gather*}
  If $\{s_i : i< \omega\}$ is any one of the four sequences just defined, then for every $i\geq 1$, 
  we have $s_{i+1} > s_i$ and $h(s_{i}) = h(s_1)$.
\end{lemma}
\begin{proof}
% It suffices to prove the claim for the sequences $\{m_n\}$ and $\{x_n\}$.
First observe that $h(x_1) = h(x \vee (y \wedge z)) = 0 \vee (1 \wedge 2)$.  We begin by proving 
$h(x_{2}) = h(x_1)$.
By definition, we have 
$h(x_{2}) = h(x \vee (y_1 \wedge z_1)) = h(x) \vee \bigl[h(y_1) \wedge h(z_1)\bigr]$,
and $h(y_1) = h(y \vee (x \wedge z)) =   h(y) \vee \bigl[h(x) \wedge h(z)\bigr] 
= 1 \vee (0 \wedge 2)$. Similarly, $h(z_1) = 2 \vee (0 \wedge 1)$.
Therefore, 
\begin{equation}\label{eq:h2h1} 
  h(x_{2}) = 
0 \vee \bigl\{[1 \vee (0 \wedge 2)] \wedge [2 \vee (0 \wedge 1)]\bigr\}.
\end{equation}
Recall, the modular law: 
$x \leq b$ implies $x \vee (a \wedge b) = (x \vee a) \wedge b$.
Applying this law with $a = 1$ and $x = 0 \wedge 2 \leq 2 \vee (0 \wedge 1) = b$, we have 
\[
  (0 \wedge 2) \vee \bigl\{ 1 \wedge [2 \vee (0 \wedge 1)]\bigr\} = [1 \vee (0 \wedge 2)] \wedge [2 \vee (0 \wedge 1)],
\]  
which is the right joinand in~\ref{eq:h2h1}.  Therefore,
\[
  h(x_{2}) = 
  0 \vee 
  (0 \wedge 2) \vee \bigl\{ 1 \wedge [2 \vee (0 \wedge 1)]\bigr\} = 
  0 \vee 
  \bigl\{ 1 \wedge [2 \vee (0 \wedge 1)]\bigr\}.
\]
Applying the modular law once more to $1 \wedge [2 \vee (0 \wedge 1)]$, 
with $a = 2$ and $x = 0\wedge 1 \leq 1 = b$, we have
\[
  1 \wedge [2 \vee (0 \wedge 1)]  = (0 \wedge 1) \vee (1\wedge 2).
\]
% [0 \vee (1 \wedge 2)] \wedge [1 \vee (0 \wedge 2)]
Therefore,
\[
  h(x_{2}) = 
  0 \vee 
  \bigl\{ 1 \wedge [2 \vee (0 \wedge 1)]\bigr\} = 
  0 \vee (0 \wedge 1) \vee (1\wedge 2) = 0 \vee (1\wedge 2),
 \] 
as desired. 
Of course, $h(y_{2}) = h(y_1)$ and $h(z_{2}) = h(z_1)$ can be checked similarly.

With the base cases established, we proceed with the induction. 
Fix $n\geq 1$ and assume $h(x_{n}) = h(x_1)$, $h(y_{n}) = h(y_{1})$, and $h(z_{n}) = h(z_{1})$. 
  Then,
  \[h(x_{n+1}) = h(x \vee (y_n \wedge z_n))
    =h(x) \vee (h(y_n) \wedge h(z_n))= h(x) \vee (h(y_{1}) \wedge h(z_{1})).\]
As observed above, this is equal to $h(x_2)$, which in turn is 
$h(x_1)$, as desired.  By the same argument, $h(y_{n+1})=h(y_{1})$ and $h(z_{n+1})=h(z_{1})$.
This proves that for all $n \geq 1$, we have $h(s_{n})=h(s_{1})$, when 
$\{s_n\}$ is $\{x_n\}$ or $\{y_n\}$ or $\{z_n\}$.

Finally, consider $\{m_n\}$. 
For all $n\geq 1$, we have
\begin{align*}
  h(m_n) &= \bigl[h(x_n) \wedge h(y_n)\bigr] \vee \bigl[h(x_n) \wedge h(z_n)\bigr]\vee \bigl[h(y_n) \wedge h(z_n)\bigr] \\
  &= \bigl[h(x_1) \wedge h(y_1)\bigr] \vee \bigl[h(x_1) \wedge h(z_1)\bigr]\vee \bigl[h(y_1) \wedge h(z_1)\bigr]\\
  &=h(m_1).
  \end{align*}
(In fact, in this case we can show that $h(m_1) = h(m_0)$, but this is unnecessary.)


It remains to show that $s_{i+1} > s_i$ for all $i<\omega$ (for each $s \in \{x, y, z, m\}$).  
For this we will need the following.

\begin{claim}\label{claim:gens-below} For all $i \geq 0$, $x\nleq y_i$ and $x \nleq z_i$.
\end{claim}
\begin{proof}
When $i=0$, the claim is that $x\nleq y$ and $x \nleq z$, which is clear.  
Fix $k\geq 0$ and suppose $x\nleq y_k$ and $x \nleq z_k$. We show 
$x\nleq y_{k+1}$ and $x \nleq z_{k+1}$.
Assume the contrary, say, $x\leq y_{k+1} = y \vee (x_k \wedge z_k)$.
Since $x$ is a generator, it is join prime. Therefore, as $x\nleq y$, we must have
$x \leq x_k \wedge z_k$. But this is impossible since 
$x \nleq z_k$, by the induction hypothesis. Thus, Claim~\ref{claim:gens-below} is proved.
\end{proof}
Observe that the claim just proved also yields the following: $\forall m, n$, $x_m \nleq y_n$. 

We will now prove $\forall i \geq 0$ that all of the following strict inequalities hold:
\begin{gather}
x_i < x_{i+1}, \quad y_i < y_{i+1}, \quad z_i < z_{i+1}\nonumber \\
x_i \wedge y_i < x_{i+1} \wedge  y_{i+1}, \quad 
x_i \wedge z_i < x_{i+1} \wedge  z_{i+1}, \quad 
y_i \wedge z_i < y_{i+1} \wedge  z_{i+1}.\label{strict-ineq2}
\end{gather}
Clearly $x_1 = x \vee (y \wedge z) > x = x_0$, by Theorem~\ref{thm:wordprob}. 
Similarly, $y_1 > y_0$ and $z_1 > z_0$.  
Also, 
\[
x_1 \wedge y_1 = \bigl[x \vee (y \wedge z)\bigr] \wedge \bigl[y \vee (x \wedge z)\bigr]
 > x \wedge y.\]
For otherwise we would have
$\bigl[x \vee (y \wedge z)\bigr] \wedge \bigl[y \vee (x \wedge z)\bigr]
= x \wedge y$, which implies
$\bigl[x \vee (y \wedge z)\bigr] \wedge \bigl[y \vee (x \wedge z)\bigr]
\leq x$; but $x$ is meet prime, so we would have
either $x \vee (y \wedge z) \leq x$ or 
$y \vee (x \wedge z)\leq x$, both of which are clearly false.
The arguments establishing the remaining inequalities in~(\ref{strict-ineq2}) 
in case $i=0$ are similar.

Now, fix $n\geq 0$ and suppose the strict
inequalities in~(\ref{strict-ineq2}) hold when $i=n$.
We show they also hold when $i = n+1$.  It suffices to prove just two of these, namely,
$x_{n+1} > x_n$ and $x_{n+1} \wedge y_{n+1} > x_n \wedge y_n$; the other cases are symmetric. 

Recall $x_{n+1} = x \vee (y_n \wedge z_n)$ and 
$x_{n} = x \vee (y_{n-1} \wedge z_{n-1})$, and 
the induction hypothesis implies that $y_n \wedge z_n > y_{n-1} \wedge z_{n-1}$.
Therefore, 
$x_{n+1} \geq x_{n}$, so we must show $x_{n+1} \neq x_{n}$.  
Suppose on the contrary that 
\begin{equation}\label{eq:w+app}
x_{n+1} = x \vee (y_n \wedge z_n) = x \vee (y_{n-1} \wedge z_{n-1}) = x_n.
\end{equation}
Then $y_n \wedge z_n \leq x \vee (y_{n-1} \wedge z_{n-1})$.  
Recall Whitman's Theorem~(\ref{thm:whitman2}) asserting that 
$\mathbf{F}(X)$ satisfies condition (W+), repeated here for easy reference.
\begin{quote}  
  If $v = v_1 \wedge \cdots \wedge v_r \wedge x_1 \wedge \cdots
  \wedge x_n \leqslant u_1 \vee \cdots \vee u_s \vee
  y_1 \vee \cdots \vee y_m = u$, where $x_i, y_j\in X$, then either 
  $x_i = y_j$ for some $i$ and $j$, or $v_i \leqslant u$ for some $i$, or
  $v \leqslant u_j$ for some $j$. 
\end{quote}  
This condition and~(\ref{eq:w+app}) together imply that one of the following must hold:
\begin{enumerate}
\item $y_n \leq x \vee (y_{n-1} \wedge z_{n-1})$;
\item $z_n \leq x \vee (y_{n-1} \wedge z_{n-1})$;
%\item $y_n \wedge z_n \leq x$;  
\item $y_n \wedge z_n \leq y_{n-1} \wedge z_{n-1}$.  
\end{enumerate}
The first two inequalities are simply $y_n \leq x_n$ and $z_n \leq x_n$,
which are both ruled out by the remark immediately following the proof of Claim~\ref{claim:gens-below},
%Since $x$ is meet prime, $y_n \wedge z_n \leq x$ is also easily dismissed.
and $y_n \wedge z_n \leq y_{n-1} \wedge z_{n-1}$ is ruled out by the induction hypothesis.
By reaching this contradiction we have established that $x_{n+1} > x_n$.  

It remains to prove $x_{n+1} \wedge y_{n+1} > x_n \wedge y_n$ under the assumption 
that all of the strict inequalities in~(\ref{strict-ineq2}) hold when $i=n$.
Since we have already established that $x_{n+1} > x_n$ and (by an identical argument)
that $y_{n+1} > y_n$, it's clear that $x_{n+1} \wedge y_{n+1}  \geq x_n \wedge y_n$,
so we just have to rule out equality.  Suppose on the contrary that  
% x_n \wedge y_n  < x_{n+1} \wedge y_{n+1} > 
\begin{align*}
x_n \wedge y_n &:= \bigl[x \vee (y_{n-1} \wedge z_{n-1})\bigr] \wedge 
\bigl[y \vee (x_{n-1} \wedge z_{n-1})\bigr]\\
&= \bigl[x \vee (y_n \wedge z_n) \bigr] \wedge \bigl[y \vee (x_n \wedge z_n)\bigr] =: x_{n+1} \wedge y_{n+1}.
\end{align*}
Then we have both $x_{n+1} \wedge y_{n+1} \leq x_{n}$ and $x_{n+1} \wedge y_{n+1} \leq y_{n}$.
Equivalently,
\begin{align}
\bigl[x \vee (y_{n} \wedge z_{n})\bigr] \wedge 
\bigl[y \vee (x_{n} \wedge z_{n})\bigr] &\leq x \vee (y_{n-1} \wedge z_{n-1})\nonumber\\
\bigl[x \vee (y_{n} \wedge z_{n})\bigr] \wedge 
\bigl[y \vee (x_{n} \wedge z_{n})\bigr] &\leq y \vee (x_{n-1} \wedge z_{n-1}).\label{ineqs}
\end{align}
% The first of these is equivalent to 
% \[
% \bigl[x \vee (y_{n} \wedge z_{n})\bigr] \wedge 
% \bigl[y \vee (x_{n} \wedge z_{n})\bigr] \leq x \vee (y_{n-1} \wedge z_{n-1}). \]
Again, by (W+), for the first of these inequalities to hold, we must have one of the following:
\begin{enumerate}
\item $x \vee (y_{n} \wedge z_{n})\leq x \vee (y_{n-1} \wedge z_{n-1})$;
\item $y \vee (x_{n} \wedge z_{n}) \leq x \vee (y_{n-1} \wedge z_{n-1})$;
\item $\bigl[x \vee (y_{n} \wedge z_{n})\bigr] \wedge 
\bigl[y \vee (x_{n} \wedge z_{n})\bigr] \leq y_{n-1} \wedge z_{n-1}$.
\end{enumerate}
The first of these is equivalent to $x_{n+1} \leq x_{n}$, and the second to 
$y_{n+1} \leq x_n$; we have already ruled out both of these cases.
Thus the only remaining possibility is the third, which can be restated as 
$x_{n+1} \wedge y_{n+1} \leq y_{n-1} \wedge z_{n-1}$. Let's assume this holds.

In order for the second of the inequalities in~(\ref{ineqs}) to hold, 
according to (W+), one of the following must also be true:
\begin{enumerate}
\item $x \vee (y_{n} \wedge z_{n})\leq y \vee (x_{n-1} \wedge z_{n-1})$;
\item $y \vee (x_{n} \wedge z_{n}) \leq y \vee (x_{n-1} \wedge z_{n-1})$;
\item $\bigl[x \vee (y_{n} \wedge z_{n})\bigr] \wedge 
\bigl[y \vee (x_{n} \wedge z_{n})\bigr] \leq x_{n-1} \wedge z_{n-1}$.
\end{enumerate}
Again, the first two---$x_{n+1}\leq y_n$ and $y_{n+1} \leq y_{n}$---have already 
been ruled out. So the only possibility is
$x_{n+1} \wedge y_{n+1} \leq x_{n-1} \wedge z_{n-1}$. Let us assume we are in this case.
Taken all together, if 
$x_n \wedge y_n$ were equal to  $x_{n+1} \wedge y_{n+1}$,
then we would have,
\[
x_n \wedge y_n = x_{n+1} \wedge y_{n+1} \leq x_{n-1} \wedge y_{n-1} \wedge z_{n-1}.
\]
In particular $x_n \wedge y_n \leq x_{n-1} \wedge y_{n-1}$.  But this contradicts 
the inductive hypothesis. 

% Finally, to complete the proof, we should show that $m_{k+1} > m_k$ for all $k\geq 0$.

To complete the proof of  Lemma~\ref{claim:2.3}, we must show that
$m_{k+1} > m_k$ for all $k\geq 0$. Recall,
\[
  m_0 = (x \wedge y) \vee (x \wedge z) \vee (y \wedge z), \text{ and }
  \]
\[
  m_1 = (x_1 \wedge y_1) \vee (x_1 \wedge z_1) \vee (y_1 \wedge z_1).
\]
Suppose $m_1 \leq m_0$.  Then, in particular, we would have
\[
x_1 \wedge y_1 \leq  (x \wedge y) \vee (x \wedge z) \vee (y \wedge z).
\]
Then, by Whitman's Theorem, one of the following must hold:
\begin{enumerate}
\item $x_1 \leq  (x \wedge y) \vee (x \wedge z) \vee (y \wedge z)$,
\item $y_1 \leq  (x \wedge y) \vee (x \wedge z) \vee (y \wedge z)$,
\item $x_1 \wedge y_1 \leq  x \wedge y$,
\item $x_1 \wedge y_1 \leq x \wedge z$,
\item $x_1 \wedge y_1 \leq  y \wedge z$.
\end{enumerate}
Expanding the first of these, we would have
\[
x_1 = x \vee (y \wedge z) \leq  (x \wedge y) \vee (x \wedge z) \vee (y \wedge z) = m_0
\]
in this case.  This is easily ruled out since $x$ is join prime, so it cannot lie below $m_0$.
Similarly, the second case is ruled out because $y$ does not lie below $m_0$.
The third case is ruled out by what we showed earlier in the proof.
The fourth case is $x_1 \wedge y_1 \leq x \wedge z$, 
which implies $x_1 \wedge y_1 \leq x$. Since $x$ is meet prime, this puts either $x_1$ or $y_1$ below $x$,
which, as we noted above, is impossible.
The fifth case is $x_1 \wedge y_1 \leq y \wedge z$, 
which implies $x_1 \wedge y_1 \leq y$ and, again, this is impossible since $y$ is meet prime.
Thus we have proved $m_1 > m_0$.  

Finally, fix $k> 0$ and assume $m_k > m_{k-1}$.  We show $m_{k+1} > m_k$.
% \[  m_k = (x_k \wedge y_k) \vee (x_k \wedge z_k) \vee (y_k \wedge z_k).\]
% \[  m_{k+1} = (x_{k+1} \wedge y_{k+1}) \vee (x_{k+1} \wedge z_{k+1}) \vee (y_{k+1} \wedge z_{k+1}).\]
It's clear that $m_{k+1} \geq m_k$, so as usual we must rule out equality.  Suppose
$m_{k+1} \leq m_k$; that is,
\[
(x_{k+1} \wedge y_{k+1}) \vee (x_{k+1} \wedge z_{k+1}) \vee (y_{k+1} \wedge z_{k+1}) \leq
(x_k \wedge y_k) \vee (x_k \wedge z_k) \vee (y_k \wedge z_k).
\]
Then, in particular, $x_{k+1} \wedge y_{k+1} \leq (x_k \wedge y_k) \vee (x_k \wedge z_k) \vee (y_k \wedge z_k)$.
Whitman's Theorem then implies that one of the following must hold:
\begin{enumerate}
\item $x_{k+1}\leq (x_k \wedge y_k) \vee (x_k \wedge z_k) \vee (y_k \wedge z_k)$;
\item $y_{k+1} \leq (x_k \wedge y_k) \vee (x_k \wedge z_k) \vee (y_k \wedge z_k)$;
\item $x_{k+1} \wedge y_{k+1} \leq x_k \wedge y_k$;
\item $x_{k+1} \wedge y_{k+1} \leq x_k \wedge z_k$;
\item $x_{k+1} \wedge y_{k+1} \leq y_k \wedge z_k$.
\end{enumerate}
We rule out the first three cases by the same argument used above.
Suppose (4) holds.  Then, in particular
$x_{k+1} \wedge y_{k+1} \leq z\vee (x_{k-1} \wedge y_{k-1})$.
By (W+) again, one of the following must hold:
\begin{enumerate}
\item $x_{k+1} \wedge y_{k+1} \leq x_{k-1} \wedge y_{k-1}$;
\item $x_{k+1} \leq z\vee (x_{k-1} \wedge y_{k-1})$;
\item $y_{k+1} \leq z\vee (x_{k-1} \wedge y_{k-1})$;
\end{enumerate}
The first two items are impossible since we already proved that for all 
$n\geq 0$, $x_{n+1} \wedge y_{n+1} > x_{n} \wedge y_n$ and  
$x_{n+1} \nleq z_n$ (cf.~the observation immediately after
Claim~\ref{claim:gens-below}). Clearly the same argument can be used 
to rule out Case (5). This completes the proof.
\end{proof}

\noindent \textbf{Proof of Proposition~\ref{prop:2.2}.}
As above, define 
\begin{gather*}
x_1 = x \vee (y \wedge z), \qquad a = 0\vee (1 \wedge 2),\\
y_1 = y \vee (x \wedge z), \qquad b = 1\vee (0 \wedge 2),\\
z_1 = z \vee (x \wedge y), \qquad c = 2\vee (0 \wedge 1),\\
m_1 = (x_1 \wedge y_1) \vee (x_1 \wedge z_1) \vee (y_1 \wedge z_1),\\
d = (a \wedge b) \vee (a \wedge c) \vee (b \wedge c).
\end{gather*}
so that $h(x_1) = a$,  $h(y_1) = b$, $h(z_1) = c$, and $h(m_1) = d$.

Actually, by modularity, since $a\wedge b \leq a$, we have 
\[
(a \wedge b) \vee (a \wedge c) = a \wedge ((a\wedge b) \vee c).
\]
In fact, we claim that $a\wedge b \leq c$, so the last equation becomes 
\[
(a \wedge b) \vee (a \wedge c) = a \wedge c.
\]
Therefore, $d = (a \wedge c) \vee (b \wedge c)$. Similarly, since 
$c \geq b \wedge c$, modularity implies 
\[ 
d = (a \vee (b \wedge c)) \wedge c.
\]
If we also have $b \wedge c \leq a$, then this reduces further to $d = a  \wedge c$.

We now prove that, indeed, $a\wedge b \leq c$, and from this follows both $b \wedge c\leq a$ and $a \wedge c\leq b$, by symmetry. Recall that 
$a \wedge b = \bigl[0\vee (1 \wedge 2)\bigr] \wedge \bigl[1\vee (0 \wedge 2)\bigr]$.
Since $1 \wedge 2 \leq 1\vee (0 \wedge 2)$, modularity implies
$a \wedge b = \bigl\{0\wedge \bigl[1\vee (0 \wedge 2)\bigr]\bigr\} \vee 
(1 \wedge 2)$, and since $0 \geq 0 \wedge 2$, modularity also implies
\[
a \wedge b = (0\wedge 1)\vee (0 \wedge 2) \vee (1 \wedge 2).
\]
Each joinand in the last expression is below $c$,
hence $a\wedge b \leq c$, as claimed.
From what we just proved, not only do we have $d = a \wedge c$, but also 
$d = a\wedge b = a\wedge c = b \wedge c$.

Now, suppose $K \subseteq \alg{F} \times \alg{F}$ is a finite set. 
We wish to prove $\<K\> \neq \ker h$.
First, we show there exists $N < \omega$ such that 
for all $(p, q) \in \<K\>$ 
the following implications hold:
\begin{align}
p \leq x_1 &\implies q \leq x_N; \label{eq:acc01}\\
p \leq y_1 &\implies q \leq y_N; \label{eq:acc02}\\ 
p \leq z_1 &\implies q \leq z_N; \label{eq:acc03}\\
p \leq m_1 &\implies q \leq m_N. \label{eq:acc04}
\end{align}
As we proved above, for each $w \in \{x, y, z, m\}$, the sequence $\{w_n\}$
is an infinite ascending chain, so we can certainly find, for each of the four
implications above, a number $N_w < \omega$ such that the given implication
holds for $N = N_w$ and for all of the (finitely many) pairs in $K$. 
Therefore, all four implications will hold for $(p, q) \in K$ if we take 
$N = \max \{N_a, N_b, N_c, N_d\}$. In fact, we will now see that the same $N$
works for all $(p, q) \in \<K\>$.

\begin{claim}\label{claim:main}
  The implications~(\ref{eq:acc01})--(\ref{eq:acc04}) hold for all $(p,q) \in \<K\>$.
\end{claim}
\begin{proof}
Fix $(u, v) \in  \<K\>$. Obviously, if we manage to prove that the implication~(\ref{eq:acc01}) is satisfied when $(p,q) = (u,v)$, then the same argument will prove $(u,v)$ also satisfies implications~(\ref{eq:acc02}) and~(\ref{eq:acc03}).  So we need only establish the implications~(\ref{eq:acc01}) and~(\ref{eq:acc04}) when $(p,q) = (u,v)$.

In case $(u,v)$ belongs to the finite set $K$, all four implications were already established in the paragraph immediately preceding the statement of Claim~\ref{claim:main}.

Suppose $(u,v) = (p_1, q_1) \wedge (p_2, q_2)$, and assume (the induction hypothesis) that implications~(\ref{eq:acc01})--(\ref{eq:acc04}) are satisfied 
for $(p,q) = (p_i,q_i)$ $i\in \{1,2\}$.
We now prove that, under these assumptions, $(u,v)$ satisfies implications~(\ref{eq:acc01})
and~(\ref{eq:acc04}).

Assume $u = p_1\wedge p_2 \leq x_1$; we must show $v=q_1\wedge q_2 \leq x_N$.
The relation $p_1\wedge p_2 \leq x\vee (y \wedge z)$, along with Whitman's Condition (W+), implies that one of the following holds:
\begin{enumerate}
  \item   $p_1=x$;  
  \item   $p_2=x$;  
  \item   $p_1\leq x_1$;  
  \item   $p_2\leq x_1$;  
  \item   $p_1\wedge p_2 \leq y \wedge z$;  
\end{enumerate}
In the first case we have $p_1 = x \leq x \vee (y \wedge z) = x_1$, so
the induction hypothesis implies $q_1\leq x_N$, so $v = q_1\wedge q_2\leq x_N$, as desired.  Notice that the second, third, and fourth items in the list submit to the same argument, so we are left with the fifth case, 
in which $p_1\wedge p_2 \leq y \wedge z$. 
This implies $p_1\wedge p_2 \leq y$, and $y$ is meet prime, so 
$p_i\leq y$ for some $i\in \{1,2\}$.  But then, $p_i \leq y_1 = y\vee (x\wedge z)$, so by the induction hypothesis, $q_i \leq y_N$.
Similarly, $p_1\wedge p_2 \leq z$ implies 
$p_j\leq z$ for some $j\in \{1,2\}$, so $q_j \leq z_N$.
This yields the desired result,
\[
v = q_1\wedge q_2 \leq y_N \wedge z_N \leq x \vee (y_N\wedge z_N) = x_N.
\] 
Next, suppose $u = p_1\vee p_2 \leq x_1$; we show $v = q_1 \vee q_2 \leq x_N$. The relation $p_1\vee p_2 \leq x_1$ implies $p_1\leq x_1$ and 
$p_2\leq x_1$, so by the induction hypothesis we have 
$q_1\leq x_N$ and 
$q_2\leq x_N$.  Therefore, $v = q_1 \vee q_2 \leq x_N$, as desired.
We have thus shown that $(u,v)$ satisfies the implication~(\ref{eq:acc01}).

Now we turn our attention to implication~(\ref{eq:acc04}).  
Supposing $u \leq m_1$, we must show $v \leq m_N$.

Assume $u  = p_1 \wedge p_2$ and $v = q_1 \wedge q_2$,
where the pairs $(p_i, q_i)$ satisfy
implications~(\ref{eq:acc01})--(\ref{eq:acc04}).
Since $p_1\wedge p_2 \leq m_1 
  = (x_1 \wedge y_1) \vee (x_1 \wedge z_1) \vee (y_1 \wedge z_1)$, 
condition (W+) implies that one of the following holds:
  \begin{enumerate}
    \item   $p_1\leq m_1$;  
    \item   $p_2 \leq m_1$;  
    \item   $p_1\wedge p_2 \leq x_1 \wedge y_1$;  
    \item   $p_1\wedge p_2 \leq x_1 \wedge z_1$;  
    \item   $p_1\wedge p_2 \leq y_1 \wedge z_1$.  
  \end{enumerate}
  Cases~(1) and~(2) are easily handled. Indeed, if there is an $i$ in 
  $\{1, 2\}$ for which $p_i \leq m_1$, then $q_i \leq m_N$ by the induction hypothesis; therefore, 
  $v = q_1 \wedge q_2 \leq q_i \leq m_N$, which is the desired result.

  It suffices to handle Case~(3), since Cases~(4) and~(5) are analogous.
  In Case~(3), we have $p_1\wedge p_2 \leq x_1 \wedge y_1$. In particular, 
  both of the following hold:
  \begin{align}
  p_1\wedge p_2 &\leq x_1 \text{ and }\label{leq:x1}\\
  p_1\wedge p_2 &\leq y_1. \label{leq:y1}
  \end{align}
  Let's first consider $p_1\wedge p_2 \leq x_1 = x \vee (y \wedge z)$.
  This and Whitman's condition (W+) imply that one of the following must hold:
  \begin{enumerate}[(a)]
    \item \label{item:p1genx}  $p_1= x$;
    \item \label{item:p2genx}  $p_2= x$;
    \item \label{item:p1leqx1}  $p_1\leq x_1$;
    \item \label{item:p2leqx1}  $p_2 \leq x_1$;
    \item \label{item:leqyz}  $p_1\wedge p_2 \leq y \wedge z$.
  \end{enumerate}
  % Cases (\ref{item:p1genx}) and~(\ref{item:p2genx}) are symmetric, so we just handle the first of these.  
  Case~(\ref{item:leqyz}) can be handled without appealing to
  constraint~(\ref{leq:y1}), so we dispense with that case first.
  If $p_1\wedge p_2 \leq y \wedge z$, then $p_1 \wedge p_2 \leq y$ and 
  $p_1 \wedge p_2 \leq z$.  Since $y$ and $z$ are both meet prime,
  we have, for some $i, j \in \{1,2\}$, $p_i \leq y$ and $p_j\leq z$.
  But then $p_i \leq y \vee (x\wedge z) = y_1$, so the induction hypothesis
  implies that $q_i\leq y_N$.  Similarly, 
  $p_j\leq z \leq z \vee (x \wedge y) = z_1$ implies $q_j\leq z_N$.  It follows
  that $v = q_1 \wedge q_2 \leq y_N \wedge z_N \leq m_N$, as desired.
   
  Next, we turn to Case~(\ref{item:p1genx}). In this case we have
  $p_1 = x \leq x \vee (y \wedge z) = x_1$, so once we prove the desired result in Case~(\ref{item:p1leqx1}), Case~(\ref{item:p1genx}) will follow.
  Of course, the same comments apply to Cases~(\ref{item:p2genx}) and~(\ref{item:p2leqx1}), by symmetry. 

  In Case~(\ref{item:p1leqx1}), we have
  $p_1\leq x_1$, so the induction hypothesis implies $q_1 \leq x_N$. To 
  conclude that $v \leq m_N$ in this case, we need to exploit condition~(\ref{leq:y1}); that is, 
  $p_1\wedge p_2 \leq y_1 = y \vee (x \wedge z)$. From this and (W+),
  one of the following holds:
  \begin{enumerate}[(i)]
    \item   $p_1=y$;
    \item   $p_2=y$;
    \item   $p_1\leq y_1$;
    \item   $p_2 \leq y_1$;
    \item   $p_1\wedge p_2 \leq x \wedge z$.  
  \end{enumerate}
  Again, the first two of these imply the third and fourth, respectively.
  So we show how to handle the third and fifth cases in this list.
  Keep in mind, we have already shown $q_1 \leq x_N$ at this point.
  
  In the third case, we have $p_1\leq y_1 = y \vee (x\wedge z)$. So the induction hypothesis implies $q_1 \leq y_N$. This and $q_1 \leq x_N$ implies 
  $q_1\leq x_N \wedge y_N \leq m_N$.  Therefore, $v = q_1\wedge q_2 \leq m_N$, as desired.

  Assume we are in the fifth case, that is $p_1\wedge p_2 \leq x \wedge z$.
  Then $p_1\wedge p_2 \leq x$ and $p_1\wedge p_2 \leq z$. Since $x$ and $z$ are both meet prime, we have $p_i\leq x\leq x_1$ and $p_j\leq z\leq z_1$ for some $i, j\in \{1,2\}$.  By the induction hypothesis, then, $q_i \leq x_N$ and $q_j\leq z_N$, whence, $v = q_1 \wedge q_2 \leq x_N \wedge z_N \leq m_N$, 
  as desired.
  
  To complete the proof, we must consider the case in which
  $u  = p_1 \vee p_2$ and 
  $v = q_1 \vee q_2$, where $u \leq m_N$ and where $(p_i,q_i)$ satisfy the same induction hypotheses as above. We must prove that $v  \leq m_N$ under these assumptions.
  If $p_1 \vee p_2 \leq m_1$, then $p_1 \leq m_1$, so $q_1\leq m_N$, but the induction hypothesis.  Similarly, $p_2\leq m_1$, so $q_2\leq m_N$.  Therefore, $v = q_1\vee q_2 \leq m_N$, and the proof of Claim~\ref{claim:main} is finally complete.
\end{proof}

It follows Claim~\ref{claim:main} that 
$\<K\> \neq \ker h$. Indeed, since $\{m_n\}$ is a strictly increasing sequence, 
$m_N < m_{N+1}$, so the pair $(p, q) = (m_1, m_{N+1})$ belongs to $\ker h$, as 
$h(m_1) = d = h(m_{N+1})$, but does not satisfy condition~(\ref{eq:acc04}). On 
the other hand, we just proved that all pairs in $\<K\>$ 
satisfy~(\ref{eq:acc01})--(\ref{eq:acc04}) when $K$ is a finite set.   The 
proof of Proposition~\ref{prop:2.2} is now complete.\qed

%%%%%%%%%%%%%%%%%%%%%%%%%%%%%%%%%%%%%%%%%%%%%%%%%%%%%%%%%%%%%%%%%%%%%%%%%%%
% \newpage
%%%%%%%%%%%%%%%%%%%%%%%%%%%%%%%%%%%%%%%%%%%%%%%%%%%%%%%%%%%%%%%%%%%%%%%%%%%



\subsection{Other Examples}
In each of the propositions in this section, $X$ is a finite set and 
$\mathbf{F} = \mathbf{F}(X)$ is the free lattice generated by $X$.
The symbol $F$ denotes the universe of $\mathbf{F}$.  
The proof in each case is straightforward, but tedious;
we omit proofs of the first two, and give a detailed proof of the third.
%Ex 1.
\begin{prop}\label{prop:1}
Let $X = \{x,y,z\}$, and let $\mathbf{L} = \mathbf{2}$ be the 2-element chain.    
Then the kernel of an epimorphism $h\colon \mathbf{F} \twoheadrightarrow \mathbf{L}$ 
is a finitely generated sublattice of $\mathbf{F} \times \mathbf{F}$.
\end{prop}

\begin{prop}\label{prop:2} 
Let $X = \{x, y, z\}$ and let $\mathbf{L} = \mathbf{3}$ be the 3-element chain.    
Then the kernel of an epimorphism $h\colon \mathbf{F} \twoheadrightarrow \mathbf{L}$ 
is finitely generated.
\end{prop}

\begin{prop}\label{prop:3} 
Let $n > 2$, $X = \{x_0, x_1,\dots, x_{n-1}\}$, and $\mathbf{L} = \mathbf{2} \times \mathbf{2}$.  
Let $h\colon \mathbf{F} \twoheadrightarrow \mathbf{L}$ be an epimorphism. 
Then $K = \operatorname{ker}h$ is finitely generated.  
\end{prop}
\begin{proof} 
Let the universe of $\mathbf{L} = \mathbf{2} \times \mathbf{2}$ be $\{0, a, b, 1\}$, 
where $a\vee b = 1$ and $a\wedge b = 0$. For each $y \in L$, denote by 
$X_y = X \cap h^{-1}\{y\}$ the set of generators mapped by $h$ to $y$. 
Denote the least and greatest elements of $h^{-1}\{y\}$ (if they exist) by $\ell_y$ 
and $g_y$, respectively.  For example, 
$\ell_a = \bigwedge h^{-1}\{a\} = \bigwedge \{x\in F: h(x) = a\},\;$ 
$\quad g_b = \bigvee \{x\in F : h(x) = b\},\quad$ etc. 
In the present example, the least and greatest elements exist is each case, as we now show.

% Claim 3.1.
\begin{subclaim}\label{claim:3.1}
  $h^{-1}\{a\}$ has least and greatest elements, namely 
  $\ell_a = \bigwedge (X_a \cup X_1)$ and $g_a = \bigvee (X_a\cup X_0)$.  
  (Similarly, $h^{-1}\{b\}$ has least and greatest elements, $\ell_b$ and $g_b$.)
\end{subclaim}
\noindent {\it Proof of Subclaim~\ref{claim:3.1}.}
Let $M(a):=\bigwedge (X_a\cup X_1)$ and $J(a):=\bigvee (X_a\cup X_0)$ and 
note that these values exist in $F$, since the sets involved are finite. Also, 
Then $h(M(a)) = a = h(J(a))$. Fix $r \in h^{-1}\{a\}$.  
\begin{itemize}
\item If $r \in X_a$, then $r\geqslant \bigwedge X_a \geqslant \bigwedge (X_a\cup X_1) = M(a)$.

\item If $r = s \vee t$, where $h(s) = a$ and $h(t) \in \{a, 0\}$, then    
assume (the induction hypothesis) that $s \geqslant M(a)$, and we have
$r = s\vee t \geqslant M(a)$.

\item If $r = s \wedge t$, where $h(s) = a$ and $h(t) \in \{a, 1\}$, then 
assume (the induction hypothesis) that $s, t \geqslant M(a)$, and we have 
$s \wedge t \geqslant M(a)$.
This proves that for each $r \in h^{-1}\{a\}$ we have $r \geqslant M(a)$, and 
as we noted at the outset, $M(a)\in h^{-1}\{a\}$. Therefore, $\ell_a = M(a)$ is 
the least element of $h^{-1}\{a\}$. Similarly, every $r \in h^{-1}\{a\}$ is 
below $J(a)$, so $g_a = J(a)$.  The proofs of $\ell_b = M(b)$ and $g_b = J(b)$ 
are similar.
\end{itemize}
This proves Subclaim~\ref{claim:3.1}.


\begin{subclaim}\label{claim:3.2}
$h^{-1}\{0\}$ has least and greatest elements, 
namely, $\ell_0 = \bigwedge X$ and $g_0 = g_a \wedge g_b$.
\end{subclaim}
\noindent {\it Proof of Subclaim~\ref{claim:3.2}.}
$\ell_0 = \bigwedge X$ is obvious, so we need only verify that $g_0 = g_a \wedge g_b$. 
Observe that $h(g_a \wedge g_b) = h(g_a) \wedge h(g_b) = a \wedge b = 0$, so 
$g_a \wedge g_b \in h^{-1}\{0\}$. It remains to prove that $r \leqslant g_a \wedge g_b$ holds for all $r \in h^{-1}\{0\}$.
Fix $r \in h^{-1}\{0\}$. Then $h(r \vee g_a) = h(r) \vee h(g_a) = 0 \vee a = a$, which places $r \vee g_a$ in $h^{-1}\{a\}$.  Therefore, 
by maximality of $g_a$, we have $r \vee g_a  \leqslant g_a$, whence 
$r\leqslant g_a$.  Similarly, $r\leqslant g_b$.
This proves Subclaim~\ref{claim:3.2}.

\begin{subclaim}\label{claim:3.3}
$h^{-1}\{1\}$ has least and greatest elements, namely $\ell_1 = \ell_a \vee \ell_b$ and $g_1 = \bigvee X$.
\end{subclaim}
\noindent {\it Proof of Subclaim~\ref{claim:3.3}.}
  $g_1 = \bigvee X$ is obvious, so we need only verify that $\ell_1 = \ell_a \vee \ell_b$. 
Observe that $h(\ell_a \vee \ell_b) = h(\ell_a) \vee h(\ell_b) = a \vee b = 1$, so 
$\ell_a \vee \ell_b \in h^{-1}\{1\}$. It remains to prove that $r \geqslant \ell_a \vee \ell_b$ holds for all $r \in h^{-1}\{1\}$.
Fix $r \in h^{-1}\{1\}$. Then $h(r \wedge \ell_a) = h(r) \wedge h(\ell_a) = 1 \wedge a = a$, which places $r \wedge \ell_a$ in $h^{-1}\{a\}$.  Therefore, 
by minimality of $\ell_a$, we have $r \wedge \ell_a  \geqslant \ell_a$, whence 
$r\geqslant \ell_a$.  Similarly, $r\geqslant \ell_b$.
Now let $Y = \{(x, g_p), (g_p, x), (x, \ell_p),(\ell_p, x) : p \in \{0, a, b, 1\}, x \in X_p\}$.
This proves Subclaim~\ref{claim:3.3}.

% Claim 3.4.
\begin{subclaim}\label{claim:3.4} 
  If $r \in F$ and $h(r) = p$, then $(r, \ell_p), (r, g_p) \in \langle Y \rangle$.
\end{subclaim}
\noindent {\it Proof of Subclaim~\ref{claim:3.4}.}
Either $r \in X_p$ or $r = s \wedge t$ or $r = s \vee t$.
If $r \in X_p$, then the pair belongs to $Y$ and the claim is trivial.\\[6pt]
Suppose $r = s \wedge t$.
\begin{itemize}
\item If $h(r) = 1$, then $h(s) = h(t) = 1$.  Assume 
(the induction hypothesis) that 
$\{(s, \ell_1), (s, g_1), (t, \ell_1), (t, g_1)\} \subseteq \langle Y \rangle$.
Then $(r, \ell_1) = (s \wedge t, \ell_1) =  (s, \ell_1) \wedge (t, \ell_1) \in \langle Y \rangle$. 
\item If $h(r) = a$, then (wlog) $h(s) = a$ and $h(t) \in \{a, 1\}$.  Assume 
(the induction hypothesis) that 
$\{(s, \ell_a), (s, g_a), (t, \ell_p), (t, g_p)\} \subseteq \langle Y \rangle$. 
By Claim 1, $\ell_a \leqslant \ell_1$, so $\ell_a = \ell_a \wedge \ell_1$.\\
If $h(t) = 1$, then 
\[(r, \ell_a) = (s \wedge t, \ell_a \wedge \ell_1) =  (s, \ell_a) \wedge (t, \ell_1) \in \langle Y \rangle,\]
while If $h(t) = a$, then $(r, \ell_a) = (s \wedge t, \ell_a \wedge \ell_a) =  (s, \ell_a) \wedge (t, \ell_a) \in \langle Y \rangle$.
  
      \item If $h(r) = 0$, then (wlog) that either (i) $h(s) = 0$, or (ii) $h(s) = a$, $h(t)=b$.  
      If $h(s) = 0$, then $(s, \ell_0) \in \langle Y \rangle$ implies 
      $(r, \ell_0) = (s \wedge t, \ell_0) =  (s, \ell_0) \wedge (t, \ell_p) \in \langle Y \rangle$. If 
      If $h(s) = a$, $h(t) = b$, and
      $(s, \ell_a), (t, \ell_b) \in \langle Y \rangle$, then
      $(r, \ell_0) = (s \wedge t, \ell_0) =  (s, \ell_a) \wedge (t, \ell_b) \in \langle Y \rangle$.
  \end{itemize}
  This proves that $(r, \ell_1) \in \langle Y \rangle$ if $r = s \wedge t$.
  A similar argument shows that $(r, g_p) \in \langle Y \rangle$ in each of the three subcases. 
  We have thus proved that $\{(r, \ell_1), (r, g_p)\}\subseteq \langle Y \rangle$, if $r = s \wedge t$.\\[6pt]
  Suppose $r = s \vee t$.
  \begin{itemize}
  \item If $h(r) = 0$, then $h(s) = h(t) = 0$. Assume (the induction hypothesis) that 
  $\{(s, \ell_p), (s, g_p), (t, \ell_p), (t, g_p)\} \subseteq \langle Y \rangle$. Then 
  $(r, \ell_p) = (s \vee t, \ell_p) =  (s, \ell_p) \vee (t, \ell_p) \in \langle Y \rangle$.

  \item If $h(r) = a$, then (wlog) $h(s) = a$ and $h(t) \in \{a, 0\}$.  If we assume (the induction hypothesis) that $(s, \ell_p), (s, g_p), (t, \ell_p), (t, g_p)$ belong to $\langle Y \rangle$, then $(r, \ell_p) = (s \vee t, \ell_p) =  (s, \ell_p) \vee (t, \ell_p) \in \langle Y \rangle$. 

  \item If $h(r) = 1$, then (wlog) that either (i) $h(s) = 1$, or (ii) $h(s) = a$, $h(t)=b$.\\
  In the first case, $(s, \ell_1) \in \langle Y \rangle$ implies 
  $(r, \ell_1) = (s \vee t, \ell_1) =  (s, \ell_1) \vee (t, \ell_p) \in \langle Y \rangle$.
  In the second case $h(s) = a$, $h(t) = b$, and
  $(s, \ell_a), (t, \ell_b) \in \langle Y \rangle$. Then
  $(r, \ell_1) = (s \vee t, \ell_1) =  (s, \ell_a) \vee (t, \ell_b) \in \langle Y \rangle$.

  \end{itemize}
Similarly, in each of these three subcases, we have $(r, g_p) \in \langle Y \rangle$.
This proves Subclaim~\ref{claim:3.4}, and completes the proof of Prop~\ref{prop:3}.
\end{proof}

%%%%%%%%%%%%%%%%%%%%%%%%%%%%%%%%%%%%%%%%%%%%%%%%%%%%%%%%%%%%%%
\bigskip
%\newpage
%%%%%%%%%%%%%%%%%%%%%%%%%%%%%%%%%%%%%%%%%%%%%%%%%%%%%%%%%%%%%%

\section{Background}
The notation, definitions, ideas presented below are based on those 
we learned from the book by Freese, Jezek, Nation~\cite{MR1319815}, 

% Here are some useful definitions and results from the Free Lattices book by Freese, Jezek, and Nation~\cite{MR1319815}.

\begin{definition}[length of a term] Let $X$ be a set. Each element of $X$ is a 
  term of length 1, also known as a \textbf{variable}. 
  If $t_1, \dots, t_n$ are terms of lengths $k_1, \dots, k_n$, 
  then $t_1 \vee \cdots \vee t_n$ and $t_1 \wedge \cdots \wedge t_n$ are 
  both terms of length $1+ k_1 + \cdots + k_n$.
\end{definition}

\noindent \textbf{Examples.} By the above definition, the terms 
\[x \vee y \vee z \qquad x \vee (y \vee z) \qquad (x \vee y) \vee z\]
have lengths 4, 5, and 5, respectively. Reason: variables have length 1, 
so $x \vee y \vee z$ has length $1 + 1 + 1 + 1$.  On the other hand, 
$x\vee y$ is a term of length $3$, so $(x \vee y) \vee z$ has length 
$1 + 3 + 1$. Similarly, $x \vee (y \vee z)$ has length $1 + 1 + 3$.

\begin{lemma}[\protect{\cite[Lem.~1.2]{MR1319815}}]
Let $\mathcal{V}$ be a nontrivial variety of lattices and let $\mathbf{F}_{\mathcal V}(X)$ be the relatively free lattice in $\mathcal V$ over $X$.  Then,
\begin{equation}
  \label{eq:star2}
\bigwedge S \leqslant \bigvee T \text{ implies }S \cap T \neq \emptyset
\text{ for each pair of finite subsets $S, T \subseteq X$.}
\end{equation}
\end{lemma}

\begin{lemma}[\protect{\cite[Lem.~1.4]{MR1319815}}]\label{lem:prime}
Let $\mathbf L$ be a lattice generated by a set $X$ and let $a \in L$.
\begin{enumerate}
\item 
If $a$ is join prime, then $a = \bigwedge S$ for some finite subset $S \subseteq X$.
\item 
If $a$ is meet prime, then $a = \bigvee S$ for some finite subset $S \subseteq X$.

If $X$ satisfies condition~(\ref{eq:star2}) above, then 
\item for every finite, nonempty subset $S \subset X$, $\bigwedge S$ is join prime and $\bigvee S$ is meet prime.
\end{enumerate}
\end{lemma}

\begin{corollary}[\protect{\cite[Cor.~1.5]{MR1319815}}]
Let $\mathcal V$ be a nontrivial variety of lattices and let $\mathbf{F}_{\mathcal V}(X)$ be the relatively free lattice in $\mathcal V$ over $X$.  For each finite nonempty subset $S \subseteq X$, $\bigwedge S$ is join prime and $\bigvee S$ is meet prime. In particular, every $x\in X$ is both join and meet prime.  Moreover, if $x\leqslant y$ for $x, y \in X$, then $x = y$.
\end{corollary}

\begin{theorem}[Whitman's Condition, ver.~1]  
The free lattice $\mathbf{F}(X)$ satisfies the following condition:

(W)  If $v = v_1 \wedge \cdots \wedge v_r \leqslant u_1 \vee \cdots \vee u_s = u$, 
then either $v_i \leqslant u$ for some $i$, or $v \leqslant u_j$ for some $j$. 
\end{theorem}

\begin{corollary}[\protect{\cite[Cor.~1.9]{MR1319815}}]
Every sublattice of a free lattice satisfies (W). Every element of a lattice satisfying (W) is either join or meet irreducible.
\end{corollary}

\begin{theorem}[Whitman's Condition, ver.~2]  \label{thm:whitman2}
  The free lattice $\mathbf{F}(X)$ satisfies the following condition:
  
  (W+)  If $v = v_1 \wedge \cdots \wedge v_r \wedge x_1 \wedge \cdots
  \wedge x_n \leqslant u_1 \vee \cdots \vee u_s \vee
  y_1 \vee \cdots \vee y_m = u$, where $x_i, y_j\in X$, then either 
  $x_i = y_j$ for some $i$ and $j$, or $v_i \leqslant u$ for some $i$, or
  $v \leqslant u_j$ for some $j$. 
  \end{theorem}
  
\begin{theorem}[\protect{\cite[Thm.~1.11]{MR1319815}}]
  \label{thm:wordprob}
If $s = s(x_1, \dots, x_n)$ and $t = t(x_1, \dots, x_n)$ are terms and $x_1, \dots, x_n \in X$, then the truth of 
\begin{equation}
  \label{eq:ast}
s^{\mathbf{F}(X)} \leqslant t^{\mathbf{F}(X)}
\end{equation}
can be determined by applying the following rules.
\begin{enumerate}
\item If $s=x_i$ and $t=x_j$, then (\ref{eq:ast}) holds iff $x_i = x_j$.
\item If $s = s_1 \vee \dots \vee s_k$ is a formal join, then (\ref{eq:ast}) holds iff $s_i^{\mathbf{F}(X)} \leqslant t^{\mathbf{F}(X)}$ for all $i$.
\item If $t = t_1 \wedge \dots \wedge t_k$ is a formal meet, then (\ref{eq:ast}) holds iff 
$s^{\mathbf{F}(X)} \leqslant t_i^{\mathbf{F}(X)}$ for all $i$.
\item If $s = x_i$ and $t = t_1 \vee \dots \vee t_k$ is a formal join, 
   then (\ref{eq:ast}) holds iff $x_i \leqslant t_j^{\mathbf{F}(X)}$ for some $j$.
\item If $s = s_1 \wedge \dots \wedge s_k$ is a formal meet and $t = x_i$, then (\ref{eq:ast}) holds iff $s_j^{\mathbf{F}(X)} \leqslant x_i$ for some $j$.
\item If $s = s_1 \wedge \dots \wedge s_k$ is a formal meet and 
and $t = t_1 \vee \dots \vee t_m$ is a formal join, then (\ref{eq:ast}) holds iff 
$s_i^{\mathbf{F}(X)} \leqslant t^{\mathbf{F}(X)}$ for some $i$
or $s^{\mathbf{F}(X)} \leqslant t_j^{\mathbf{F}(X)}$ for some $j$
\end{enumerate}
\end{theorem}

\begin{theorem}[\protect{\cite[Thm.~1.17]{MR1319815}}] For each $w \in \alg{F} (X)$ 
  there is a term of minimal length representing $w$, unique up to commutativity. 
  This term is called the \textbf{canonical form} of $w$.
\end{theorem}


Let $w \in \alg{F}(X)$ be join reducible and suppose 
$t = t_1 \vee \cdots \vee t_n$ (with $n > 1$) is the
canonical form of $w$. 
Let $w_i = t_i^{\alg{F}(X)}$.  Then $\{w_1, \dots, w_n\}$ are called the 
\textbf{canonical joinands} of $w$. We also say $w = w_1 \vee \dots \vee w_n$
\textit{canonically} and that  $w = w_1 \vee \dots \vee w_n$
is the \textbf{canonical join representation} of $w$. 
If $w$ is join irreducible, we define the canonical joinands of $w$ to be the set 
$\{w\}$. Of course the \textbf{canonical meet representation} and 
\textbf{canonical meetands} of an element in a free lattice are defined dually. 

A join representation $a = a_1 \vee \cdots \vee a_n$ in an arbitrary lattice is 
said to be a \textbf{minimal join representation} if 
$a = b_1 \vee \cdots \vee b_m$  and $\{b_1, \dots, b_m\} \ll \{a_1, \dots, a_n\}$
imply $\{a_1, \dots, a_n\} \subseteq \{b_1, \dots, b_m\}$. Equivalently, 
a join representation is minimal if it is an antichain and nonrefinable.

\begin{theorem}[\protect{\cite[Thm.~1.19]{MR1319815}}] \label{thm:1.19}
Let $w = w_1 \vee \dots \vee w_n$ canonically in $\alg{F}(X)$.
If also $w = u_1 \vee \dots \vee u_m$, then 
$\{w_1, \dots, w_n\} \ll \{u_1, \dots, u_m\}$.
Thus $w = w_1 \vee \cdots \vee w_n$ is the unique minimal join representation of $w$.
\end{theorem}

\begin{theorem}[\protect{\cite[Thm.~1.20]{MR1319815}}] 
Let $w \in \alg{F} (X)$ and let $u$ be a join irreducible element in
$\alg{F}(X)$. Then $u$ is a canonical joinand of $w$ if and only if there 
is an element $a$ such that $w = u \vee a$ and $w > v \vee a$ for every $v < u$.
\end{theorem}

\begin{definition}[up directed, continuous] A subset $A$ of a lattice $L$ is said 
  to be \textbf{up directed} if every finite subset of $A$ has an upper bound in $A$.
  It suffices to check this for pairs.  $A$ is up directed iff for all $a, b \in A$
  there exists $c\in A$ such that $a\leqslant c$ and $b\leqslant c$.  
  A lattice is \textbf{upper continuous} if whenever $A\subseteq L$ is an up 
  directed set having a least upper bound $u = \bigvee A$, then for every $b$,
  \[\bigvee_{a\in A} (a \wedge b) = 
  \bigvee_{a\in A} a \wedge b =  u \wedge b.\]
\noindent  \textbf{Down directed} and \textbf{down continuous} are defined dually.  A lattice that is 
  both up and down continuous is called \textbf{continuous}.
\end{definition}

\begin{theorem}[\protect{\cite[Thm.~1.22]{MR1319815}}]
  Free lattices are continuous.
\end{theorem}


\subsection{Bounded Homomorphisms}
\label{sec:introduction}
We continue to follow~\cite{MR1319815} very closely,
although the authors of that book indicate that the ideas in this 
subsection have their roots in Ralph McKenzie's work on nonmodular 
lattice varieties~\cite{MR0313141}, and Bjarni J\'onsson's work on 
sublattices of free lattices~\cite{MR0472614}.

If $x, y$ are elements of a lattice $L$, and if $x \leq y$, then we write
$\llbracket x, y \rrbracket$ to denote the sublattice of elements between $x$ and $y$.
That is, 
\[
\llb x, y \rrb := \{z \in L \mid x \leq z \leq y \}.
\]

Let $\mathbf K$ and $\mathbf L$ be lattices and suppose $\mathbf L$ has bottom and top elements, 
$0_{\mathbf L}$ and $1_{\mathbf L}$, resp.  
If $h \colon \mathbf K \to \mathbf L$ is a lattice homomorphism, then for each
$a \in L$ we consider the sets $h^{-1}\llb a, 1 \rrb = \{x \in K \mid h(x) \geq a\}$ and
$h^{-1}\llb 0, a \rrb = \{x \in K \mid h(x) \leq a \}$. 
When $h^{-1}\llb a, 1 \rrb$ is nonempty,
it is a filter of $\mathbf K$; dually a nonempty $h^{-1}\llb 0, a \rrb$ is an ideal. 
If $K$ is infinite, then 
$h^{-1}\llb a, 1 \rrb$ need not have a least element, nor 
$h^{-1}\llb 0, a \rrb$ a greatest element. However, 
considering when such extrema exist leads to the notion of
bounded homomorphism, which in turn helps us understand the structure of free lattices.


A lattice homomorphism $h \colon \mathbf K \to \mathbf L$ is \textbf{lower bounded} if for every 
$a \in L$, the set $h^{-1} \llb a, 1 \rrb$ is either empty or has a least element. 
The least element of a nonempty $h^{-1}\llb a, 1\rrb$ is denoted by $\beta_h(a)$, or  
by $\beta(a)$ when $h$ is clear from context.
Thus, if $h$ is a lower bounded homomorphism, then 
$\beta_h \colon \mathbf L \rightharpoonup \mathbf K$ 
is a partial mapping whose domain is an ideal of $\mathbf L$. 

Dually, $h$ is an \textbf{upper bounded} homomorphism if, whenever the set
$h^{-1}\llb 0, a \rrb$ is nonempty, it has a greatest element, denoted by 
$\alpha_h(a)$, or $\alpha (a)$.
For an upper bounded homomorphism, the domain of 
$\alpha_h \colon \mathbf L \rightharpoonup \mathbf K$ 
is clearly a filter of $\mathbf L$.  
A \textbf{bounded} homomorphism is one that is both upper and lower bounded.

These definitions simplify when $h$ is an epimorphism.
In that case $h$ is lower bounded if and only if each preimage 
$h^{-1}\{ a \}$ has a least element.
Likewise, if $L$ is finite, then $h \colon \mathbf K \to \mathbf L$ is lower 
bounded if and only if $h^{-1}\{a\}$ has a least element whenever it is nonempty. 
On the other hand, every homomorphism $h$ from a finite lattice $\mathbf K$ is bounded.

Note that $\beta$  is monotonic and a left adjoint for $h$, i.e., $a \leq h(x)$ iff
$\beta (a) ≤ x$. It then follows from a standard argument that $\beta$  is a join
preserving map on its domain: if $h^{-1}\llb a, 1 \rrb \neq \emptyset$ and
$h^{-1}\llb b, 1 \rrb \neq \emptyset$, then
$\beta (a \vee b) = \beta (a) \vee \beta (b)$.
Similarly, $\alpha$  is a right adjoint for $h$, so that $h(y) \leq a$ iff
$y \leq \alpha (a)$, and for $a, b \in \dom \alpha$,
$\alpha (a \wedge b) = \alpha (a) \wedge \alpha (b)$.
In particular, if $h$ is an epimorphism, then $\alpha$  and $\beta$  are respectively
meet and join homomorphisms of $\mathbf L$ into $\mathbf K$.
For future reference, we note that $\alpha$  and $\beta$  behave correctly with
respect to composition.

\begin{theorem}[\protect{\cite[Thm.~2.1]{MR1319815}}]
  Let $f \colon  \mathbf K \to \mathbf L$ and $g \colon \mathbf L \to \mathbf M$ be
  homomorphisms. If $f, g$ are lower bounded, then $gf \colon  \mathbf K \to  \mathbf M$ is lower
  bounded and $\beta_{gf} = \beta_f \beta_g$. Similarly, if $f, g$
  are upper bounded, then $\alpha_{gf} = \alpha_f \alpha_g$.
\end{theorem}
\begin{proof}  For $x \in K$ and $a \in M$, we have
  \[ a \leq gf (x) \quad \text{ iff } \quad \beta_g(a) \leq f (x)
  \quad \text{ iff } \quad \beta_f \beta_g (a) \leq x.
  \]
  The upper bounded case is dual.
\end{proof}

We need a way to determine whether a lattice homomorphism
$h\colon \mathbf K \to \mathbf L$
is upper or lower bounded. The most natural setting for this is
when the lattice $\mathbf K$ is finitely generated, so from now on 
we assume $\mathbf K$ is generated by a finite set $X$.
% There are no special assumptions about $\mathbf L$, nor do we assume that
% $h$ is upper or lower bounded. 
We want to analyze the sets $h^{-1}\llb a, 1 \rrb$
for $a \in L$, with the possibility of lower boundedness in mind.
(The corresponding results for
$h^{-1}\llb 0, a\rrb$
are obtained by duality.) Note that $\mathbf K$ has a greatest element 
$1_{\mathbf K} = \bigvee X$, and that $h^{-1}\llb a, 1 \rrb$ is nonempty 
if and only if $a \leq h(1_{\mathbf K})$.

Define a pair of closure operators, 
denoted by $^\wedge$ and $^\vee$, on  subsets of an arbitrary 
lattice %$\mathbf L = \langle L, \vee, \wedge\rangle$ as follows: for each 
$L$ as follows: for each $A\subseteq L$,
\[
A^\wedge := \{\bigwedge B \mid B \text{ is a finite subset of } A\}.
\]
We adopt the following convention: if 
$\mathbf L$ has a greatest element 
$1_{\mathbf L}$, then $\bigwedge \emptyset = 1_{\mathbf L}$, and we include 
this in $A^\wedge$ for every $A \subseteq L$; otherwise, $\bigwedge \emptyset$ is undefined.  
The set $A^\vee$ is defined dually.
% If $\mathbf K = \langle K, \vee, \wedge\rangle$ is a lattice generated by a 
% finite set $X$, then 

We can write $K$ as the union of a chain of subsets 
$H_0\subseteq H_1 \subseteq \cdots$ defined inductively by setting 
$H_0 := X^\wedge$ and $H_{k+1} := (H_k)^{\vee \wedge}$, for all $k\geqslant 0$. 
By induction, each $H_n = X^{\wedge(\vee\wedge)^n}$ is a finite meet-closed 
subset of $K$, and $\bigcup H_n = K$, since $X$  generates $K$.

Let $h \colon \mathbf K \to \mathbf L$ be an epimorphism and, 
for each $y \in L$ and $k< \omega$, define
\[
\beta_k(y) = \bigwedge \{w \in H_k : h(w) \geqslant y\}.
\]

\begin{theorem}[\protect{\cite[Thm.~2.2]{MR1319815}}]\label{thm:2.2}
   Let $\mathbf K$ be finitely generated, and let $h \colon \mathbf K \to \mathbf L$ be a lattice
  homomorphism. If $a \leq h(1_{\mathbf K})$, then
\begin{enumerate}
\item $j \leq k$ implies $\beta_j(a) \geq \beta_k(a)$,
\item  $\beta_k(a)$ is the least element of $H_k \cap h^{-1}\llb a, 1 \rrb$,
\item  $h^{-1}\llb a, 1 \rrb = \bigcup_{k\in \omega}\llb \beta_k(a), 1 \rrb$.
\end{enumerate}  

\end{theorem}  

%%%%%%%%%%%%%%%%%%%%

\subsubsection{Minimal join covers and refinement}

A \textbf{join cover} of the element $a \in L$ is a finite
subset $S \subseteq L$ such that $a \leq \bigvee S$. 
A join cover $S$ of $a$ is \textit{nontrivial} if $a \nleq s$ for all
$s \in S$. Let $\mathcal C (a)$ be the set of all nontrivial join 
covers of $a$ in $\mathbf L$.

\begin{theorem}[\protect{\cite[Thm.~2.3]{MR1319815}}]\label{thm:2.3}
Suppose $\mathbf K$ is generated by a finite set $X$,
$h \colon \mathbf K \to \mathbf L$ is a homomorphism, $a \leq h(1_{\mathbf K})$, and $k\in \omega$. Then,
\begin{align*}
    \beta_0(a) &= \bigwedge \{x \in X \mid h (x) \geq a\},\\
    \beta_{k+1}(a) &= \beta_0(a) \wedge 
    \bigwedge_{\stackrel{S \in \mathcal C(a)}{\bigvee S \leq h(1_{\mathbf K})}}
    \bigvee_{s \in S} \beta_k(s).
\end{align*}
\end{theorem}
In general, the expression for $\beta_{k+1}(a)$ has some redundant terms, which we can 
exclude if $\mathbf L$ satisfies a weak finiteness condition that we now define.
For finite subsets $A$, $B \subseteq L$, we say $A$ \textit{join refines
$B$} and write $A \ll B$ if for every $a \in A$ there exists $b \in B$ 
with $a \leq b$. Theorem~\ref{thm:1.19} states that if $w \in \alg{F}(X)$ and 
$w = \bigvee B$, then the set of canonical joinands of $w$ join
refines $B$. 

Define a \textbf{minimal nontrivial join cover} of $a \in L$ to be a 
nontrivial join cover $S$ with the property that whenever 
$a\leq \bigvee T$ and $T \ll S$, then
$S \subseteq T$. This formulation is equivalent to our more 
intuitive notion of what minimality ought to mean: \textit{a nontrivial join cover $S$ of a is minimal 
if and only if}
\begin{enumerate}
\item \textit{$S$ is an antichain of join irreducible elements of $L$, and}
\item \textit{if an element of $S$ is deleted or replaced by a (finite) 
  set of strictly smaller elements, then the resulting set is no longer 
  a join cover of $a$.}
\end{enumerate}

Let $\mathcal M(a)$ denote the set of minimal nontrivial join covers of $a \in L$.
Let us say that $\mathbf L$ has the \textbf{minimal join cover refinement property} if for each
$a \in L$, $\mathcal M(a)$ is finite and every nontrivial join cover of $a$ refines 
to a minimal one. Clearly every finite lattice has the minimal join cover refinement 
property, but so do free lattices.  The following reformulation of Theorem~\ref{thm:2.3} 
simplifies the calculation of $\beta_k$ whenever the minimal join cover refinement property holds.

\begin{theorem}[\protect{\cite[Thm.~2.4]{MR1319815}}]\label{thm:2.4}
Let $\mathbf K$ be generated by the finite set $X$, and let 
$h \colon \mathbf K \to \mathbf L$ be a lattice homomorphism. 
If $\mathbf L$ has the minimal join cover refinement property, then
for each $a \in L$ with $a \leq h(1_{\mathbf K})$ and $k\in \omega$, we have
\begin{align*}
    \beta_0(a) &= \bigwedge \{x \in X \mid h (x) \geq a\},\\
    \beta_{k+1}(a) &= \beta_0(a) \wedge 
    \bigwedge_{\stackrel{S \in \mathcal M (a)}{\bigvee S \leq h(1_{\mathbf K})}}
    \bigvee_{s \in S} \beta_k(s).
\end{align*}
\end{theorem}

We now look for a condition on the $\mathbf L$ that will insure the 
homomorphism $h \colon \mathbf K \to \mathbf L$ is lower bounded. From 
Theorem~\ref{thm:2.2}, this will happen iff for each 
$a \leq h(1_{\mathbf K})$ there exists $N \in \omega$ such
that $\beta_n(a) = \beta_N(a)$ for all $n \geq N$. In this case, 
$\beta(a) = \beta_N(a)$ for all $a\in \dom \beta = \llb 0, h(1_{\mathbf K})\rrb$, 
where $N$ depends upon $a$.

\begin{fact}
\label{fact:1}
The following are equivalent:
\wjd{To do: verify this!}
\begin{enumerate}
  \item $h$ is not lower bounded;
  \item $(\exists y_0 \in L)(\forall N)(\exists n> N)\, \beta_n(a) \neq \beta_N(a)$;
  \item $(\exists y_0 \in L)(\exists N)(\forall n> N) \beta_n(a) \neq \beta_N(a)$.
\end{enumerate}
\end{fact}

Let $D_0(\mathbf L)$ be the set of all join prime elements of $\mathbf L$, 
i.e., the set of elements that have no nontrivial join cover. 
Given $D_k(\mathbf L)$, define $D_{k+1}(\mathbf L)$ to be the set of $p \in L$ 
such that every nontrivial join cover of $p$ refines to a join cover 
contained in $D_k(\mathbf L)$, i.e., $p \leq \bigvee S$ nontrivially implies 
there exists $T \ll S$ with $p \leq \bigvee T$ and $T \subseteq D_k(\mathbf L)$. 
Note that if $L$ has the minimal join cover refinement property, then 
$p \in D_{k+1}(\mathbf L)$ iff every minimal nontrivial join
cover of $p$ is contained in $D_k(\mathbf L)$.

The definition clearly implies 
$D_0(\mathbf L) \subseteq D_1(\mathbf L) \subseteq D_2(\mathbf L) \subseteq \cdots$.  
Let $D(\mathbf L) = \bigcup D_i$. For $a \in D(\mathbf L)$, define the $D$-\textbf{rank}, $\rho(a)$, 
to be the least integer $N$ such that $a \in D_N(\mathbf L)$; for $a \notin D(L)$, 
$\rho (a)$ is undefined. 
The duals of $D_k(\mathbf L)$, $D(\mathbf L)$, and $\rho(a)$ are denoted by $D^d_k(\mathbf L)$, 
$D^d(\mathbf L)$, and $\rho^d(a)$, respectively.

% We are interested in the property D(L) = L, i.e., when every element is in
% Dfc(L) for some k. If A is a finite subset of Dk(L), then \J A G D/c+i(L). Indeed, if
% V A < \/ S for some set 5, then, for a G A, S is a join cover of a. Since a G D/C(L),
% if this is a nontrivial join cover, there is a set Ua C D/c_i(L) with Ua <^ S and
% a < \jUa- For those a's for which 5 is a trivial join cover, set Ua = {a}. Let
% U = \jUa- Then U C D/C(L) and is a join cover of V-^ and U <C 5, which
% shows that \J A is in D/c+i(L). Hence D(L) is a join subsemilattice of L. On the
% other hand, it is easy to see that in a lattice with the minimal join cover refinement
% property, every element is a join of join irreducibles. Combining these observations,
% we obtain the following useful equivalence.
% Lemma 2.5. For a lattice with the minimal join cover refinement property,
% D(L) = L if and only if J(L) C D(L).
% For example, if K is a finite distributive lattice, then D(K) = Di(K) = K
% because every join irreducible element is join prime. Likewise, it is easy to see that
% for the pentagon N5 we have D(N5) = Di(Ns) = AT5. On the other hand, for the
% diamond we have D(M3) = {0} ^ M3.
% We need to analyze the property D(L) = L at length, but first let us show that
% it does what we need it to do.
% Theorem 2.6. Let K be a finitely generated lattice and let h : K —► L be a
% homomorphism. If D(L) = L, then h is lower bounded.
% The proof of this theorem yields a slightly stronger statement which, because
% we will refer to it a couple of times, we give as a separate lemma.
% Lemma 2.7. Let K be a finitely generated lattice and let h : K —► L be a
% homomorphism. For all w G K and a G Dn(L),
% (*) h(w) > a if and only if w > (3n(a)>
% Thus in the case when D(L) = L, we obtain that h is lower bounded with
% /3(a) = /?p(a)(a) for all a G L.
% Proof of Lemma 2.7. We of course use Theorem 2.3. In order to prove (*)
% by induction, for each nGw define
% Wn = {w G K : for all a G Dn(L), h(w) > a iff w > 0n(a)}.
% We need to show that Wn — K for all n. Clearly the generating set X for K
% is contained in Wn, and Wn is closed under meets for each n. So it remains to
% show that Wn is closed under joins. For n = 0 this follows immediately from the
% definition of D0(L) as the set of all join prime elements.
%%%%%%%%%%%%%%%%%%%%%%%%%%%%%%%%%%%%%%%%%%%%%%%%%%%%%%%%%%%%%%%%%%%%%%%%%%%%%%%%%%%%%%%%%%%%%%
%\newpage
%%%%%%%%%%%%%%%%%%%%%%%%%%%%%%%%%%%%%%%%%%%%%%%%%%%%%%%%%%%%%%%%%%%%%%%%%%%%%%%%%%%%%%%%%%%%%%

%%%%%%%%%%%%%%%%%%%%%%%%%%%%%%%%%%%%%%%%%%%%%%%%%%%%%%%%%%%%%%%%%%%%%%%%%%%%%%%%%%%%%%%%%%%%%%
%\newpage
%%%%%%%%%%%%%%%%%%%%%%%%%%%%%%%%%%%%%%%%%%%%%%%%%%%%%%%%%%%%%%%%%%%%%%%%%%%%%%%%%%%%%%%%%%%%%%







































\section{Proof of Conjecture under special assumptions}\label{app:conjecture}
As usual, let $X$ be a finite set and let $\alg{F} := \alg{F} (X)$ be the free 
lattice generated by $X$.
\begin{proposition}\label{prop:conjecture}
Suppose $\mathbf L = \langle L, \wedge, \vee\rangle$ is a finite lattice and 
$h\colon \mathbf{F} \twoheadrightarrow \mathbf{L}$ a lattice epimorphism.
Suppose also that there is a class of $\ker h$ containing an infinite descending 
chain as well as a join prime element of $\alg{F}$. 
Then the kernel of $h$ is not a finitely generated sublattice 
of $\alg{F} \times \alg{F}$.
\end{proposition}
\begin{proof} Let $y_0\in L$ and suppose 
  $x_0 \in h^{-1}\{y_0\}$ is a join prime element of $\alg{F}$. 
  Suppose also that the class $h^{-1}\{y_0\}$ contains an 
  infinite descending chain, $\beta_0(y_0) > \beta_1(y_0) > \cdots$.
  Let $K$ be a finite subset of $\ker h$, say, 
  $K = \{(p_1, q_1), \dots, (p_{m}, q_{m})\} \subseteq \ker h$.
  We prove $\langle K \rangle \neq \ker h$. 
  Since $K$ is an arbitrary finite subset of $\ker h$, this will prove 
  that $\ker h$ is not finitely generated.

\begin{claim}
  \label{claim:1}
There exists $N<\omega$ such that for all $(p_i, q_i)$ in $K$, if $p_i \geqslant x_0$, then $q_i \geqslant \beta_N (y_0)$.
\end{claim}
\begin{proof}
Fix $i$ and $(p_i, q_i) \in K$ (so, $h(p_i) = h(q_i)$). Define $N_i$ according 
to which of the following cases holds:
\begin{enumerate}
\item If $p_i \ngeqslant x_0$, let $N_i = 0$.  
\item If $p_i\geqslant x_0$, then $x_0 = x_0\wedge p_i$, so 
$y_0 = h(x_0) = h(x_0) \wedge h(p_i) \leqslant h(p_i)$, so $y_0\leqslant h(q_i)$. 
Also, $h(x_0 \wedge q_i) = h(x_0) \wedge h(q_i) = y_0$, so $x_0\wedge q_i \in h^{-1}\{y_0\}$. 
Therefore (since $\{\beta_i(y_0)\}$ is an infinite descending chain),
there exists $n_i>0$ such that $x_0 \wedge q_i \geqslant\beta_{n_i}(y_0)$. 
Let $N_i = n_i$ in this case (so $q_i \geqslant \beta_{N_i}(y_0)$).
\end{enumerate}
After defining $N_i$ this way for each pair $(p_i,q_i)\in K$, 
let $N := \max_i N_i$. % : 1 \leqslant i \leqslant m\}$. 
Then the desired implication holds for all $1\leqslant i \leqslant m$, that is,
\begin{equation}
\label{eq:star3}    
p_i \geqslant x_0 \quad \Longrightarrow \quad q_i \geqslant \beta_N(y_0),
\end{equation}
so Claim~\ref{claim:1} is proved.
\end{proof}

\begin{claim}
  \label{claim:2}
  There exists $N < \omega$ such that, for all $(p, q) \in \langle K \rangle$,
\begin{equation}
  \label{eq:claim2}
p \geqslant x_0 \quad \Longrightarrow \quad q \geqslant \beta_N(y_0).
\end{equation}
\end{claim}
\begin{proof} Define $N$ as in the proof of Claim~\ref{claim:1}, so that 
for all $(p_i,q_i) \in K$ the implication~(\ref{eq:star3}) holds. Fix $(p, q) \in \langle K \rangle$. We prove~(\ref{eq:claim2}) by induction on the complexity of $(p, q)$.  If $(p, q) \in K$, then there's nothing to prove.
We split the induction step into two cases.
\begin{enumerate}
\item Assume $(p, q) = (p_1, q_1) \wedge (p_2, q_2)$, where $p_i$, $q_i$ ($i = 1, 2$) satisfy~(\ref{eq:claim2}).  Assume $p\geqslant x_0$. %% (We show $q \geqslant \beta_N(y_0)$.)
  Then $p = p_1 \wedge p_2 \geqslant x_0$, so $p_1 \geqslant x_0$ and $p_2 \geqslant x_0$, so (by the induction hypothesis) $q_1\geqslant \beta_N(y_0)$ and $q_2\geqslant \beta_N(y_0).$ Therefore, $q = q_1 \wedge q_2 \geqslant \beta_N(y_0),$ as desired.
\item Assume $(p, q) = (p_1, q_1) \vee (p_2, q_2)$, where $p_i$, $q_i$ ($i = 1, 2$) 
satisfy~(\ref{eq:claim2}). Assume $p\geqslant x_0$. Then $p = p_1 \vee p_2 \geqslant x_0$.  
Since $x_0$ is join prime, either $p_1 \geqslant x_0$ or $p_2 \geqslant x_0$.  
Assume (wlog) $p_1 \geqslant x_0$. Then, by the induction hypothesis, $q_1\geqslant \beta_N(y_0)$.
Therefore, $q = q_1 \vee q_2 \geqslant q_1 \geqslant \beta_N(y_0),$ as desired.
\end{enumerate}
This completes the proof of Claim~\ref{claim:2}.
\end{proof}
We can now see that $K$ does not generate $\ker h$. Indeed, 
% $\beta_0(y_0) > \beta_1(y_0) > \cdots$ is an infinite descending chain, and
if $N$ is as in the proof of Claim~\ref{claim:2}, then $\beta_{N}(y_0) > \beta_{N+1}(y_0)$, 
so the pair $(p, q) = (x_0, \beta_{N+1}(y_0))$ does not 
satisfy condition~\ref{eq:claim2}, so does not belong to $\langle K\rangle$.
Yet, $(x_0, \beta_{N+1}(y_0)) \in \ker h$. Thus, $K$ does not generate $\ker h$.  
% Since $K$ was an arbitrary finite subset of $\ker h$, this completes the proof of 
% Proposition~\ref{prop:conjecture}.
\end{proof}

%%%%%%%%%%%%%%%%%%%%%%%%%%%%%%%%%%%%%%%%%%%%%%%%%%%%%%%%%%%%%%%%%%%%%%%%%%%%%%%%%%%%%%%%%%%%%%
\bigskip
%\newpage
%%%%%%%%%%%%%%%%%%%%%%%%%%%%%%%%%%%%%%%%%%%%%%%%%%%%%%%%%%%%%%%%%%%%%%%%%%%%%%%%%%%%%%%%%%%%%%


\begin{comment}
If $(t,s) \in \<K\>$, then the length of $t$ is at least $m$, and if 
the length of $t$ is exactly $m$, then $t$ can only be a meet of 
first coordinates of generators; that is, $t = p_1 \wedge \cdots \wedge p_j$
where $(p_1, q_1), \dots, (p_n, q_j) \in K$ for some $q_1, \dots, q_j$,
and where all $p_i$ have length $m$. 
In this case, we want to show $s\geq \beta_{m+n}h(t)$.

Since all $p_i$ have length $m$, we have $p_i \geq \beta_{m}h(p_i)$ for all 
$1\leq i\leq j$. Therefore,
\[
t = p_1 \wedge \cdots \wedge p_j \geq \beta_{m}h(p_1)\wedge \cdots \wedge \beta_{m}h(p_j).
\]
Also, since $(p_i, q_i) \in K$ for all $1\leq i\leq j$, we have 
$q_i\geq \beta_{m+n}h(p_i)$ for all $1\leq i\leq j$.

We have
$h(t) = h(p_1 \wedge \cdots \wedge p_j) = h(p_1) \wedge \cdots \wedge h(p_j)$, 
so $h(p_i) \geq h(t)$ for all $1\leq i\leq j$.

Let $q = 
q_1 \wedge \cdots \wedge q_j$. Then $(t,q) \in \<K\>$ and,
by our choice of $n$, we have $q_i\geq \beta_{m+n}h(p_i)$ for all $1\leq i\leq j$.
It is not necessarily the case that $q = s$, however, since $(t,q)$ and $(t,s)$ 
both belong to $\<K\>$, we do have $h(q) = h(t) = h(s)$.
Also, $q_i\geq \beta_{m+n}h(p_i)$ ($1\leq i\leq j$) implies
\[
q_1 \wedge \cdots \wedge q_j \geq \beta_{m+n}h(p_1) \wedge \cdots \wedge \beta_{m+n}h(p_j).
\]
$t = p_1 \wedge \cdots \wedge p_n$
where $(p_1, q_1), \dots, (p_n, q_n) \in K$ for some $q_1, \dots, q_2$.
% 
% In that case, we could have $(t,s) = (t, q_1) \wedge (t, q_2)$
% or $(t,s) = (t, q_1) \vee (t, q_2)$, where $(t, q_i) \in K$.
% Notice that $(t, q_i) \in K$ implies $q_i \geq \beta_{m+n}h(t)$.

Suppose $(p, q) = (p_1\vee p_2, q_1 \vee q_2)$, 
where $p$ has length $\ell$ and each $p_i$ has length at most $\ell-1$.
Then, by the induction hypothesis, $q_i \geq \beta_{\ell-1 + n}h(p_i)$.
We show $q\geq \beta_{\ell+n}(a)$.
We have 
\[
q = q_1 \vee q_2 \geq \beta_{\ell-1+n}h(p_1) \vee \beta_{\ell-1+n}h(p_2) \geq 
\beta_{\ell+n}h(p_1\vee p_2) =
\beta_{\ell+n}h(p).
\]

Suppose $(p, q) = (p_1\wedge p_2, q_1 \wedge q_2)$, where $p$ has length $\ell$ 
and each $p_i$ has length at most $\ell$.
Then, by the induction hypothesis, $q_i \geq \beta_{\ell+ n}h(p_i)$.
We show $q\geq \beta_{\ell+n}(a)$.
We have 
\[
q = q_1 \wedge q_2 \geq \beta_{\ell+n}h(p_1) \wedge \beta_{\ell+n}h(p_2) \geq 
\beta_{\ell+n}h(p_1\wedge p_2) =
\beta_{\ell+n}h(p).
\]
\end{proof}
\end{comment}




\section{Miscellaneous Notes}
\noindent Let $K$ be a finite subset of $\ker h$.  Since $K$ is finite, we can find an $N < \omega$ such that for all $\binom{p}{q} \in K$, the following implications are satisfied:
\begin{align}
p &\leqslant x \quad \Longrightarrow \quad q \leqslant x_N\nonumber\\
p &\leqslant y \quad \Longrightarrow \quad q \leqslant y_N\label{eq:imp1}\\
p &\leqslant z \quad \Longrightarrow \quad q \leqslant z_N\nonumber
\end{align}
\begin{align}
p &\leqslant x \vee (y \wedge z) \quad \Longrightarrow \quad q \leqslant x_{N+1}\nonumber\\
p &\leqslant y \vee (x \wedge z) \quad \Longrightarrow \quad q \leqslant y_{N+1}\label{eq:imp2}\\
p &\leqslant z \vee (x \wedge y) \quad \Longrightarrow \quad q \leqslant z_{N+1}\nonumber
\end{align}

\medskip

\noindent {\bf Claim 4.3} If $N$ is chosen as just described, and if $\binom{p}{q} \in \langle K \rangle$ then the implications~\ref{eq:imp1} and~\ref{eq:imp2} hold.

\begin{proof}
  As usual, we proceed by induction on term complexity.
If $\binom{p}{q} \in K$, then by choice of $N$, there is nothing to prove.

\medskip
\noindent {\it Case 1.} Suppose $\binom{p}{q} = \binom{p_1}{q_1} \vee \binom{p_2}{q_2}$, where $\binom{p_1}{q_1}$ and $\binom{p_2}{q_2}$ satisfy~(\ref{eq:imp1}) and~(\ref{eq:imp2}). We show that $\binom{p}{q}$ satisfies these two implications as well.
Recall, in the notation above, $x_1:=x \vee (y \wedge z)$.


Assume $p\leqslant x_1$. We show $q \leq x_{N+1}$.
Since $p = p_1 \vee p_2 \leq x_1$, we have 
$p_1 \leqslant x_1$ and 
$p_2 \leqslant x_1$, so by the induction hypothesis,  
$q_1 \leqslant x_{N+1}$ and 
$q_2 \leqslant x_{N+1}$.  Therefore, $q = q_1 \vee q_2 \leq x_{N+1}$, as desired.

Now assume $p\leqslant x$. We show $q\leqslant x_N$. 
Since $p = p_1 \vee p_2 \leq x$, we have 
$p_1 \leqslant x$ and 
$p_2 \leqslant x$, so by the induction hypothesis,  
$q_1 \leqslant x_{N}$ and 
$q_2 \leqslant x_{N}$.  Therefore, $q = q_1 \vee q_2 \leq x_{N}$, as desired.


\medskip
\noindent {\it Case 2.} Suppose $\binom{p}{q} = \binom{p_1}{q_1} \wedge \binom{p_2}{q_2}$, where $\binom{p_1}{q_1}$ and $\binom{p_2}{q_2}$ satisfy~(\ref{eq:imp1}) and~(\ref{eq:imp2}). 

Assume $p\leqslant x_1 = x \vee (y \wedge z)$. 
We must show $q\leqslant x_{N+1}$. 
Since $p_1 \wedge p_2 \leqslant x_1$, then according to Theorem~\ref{thm:wordprob}, at least one of the following inequalities must hold:
\begin{enumerate}
  \item $p_1 \leqslant x_1$;
  \item $p_2 \leqslant x_1$;
  \item $p_1 \wedge p_2 \leqslant x$;
  \item $p_1 \wedge p_2 \leqslant y \wedge z$.
\end{enumerate}
By the induction hypothesis, (1) implies $q_1 \leq x_{N+1}$  and (2) implies $q_2 \leq x_{N+1}$.  In either case, $q = q_1 \wedge q_2 \leq x_{N+1}$, as desired.  In case (3), Theorem~\ref{thm:wordprob} implies that either $p_1 \leq x$ or $p_2 \leq x$, since $x$ is a generator. Therefore, $q_1 \leq x_N$ or $q_2 \leq x_N$ and we conclude that 
$q \leq x_N \leq x_{N+1}$, as desired.  It remains to prove $q \leq x_{N+1}$ for the final case in which $p_1 \wedge p_2 \leq y \wedge z$. 

If $p_1 \wedge p_2 \leq y \wedge z$, then 
$p_1 \wedge p_2 \leq y$ and $p_1 \wedge p_2 \leq z$. Therefore, both of the following disjunctions hold:
\begin{itemize}
  \item $p_1 \leq y$ or $p_2 \leq y$, and 
  \item $p_1 \leq z$ or $p_2 \leq z$. 
\end{itemize}
If $p_1 \leq y$ and $p_1 \leq z$, then $p_1 \leq x \vee (y\wedge z) = x_1$, so $q_1 \leq x_{N+1}$, so $q = q_1\wedge q_2 \leq x_{N+1}$, as desired.
Similarly, if $p_2 \leq y$ and $p_2 \leq z$, the desired conclusion holds.
Finally, consider the case in which $p_1 \leq y$ and $p_2 \leq z$. In this case $q_1 \leq y_N$ and $q_2 \leq z_N$.  Therefore, $q = q_1 \wedge q_2 \leq y_N \wedge z_N \leq x_N \vee (y_N\wedge z_N) = x_{N+1}$, as desired.
\end{proof}





  In the first case we have $p_1 = x \leq x \vee (y \wedge z) = x_1$, so
  the induction hypothesis implies $q_1\leq x_N$, so $v = q_1\wedge q_2\leq x_N$, as desired.  Notice that the second, third, and fourth items in the list submit to the same argument, so we are left with the fifth case, 
  in which $p_1\wedge p_2 \leq y \wedge z$. 
  This implies $p_1\wedge p_2 \leq y$, and $y$ is meet prime, so 
  $p_i\leq y$ for some $i\in \{1,2\}$.  But then, $p_i \leq y_1 = y\vee (x\wedge z)$, so by the induction hypothesis, $q_i \leq y_N$.
  Similarly, $p_1\wedge p_2 \leq z$ implies 
  $p_j\leq z$ for some $j\in \{1,2\}$, so $q_j \leq z_N$.
  Therefore, 
  \[
  v = q_1\wedge q_2 \leq y_N \wedge z_N \leq x \vee (y_N\wedge z_N) = x_N,\] 
  as desired.



  
  We show how to handle Cases~(\ref{item:p1leqx1}) and~(\ref{item:leqyz}). (Case~(\ref{item:p2leqx1}) is similar to~(\ref{item:p1leqx1}).)
  In Case~(\ref{item:p1leqx1}), we have
  $p_1\leq x_1$, so the induction hypothesis implies $q_1 \leq x_N$. To 
  conclude that $q \leq m_N$ in this case, we need to exploit condition~(\ref{leq:y1}); that is, 
  $p_1\wedge p_2 \leq y_1 = y \vee (x \wedge z)$. From this and (W+),
  one of the following holds:
  \begin{enumerate}[(i)]
    \item   $p_1=y$;
    \item   $p_2=y$;
    \item   $p_1\leq y_1$;
    \item   $p_2 \leq y_1$;
    \item   $p_1\wedge p_2 \leq x \wedge z$.  
  \end{enumerate}

\noindent {\bf Proof of Claim~\ref{claim:m}.}


\end{claim}
\begin{proof}






















\subsection{Bounded Homomorphisms}
\label{sec:bound-homom}

Following Freese, Jezek, Nation~\cite{MR1319815}, we define a pair of closure operators, denoted by superscripts $^\wedge$ and $^\vee$, on  subsets of an arbitrary lattice
$\mathbf L = \langle L, \vee, \wedge\rangle$ as follows: For each $A\subseteq L$, let
\[
A^\wedge = \{\bigwedge B : B \text{ is a finite subset of } A\}.
\]
We adopt the convention that if $\mathbf L$ has a greatest element $1_{\mathbf L}$, then 
$\bigwedge \emptyset = 1_{\mathbf L}$, and we include this in $A^\wedge$ for every $A \subseteq L$.
(For lattices without a greatest element, $\bigwedge \emptyset$ is undefined.)  The set $A^\vee$ is defined dually.

If $\mathbf K = \langle K, \vee, \wedge\rangle$ is a lattice generated by a finite set $X$, then we can write $K$ as a union of a chain of subsets $H_0\subseteq H_1 \subseteq \cdots$ defined inductively by setting $H_0 = X^\wedge$ and $H_{k+1} = (H_k)^{\vee \wedge}$, for all $k\geqslant 0$. By induction, each $H_n = X^{\wedge(\vee\wedge)^n}$ is a finite meet-closed subset of $K$, and $\bigcup H_n = K$, since $X$  generates $\mathbf X$.

Let $h \colon \mathbf K \to \mathbf L$ be a lattice epimorphism and for 
each $y \in L$ and $k< \omega$, define
\[
\beta_k(y) = \bigwedge \{w \in H_k : h(w) \geqslant a\}.
\]

On page 30 of~\cite{MR1319815}, immediately after Theorem 2.4, the authors make the following remark, which is a crucial ingredient of our proof:

``...[$h$ is lower bounded] if and only if for each $a \leqslant h(1_{\mathbf K})$ there exists $N\in \omega$ such that $\beta_n(a) = \beta_N(a)$ for all $n\geqslant N$.''

Equivalently, 
\begin{align}
h \text{ is not lower bounded } &\iff (\exists y_0 \in L)(\forall N)(\exists n> N)\, \beta_n(a) \neq \beta_N(a) \nonumber \\
&\iff (\exists y_0 \in L)(\exists N)(\forall n> N) \beta_n(a) \neq \beta_N(a).\nonumber
\end{align}
