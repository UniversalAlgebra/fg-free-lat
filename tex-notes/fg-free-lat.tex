%% FILE: diffTerm.tex
%% AUTHOR: William DeMeo, Peter Mayr, Nik Ruskuc
%% DATE: 14 May 2018
%% COPYRIGHT: (C) 2018 DeMeo, Mayr, Ruskuk

%%%%%%%%%%%%%%%%%%%%%%%%%%%%%%%%%%%%%%%%%%%%%%%%%%%%%%%%%%
%%                         BIBLIOGRAPHY FILE            %%
%%%%%%%%%%%%%%%%%%%%%%%%%%%%%%%%%%%%%%%%%%%%%%%%%%%%%%%%%%
%% The `filecontents` command will crete a file in the inputs directory called
%% refs.bib containing the references in the document, in case this file does
%% not exist already.
%% If you want to add a BibTeX entry, please don't add it directly to the
%% refs.bib file.  Instead, add it in this file between the
%% \begin{filecontents*}{refs.bib} and \end{filecontents*} lines
%% then delete the existing refs.bib file so it will be automatically generated
%% again with your new entry the next time you run pdfaltex.
\begin{filecontents*}{inputs/refs.bib}
@book {MR1319815,
    AUTHOR = {Freese, Ralph and Je{\v{z}}ek, Jaroslav and Nation, J. B.},
     TITLE = {Free lattices},
    SERIES = {Mathematical Surveys and Monographs},
    VOLUME = {42},
 PUBLISHER = {American Mathematical Society, Providence, RI},
      YEAR = {1995},
     PAGES = {viii+293},
      ISBN = {0-8218-0389-1},
   MRCLASS = {06B25 (06-02 06-04 06B20 68Q25)},
  MRNUMBER = {1319815 (96c:06013)},
MRREVIEWER = {T. S. Blyth},
       DOI = {10.1090/surv/042},
       URL = {http://dx.doi.org/10.1090/surv/042},
}
@article {MR3239624,
    AUTHOR = {Valeriote, M. and Willard, R.},
     TITLE = {Idempotent {$n$}-permutable varieties},
   JOURNAL = {Bull. Lond. Math. Soc.},
  FJOURNAL = {Bulletin of the London Mathematical Society},
    VOLUME = {46},
      YEAR = {2014},
    NUMBER = {4},
     PAGES = {870--880},
      ISSN = {0024-6093},
   MRCLASS = {08A05 (06F99 68Q25)},
  MRNUMBER = {3239624},
       DOI = {10.1112/blms/bdu044},
       URL = {http://dx.doi.org/10.1112/blms/bdu044},
}
@book {MR2839398,
    AUTHOR = {Bergman, Clifford},
     TITLE = {Universal algebra},
    SERIES = {Pure and Applied Mathematics (Boca Raton)},
    VOLUME = {301},
      NOTE = {Fundamentals and selected topics},
 PUBLISHER = {CRC Press, Boca Raton, FL},
      YEAR = {2012},
     PAGES = {xii+308},
      ISBN = {978-1-4398-5129-6},
   MRCLASS = {08-02 (06-02 08A40 08B05 08B10 08B26)},
  MRNUMBER = {2839398 (2012k:08001)},
MRREVIEWER = {Konrad P. Pi{\'o}ro},
}
@article{Freese:2009,
    AUTHOR = {Freese, Ralph and Valeriote, Matthew A.},
    TITLE = {On the complexity of some {M}altsev conditions},
    JOURNAL = {Internat. J. Algebra Comput.},
    FJOURNAL = {International Journal of Algebra and Computation},
    VOLUME = {19},
    YEAR = {2009},
    NUMBER = {1},
    PAGES = {41--77},
    ISSN = {0218-1967},
    MRCLASS = {08B05 (03C05 08B10 68Q25)},
    MRNUMBER = {2494469 (2010a:08008)},
    MRREVIEWER = {Clifford H. Bergman},
    DOI = {10.1142/S0218196709004956},
    URL = {http://dx.doi.org/10.1142/S0218196709004956}
  }
@article {MR3076179,
    AUTHOR = {Kearnes, Keith A. and Kiss, Emil W.},
     TITLE = {The shape of congruence lattices},
   JOURNAL = {Mem. Amer. Math. Soc.},
  FJOURNAL = {Memoirs of the American Mathematical Society},
    VOLUME = {222},
      YEAR = {2013},
    NUMBER = {1046},
     PAGES = {viii+169},
      ISSN = {0065-9266},
      ISBN = {978-0-8218-8323-5},
   MRCLASS = {08B05 (08B10)},
  MRNUMBER = {3076179},
MRREVIEWER = {James B. Nation},
       DOI = {10.1090/S0065-9266-2012-00667-8},
       URL = {http://dx.doi.org/10.1090/S0065-9266-2012-00667-8},
}
@misc{william_demeo_2016_53936,
  author       = {DeMeo, William and Freese, Ralph},
  title        = {AlgebraFiles v1.0.1},
  month        = May,
  year         = 2016,
  doi          = {10.5281/zenodo.53936},
  url          = {http://dx.doi.org/10.5281/zenodo.53936}
}
@article{FreeseMcKenzie2016,
	Author = {Freese, Ralph and McKenzie, Ralph},
	Date-Added = {2016-08-22 19:43:56 +0000},
	Date-Modified = {2016-08-22 19:45:50 +0000},
	Journal = {Algebra Universalis},
	Title = {Mal'tsev families of varieties closed under join or Mal'tsev product},
	Year = {to appear}
}
@misc{UACalc,
	Author = {Ralph Freese and Emil Kiss and Matthew Valeriote},
	Date-Added = {2014-11-20 01:52:20 +0000},
	Date-Modified = {2014-11-20 01:52:20 +0000},
	Note = {Available at: {\verb+www.uacalc.org+}},
	Title = {Universal {A}lgebra {C}alculator},
	Year = {2011}
}
@incollection {MR0472614,
    AUTHOR = {J{\'o}nsson, B. and Nation, J. B.},
     TITLE = {A report on sublattices of a free lattice},
 BOOKTITLE = {Contributions to universal algebra ({C}olloq., {J}\'ozsef
              {A}ttila {U}niv., {S}zeged, 1975)},
     PAGES = {223--257. Colloq. Math. Soc. J\'anos Bolyai, Vol. 17},
 PUBLISHER = {North-Holland, Amsterdam},
      YEAR = {1977},
   MRCLASS = {06A20},
  MRNUMBER = {0472614},
MRREVIEWER = {Ralph Freese},
}
@article {MR0313141,
    AUTHOR = {McKenzie, Ralph},
     TITLE = {Equational bases and nonmodular lattice varieties},
   JOURNAL = {Trans. Amer. Math. Soc.},
  FJOURNAL = {Transactions of the American Mathematical Society},
    VOLUME = {174},
      YEAR = {1972},
     PAGES = {1--43},
      ISSN = {0002-9947},
   MRCLASS = {06A20 (08A15)},
  MRNUMBER = {0313141},
MRREVIEWER = {G. Gratzer},
}
\end{filecontents*}
%:biblio
%\documentclass[12pt]{amsart}
\documentclass[12pt,reqno]{amsart}

%%%%%%% wjd: added these packages vvvvvvvvvvvvvvvvvvvvvvvvv
% PAGE GEOMETRY
% These settings are for letter format
\def\OPTpagesize{8.5in,11in}     % Page size
\def\OPTtopmargin{1in}     % Margin at the top of the page
\def\OPTbottommargin{1in}  % Margin at the bottom of the page
%% \def\OPTinnermargin{0.5in}    % Margin on the inner side of the page
\def\OPTinnermargin{1.5in}    % Margin on the inner side of the page
\def\OPTbindingoffset{0in} % Extra offset on the inner side
%% \def\OPToutermargin{0.75in}   % Margin on the outer side of the page
\def\OPToutermargin{1.5in}   % Margin on the outer side of the page
\usepackage[papersize={\OPTpagesize},
             twoside,
             includehead,
             top=\OPTtopmargin,
             bottom=\OPTbottommargin,
             inner=\OPTinnermargin,
             outer=\OPToutermargin,
             bindingoffset=\OPTbindingoffset]{geometry}
\usepackage{url,amssymb,enumerate,tikz,scalefnt}
\usepackage[normalem]{ulem} % for \sout (strikeout)   wjd: could remove this in final draft
\usepackage[colorlinks=true,urlcolor=blue,linkcolor=blue,citecolor=blue]{hyperref}
\usepackage{algorithm2e}
\usepackage{stmaryrd}

\newcommand{\mysetminus}{\ensuremath{-}}
%% uncomment the next line if we want to revert to the "set" minus notation
%% \renewcommand{\mysetminus}{\ensuremath{\setminus}}

\usepackage[yyyymmdd,hhmmss]{datetime}
\usepackage{background}
\backgroundsetup{
  position=current page.east,
  angle=-90,
  nodeanchor=east,
  vshift=-1cm,
  hshift=8cm,
  opacity=1,
  scale=1,
  contents={\textcolor{gray!80}{WORK IN PROGRESS.  DO NOT DISTRIBUTE. (compiled on \today\ at \currenttime)}}
}
%%%%%%  (end wjd addition of packages)


\usepackage{pdfcomment}
\usepackage{color}
\usepackage{amsmath}
\usepackage{amssymb}
\usepackage{amsfonts}
\usepackage{mathtools}
\usepackage{amscd}
%% \usepackage{exers}
\usepackage{inputs/wjdlatexmacs}

\usepackage[mathcal]{euscript}
\usepackage{comment}
\usepackage{tikz}
\usetikzlibrary{math} %needed tikz library

\renewcommand{\th}[2]{#1\mathrel{\theta}#2}
\newcommand{\infixrel}[3]{#2\mathrel{#1}#3}
\newcommand\llb{\ensuremath{\llbracket}}
\newcommand\rrb{\ensuremath{\rrbracket}}

%%////////////////////////////////////////////////////////////////////////////////
%% Theorem styles
\numberwithin{equation}{section}
\theoremstyle{plain}
\newtheorem{theorem}{Theorem}[section]
\newtheorem{lemma}[theorem]{Lemma}
\newtheorem{proposition}[theorem]{Proposition}
\newtheorem{prop}[theorem]{Prop.}
\theoremstyle{definition}
\newtheorem{conjecture}{Conjecture}
\newtheorem{claim}[theorem]{Claim}
\newtheorem{subclaim}{Subclaim}
\newtheorem{corollary}[theorem]{Corollary}
\newtheorem{definition}[theorem]{Definition}
\newtheorem{notation}[theorem]{Notation}
\newtheorem{Fact}[theorem]{Fact}
\newtheorem*{fact}{Fact}
\newtheorem{example}[theorem]{Example}
\newtheorem{examples}[theorem]{Examples}
\newtheorem{exercise}{Exercise}
\newtheorem*{lem}{Lemma}
\newtheorem*{cor}{Corollary}
\newtheorem*{remark}{Remark}
\newtheorem*{remarks}{Remarks}
\newtheorem*{obs}{Observation}

\title[Kernels of Lattice Epimorphisms]{Kernels of epimorphisms of finitely generated free lattices}
\author[W.~DeMeo]{William DeMeo}
\email{williamdemeo@gmail.com}
%% \urladdr{http://williamdemeo.github.io}
%% \address{University of Colorado\\Mathematics Dept\\Boulder 80309\\USA}

\author[P.~Mayr]{Peter Mayr}
%% \email{}\urladdr{}
%% \address{University of Colorado\\Mathematics Dept\\Boulder 80309\\USA}

\author[N.~Ru\v{s}kuc]{Nik Ru\v{s}kuc}
%% \email{}\urladdr{}
%% \address{University of St. Andrews\\Mathematics Dept\\St. Andrews, Scottland}

%% \thanks{The first and second authors were supported by the National
%% Science Foundation under Grant No...}

\date{\today}

\begin{document}

\maketitle

\section{Introduction}


\section{Main Theorem}

Let $X$ be a finite set and $\mathbf F := \mathbf F(X)$ the free lattice generated by $X$.

\begin{theorem}
\label{thm:forward}
Suppose $\mathbf L = \langle L, \wedge, \vee\rangle$ is a finite lattice and 
$h\colon \mathbf{F} \twoheadrightarrow \mathbf{L}$ a lattice epimorphism.
If $h$ is bounded then the kernel of $h$ is a finitely generated sublattice 
of $\mathbf F \times \mathbf F$.
\end{theorem}

\begin{proof}
Assume $h$ is bounded.  That is, the preimage of each $y\in L$ under $h$ is bounded.  For each $y\in L$, let $\alpha y= \bigvee h^{-1}\{y\}$ and $\beta y = \bigwedge h^{-1}\{y\}$ denote the greatest and least elements of $h^{-1}\{y\}$, respectively (both of which exist by the boundedness assumption).  Observe that $h \alpha h = h$, and $h \beta h = h$. In fact, $\alpha$ and $\beta$ are adjoint to $h$. Indeed, it is easy to see that
\[
h x \leqslant y \quad \Leftrightarrow \quad x \leqslant \alpha y,
\]
\[
y \leqslant h x \quad \Leftrightarrow \quad \beta y \leqslant x.
\]

For each $y \in L$, let $X_y := X\cap h^{-1}\{y\}$, the set of generators that lie in the inverse image of $y$ under $h$.
Let $G$ be the (finite) set of pairs in $\mathbf F \times \mathbf F$ defined as follows:
\[
G = \bigcup_{y \in L}\{(x, \alpha y), (\alpha y, x), (x, \beta y), (\beta y, x), (\alpha y, \beta y), (\beta y, \alpha y) : x \in X_y\}.
\]
We claim that $G$ generates $\ker h$.  To prove this, we first show, by induction on term complexity, that for every $y \in L$, for every $r \in h^{-1}\{y\}$, the pairs $(r,\alpha y)$ and $(r,\beta y)$ belong to the sublattice $\langle G \rangle \leqslant \mathbf F \times \mathbf F$ generated by $G$.

\begin{itemize}
  \item {\bf Case 0.} Suppose $r \in X$. Then $(r,\alpha y)$ and $(r,\beta y)$ belong to $G$ itself, so there's nothing to prove.  

\item {\bf Case 1.} Suppose $r = s \vee t$. Assume (the induction hypothesis) that
$(s, \alpha {h(s)})$, $(s, \beta{h(s)})$, $(t, \alpha {h(t)})$, and 
$(t, \beta{h(t)})$ belong to $\langle G \rangle$. Then 
$y = h (r) = h(s\vee t) = h (s)\vee h(t)$, so 
\[
h(\alpha {h(s)} \vee \alpha {h(t)})= h\alpha h(s) \vee h\alpha h(t)=
h(s) \vee h(t) = y.
\]
Likewise, $h(\beta{h(s)} \vee \beta {h(t)})= h(s) \vee h(t) = y$.
Therefore, 
\[
\beta y \leqslant \beta h(s) \vee \beta h(t) \leqslant \alpha {h(s)} \vee \alpha {h(t)} \leqslant \alpha y.
\]
Also, $r \leqslant \alpha y$, so $r = \alpha y \wedge (s\vee t)$.  Taken together, these observations yield
\begin{align}
\left(\begin{array}{c} r \\ \beta y\end{array}\right) &= 
\left(\begin{array}{c} \alpha y \wedge (s\vee t) \\ \beta y\end{array}\right) = 
  \left(\begin{array}{c} \alpha y \wedge (s\vee t) \\ \beta y \wedge (\beta {h(s)} \vee \beta {h(t)}) \end{array}\right)\nonumber\\
  &= 
\left(\begin{array}{c} \alpha y\\ \beta y\end{array}\right) \wedge 
  \left[\left(\begin{array}{c}s \\ \beta {h(s)}\end{array}\right) \vee \left(\begin{array}{c}t \\ \beta {h(t)} \end{array}\right)\right], \nonumber
\end{align}
and each term in the last expression belongs to $\langle G \rangle$, so $(r, \beta y) \in \langle G \rangle$, as desired.

Similarly, $(r, \alpha y) \in \langle G \rangle$.  Indeed, $\beta y \leqslant r$ 
implies $r = \beta y \vee s\vee t$, and 
$\beta {h(s)} \vee \beta {h(t)} \leqslant \alpha y$ implies 
$\alpha y = \alpha y \vee \beta {h(s)} \vee \beta {h(t)}$. Therefore,
\begin{align*}
\left(\begin{array}{c} r \\ \alpha y\end{array}\right) 
&= \left(\begin{array}{c} \beta y \vee s\vee t \\ \alpha y \vee \beta {h(s)} \vee \beta {h(t)} \end{array}\right)\\
&= \left(\begin{array}{c} \beta y\\ \alpha y\end{array}\right) \vee 
\left(\begin{array}{c}s \\ \beta {h(s)}\end{array}\right) \vee \left(\begin{array}{c}t \\ \beta {h(t)} \end{array}\right).
\end{align*}

\item {\bf Case 2.} Suppose $r = s \wedge t$. Assume $(s, \alpha {h(s)})$, $(s, \beta{h(s)})$, $(t, \alpha {h(t)})$, and $(t, \beta{h(t)})$ belong to $\langle G \rangle$. Then $h(s\wedge t) = h(r) = y$, so 
$h(\alpha {h(s)} \wedge\alpha {h(t)}) = y = h(\beta {h(s)} \wedge\beta {h(t)})$, so
$\beta y \leqslant \beta h(s) \wedge \beta h(t) \leqslant \alpha {h(s)} \wedge \alpha {h(t)} \leqslant \alpha y$.
Also, $\beta y \leqslant r \leqslant \alpha y$ so $r = \alpha y \wedge s\wedge t$
and $r = \beta y \vee (s\wedge t)$. Altogether, we have
\begin{align*}
\left(\begin{array}{c} r \\ \alpha a\end{array}\right) &= 
\left(\begin{array}{c} \beta y \vee (s\wedge t) \\ \alpha y \vee (\alpha {h(s)} \wedge \alpha {h(t)}) \end{array}\right)\\
&= \left(\begin{array}{c} \beta y\\ \alpha y\end{array}\right) \vee
\left[\left(\begin{array}{c}s \\ \alpha {h(s)}\end{array}\right) \wedge \left(\begin{array}{c}t \\ \alpha {h(t)} \end{array}\right)\right],
\end{align*}
and each term in the last expression belongs to $\langle Y \rangle$, as desired.
Similarly,
\begin{align*}
\left(\begin{array}{c} r \\ \beta y\end{array}\right) &= 
\left(\begin{array}{c} \alpha y \wedge s\wedge t \\ \beta y \wedge \alpha {h(s)} \wedge \alpha {h(t)} \end{array}\right)\\
&= \left(\begin{array}{c} \alpha y\\ \beta y\end{array}\right) \wedge 
\left(\begin{array}{c}s \\ \alpha {h(s)}\end{array}\right) \wedge \left(\begin{array}{c}t \\ \alpha {h(t)} \end{array}\right).
\end{align*}
Note that, in both of the derivations above, we could have used $\beta$'s instead of $\alpha$'s; that is,
\begin{align*}
\left(\begin{array}{c} r \\ \alpha y\end{array}\right) &= 
\left(\begin{array}{c} \beta y \vee (s\wedge t) \\ \alpha y \vee (\beta {h(s)} \wedge \beta {h(t)}) \end{array}\right)\\
&= \left(\begin{array}{c} \beta y\\ \alpha y\end{array}\right) \vee
\left[\left(\begin{array}{c}s \\ \beta {h(s)}\end{array}\right) \wedge \left(\begin{array}{c}t \\ \beta {h(t)} \end{array}\right)\right],
\end{align*}
and
\begin{align*}
\left(\begin{array}{c} r \\ \beta y\end{array}\right) &= 
\left(\begin{array}{c} \alpha y \wedge s\wedge t \\ \beta y \wedge \beta {h(s)} \wedge \beta {h(t)} \end{array}\right)\\
&= \left(\begin{array}{c} \alpha y\\ \beta y\end{array}\right) \wedge 
\left(\begin{array}{c}s \\ \beta {h(s)}\end{array}\right) \wedge \left(\begin{array}{c}t \\ \beta {h(t)} \end{array}\right).
\end{align*}

In each case, we end up with an expression involving terms from $\langle G \rangle$, 
and this proves that $(r, \alpha y)$ and $(r, \beta y)$ belong to $\langle G \rangle$, as desired.
\end{itemize}
\end{proof}

We conjecture the converse of Theorem~\ref{thm:forward}.
\begin{conjecture}
Suppose $\mathbf L = \langle L, \wedge, \vee\rangle$ is a finite lattice and 
$h\colon \mathbf F \twoheadrightarrow \mathbf L$ a lattice epimorphism.
If the kernel of $h$ is a finitely generated sublattice 
of $\mathbf F \times \mathbf F$, then $h$ is bounded.
\end{conjecture}
We know how to prove this conjecture if we assume that, 
whenever $h$ is not upper (lower) bounded, then there is a class of 
$\ker h$ that contains an infinite ascending (descending) chain as well as a meet (join)
prime element of $\mathbf F$. (See Appendix~\ref{app:conjecture} for the proof). 
% However, we don't know whether there will always be a meet (join) prime element 
% in one of the unbounded classes of $\ker h$; that is, a class containing an 
% infinite ascending (descending) chain.

J.B.~Nation brought the next result to our attention.  
It gives an example in which the unbounded
classes of $\ker h$ do not contain generators of $\mathbf F$.
\begin{proposition}
Let $\mathbf{F} = \mathbf{F}(x,y,z)$, and let $\mathbf{L} = \mathbf{F}_{\mathbf{M}_3}(0,1,2)$ (see Figure~\ref{fig:1}).  
Let $h\colon \mathbf{F} \twoheadrightarrow \mathbf{L}$ be the epimorphism induced by $x\mapsto 0$, $y\mapsto 1$, $z\mapsto 2$. Then $\operatorname{ker}h$ is not finitely generated.  
\end{proposition}

\newcommand{\dotsize}{1.5pt}
\tikzstyle{lat} = [circle,draw,inner sep=\dotsize]
\newcommand{\figscale}{1}
\begin{figure}
\begin{tikzpicture}[scale=\figscale]
  \foreach \j in {0,...,8} {
    \node[lat] (0\j) at (0,\j) {};
  }
  \foreach \j in {1,2,4,6,7} {
    \node[lat] (n1\j) at (-1,\j) {};
    \node[lat] (1\j) at (1,\j) {};
  }
  \foreach \j in {3,5} {
    \node[lat] (n2\j) at (-2,\j) {};
    \node[lat] (2\j) at (2,\j) {};
  }
  \node[lat] (n34) at (-3,4) {};
  \node[lat] (34) at (3,4) {};

  \draw[semithick] (00) -- (01);
  \draw[semithick] (02) -- (03) -- (04) -- (05) -- (06);
  \draw[semithick] (07) -- (08);
  \draw[semithick] (n16) -- (n17);
  \draw[semithick] (16) -- (17);
  \draw[semithick] (n11) -- (n12);
  \draw[semithick] (11) -- (12);

  \draw[semithick] (00) -- (n11) -- (02) -- (11) -- (00);
  \draw[semithick] (06) -- (n17) -- (08) -- (17) -- (06);

  \draw[semithick] (01) -- (n12) -- (n23) -- (n34);
  \draw[semithick] (01) -- (12) -- (23) -- (34);
  \draw[semithick] (12) -- (03) -- (n14) -- (n25);
  \draw[semithick] (n12) -- (03) -- (14) -- (25);
  \draw[semithick] (23) -- (14) -- (05) -- (n16);
  \draw[semithick] (n23) -- (n14) -- (05) -- (16);
  \draw[semithick] (34) -- (25) -- (16) -- (07);
  \draw[semithick] (n34) -- (n25) -- (n16) -- (07);
  % \node (n13) at (-1,3) [circle,fill,inner sep=\dotsize]{};
  % \node (n15) at (-1,5) [circle,fill,inner sep=\dotsize]{};
  % \node (n24) at (-2,4) [circle,fill,inner sep=\dotsize]{};
  \node[lat] (n13) at (-1,3) {};
  \node[lat] (n15) at (-1,5) {};
  \node[lat] (n24) at (-2,4) {};
  \draw[thick,blue] (02) -- (n13) -- (n24) -- (n15) -- (06);
  \draw[thick,blue] (n13) -- (04) -- (n15);

  \node (01) at (0,1) [green,circle,fill,inner sep=\dotsize]{};
  \node (n11) at (-1,1) [green,circle,fill,inner sep=\dotsize]{};
  \node (11) at (1,1) [green,circle,fill,inner sep=\dotsize]{};
  \node (n25) at (-2,5) [red,circle,fill,inner sep=\dotsize]{};
  \node (x) at (-3,4) [circle,fill,inner sep=\dotsize]{};
  \node (y) at (3,4) [circle,fill,inner sep=\dotsize]{};
  \node (25) at (2,5) [red,circle,fill,inner sep=\dotsize]{};
  \node (z) at (-2,4) [circle,fill,inner sep=\dotsize]{};
  \node (n15) at (-1,5) [red,circle,fill,inner sep=\dotsize]{};
  \node (03) at (0,3) [orange,circle,fill,inner sep=\dotsize]{};
  \draw (x) node[left] {$0$};
  \draw (n11) node[left] {$0 \wedge 2$};
  \draw (25) node[right] {$1 \vee (0 \wedge 2)$};
  \draw (y) node[right] {$1$};
  \draw (z) node[left] {$2$};
  \end{tikzpicture}
  \caption{The free lattice over $M_3$ generated by $\{0, 1, 2\}$. 
  Green dots identify elements $0 \wedge 1$, 
  $0 \wedge 2$, and $1 \wedge 2$; red dots identify $0 \vee (1 \wedge 2)$, 
  $1 \vee (0 \wedge 2)$, and $2 \vee (0 \wedge 1)$.}
  \label{fig:1}
\end{figure}


\begin{proof}
For each $s \in \{x, y, z, m\}$, define the sequence $\{s_i : i < \omega\}$
of elements of $\mathbf F$ as follows: 
$x_0 = x$, $y_0 = y$, $z_0 = z$, and for $i\geqslant 0$, 
\begin{gather*}
  x_{i+1} = x\vee (y_i \wedge z_i), \quad  y_{i+1} = y\vee (x_i \wedge z_i), \quad  z_{i+1} = z\vee (x_i \wedge y_i),\\
  m_i = (x_i \wedge y_i) \vee (x_i \wedge z_i) \vee (y_i \wedge z_i).
\end{gather*}

\begin{claim}\label{claim:000}
  If $\{s_i : i< \omega\}$ is any one of the four sequences just defined, then for $i\geq 1$, 
  we have $s_{i+1} > s_i$ and $h(s_{i+1}) = h(s_i)$.
\end{claim}
\begin{proof}
  % It suffices to prove the claim for the sequences $\{m_n\}$ and $\{x_n\}$.
First observe that $h(x_1) = 0 \vee (1 \wedge 2)$.  We begin by proving 
$h(x_{2}) = h(x_1)$.
By definition, we have 
$h(x_{2}) = h(x \vee (y_1 \wedge z_1)) = h(x) \vee \bigl[h(y_1) \wedge h(z_1)\bigr]$,
and $h(y_1) = h(y \vee (x \wedge z)) =   h(y) \vee \bigl[h(x) \wedge h(z)\bigr] 
= 1 \vee (0 \wedge 2)$. Similarly, $h(z_1) = 2 \vee (0 \wedge 1)$.
Therefore, 
\begin{equation}\label{eq:h2h1} 
  h(x_{2}) = 
0 \vee \bigl\{[1 \vee (0 \wedge 2)] \wedge [2 \vee (0 \wedge 1)]\bigr\}.
\end{equation}
Recall, the modular law: 
$x \leq b$ implies $x \vee (a \wedge b) = (x \vee a) \wedge b$.
Applying this law with $a = 1$ and $x = 0 \wedge 2 \leq 2 \vee (0 \wedge 1) = b$, we have 
\[
  (0 \wedge 2) \vee \bigl\{ 1 \wedge [2 \vee (0 \wedge 1)]\bigr\} = [1 \vee (0 \wedge 2)] \wedge [2 \vee (0 \wedge 1)],
\]  
which is the right joinand in~\ref{eq:h2h1}.  Therefore,
\[
  h(x_{2}) = 
  0 \vee 
  (0 \wedge 2) \vee \bigl\{ 1 \wedge [2 \vee (0 \wedge 1)]\bigr\} = 
  0 \vee 
  \bigl\{ 1 \wedge [2 \vee (0 \wedge 1)]\bigr\}.
\]
Applying the modular law once more to $1 \wedge [2 \vee (0 \wedge 1)]$, 
with $a = 2$ and $x = 0\wedge 1 \leq 1 = b$, we have
\[
  1 \wedge [2 \vee (0 \wedge 1)]  = (0 \wedge 1) \vee (1\wedge 2).
\]
% [0 \vee (1 \wedge 2)] \wedge [1 \vee (0 \wedge 2)]
Therefore,
\[
  h(x_{2}) = 
  0 \vee 
  \bigl\{ 1 \wedge [2 \vee (0 \wedge 1)]\bigr\} = 
  0 \vee (0 \wedge 1) \vee (1\wedge 2) = 0 \vee (1\wedge 2),
 \] 
as desired. 
Of course, $h(y_{2}) = h(y_1)$ and $h(z_{2}) = h(z_1)$ can be checked similarly.

Fix $n$ and assume $h(x_{n}) = h(x_1)$, $h(y_{n}) = h(y_{1})$, and $h(z_{n}) = h(z_{1})$. 
  Then,
  \[h(x_{n+1}) = h(x \vee (y_n \wedge z_n))
    =h(x) \vee (h(y_n) \wedge h(z_n))= 0 \vee (h(y_{1}) \wedge h(z_{1})),\]
which, as we observed in the first line of the proof, is equal to $h(x_2)$, which in turn is 
$h(x_1)$, as desired.  By the same argument, $h(y_{n+1})=h(y_{1})$ and $h(z_{n+1})=h(z_{1})$.
This proves that for all $n \geq 1$, we have $h(s_{n})=h(s_{1})$, when 
$\{s_n\}$ is $\{x_n\}$ or $\{y_n\}$ or $\{z_n\}$.

Finally, consider $\{m_n\}$. For all $n\geq 1$, we have
\begin{align*}
h(m_n) &= \bigl[h(x_n) \wedge h(y_n)\bigr] \vee \bigl[h(x_n) \wedge h(z_n)\bigr]\vee \bigl[h(y_n) \wedge h(z_n)\bigr] \\
&= \bigl[h(x_1) \wedge h(y_1)\bigr] \vee \bigl[h(x_1) \wedge h(z_1)\bigr]\vee \bigl[h(y_1) \wedge h(z_1)\bigr].
\end{align*}
(In fact, it is not hard to see that each of the joinands in the last expression is equal
to the orange element in Figure~\ref{fig:1}.)  Therefore, $h(m_n) = h(m_1)$ for all $n\geq 1$.

\wjd{TODO complete proof of the claim}
It remains to show that $s_{i+1} > s_i$ for all $i<\omega$.
\end{proof}
% This completes the proof of Claim~\ref{claim:000}.
\wjd{TODO complete proof of Prop 2.2}
\end{proof}

Notice that the example above involves an infinite ascending chain $\{x_n\} \subseteq \mathbf F$
such that $0 = h(x_0) \neq h(x_1) = h(x_2) = \cdots$; thus, for $n\geq 1$, all $x_n$ belong to the same 
(unbounded) class of $\ker h$, and this class contains $x_1 = x \vee (y \wedge z)$.
However, $x_1$ is not meet prime, since $y \nleq x \vee (y \wedge z)$ and $z \nleq x \vee (y \wedge z)$, yet 
$y \wedge z \leq x \vee (y \wedge z)$, so we cannot use our previous proof technique.

As above, let $x_1 = x \vee (y \wedge z)$ and define $b = 0\vee (1 \wedge 2)$.  
In particular, $h(x_1) = b$. 
To use the same inductive step that we used in our previous proofs, we would need to know 
that $p_1 \wedge p_2 \leq x_1$ implies either $p_1 \leq x_1$ or $p_2 \leq x_1$.  But this 
is not the case here since we could have, for example, $p_1 = y$ and $p_2 = z$. In fact, 
by Lemma~\ref{lem:prime}, the only meet prime elements of $\mathbf F(X)$ have the 
form $\bigvee S$ for $S\subseteq X$.  

Suppose $K \subseteq \mathbf F \times \mathbf F$ is a finite set. 
We wish to prove $\<K\> \neq \ker h$.
First, let's try to show there exists $N < \omega$ such that 
for all $(p, q)\in \mathbf F \times \mathbf F$ with $h(p) = b = h(q)$, 
the following implication holds:
\begin{equation}\label{eq:dcc}
\text{ if $(p, q) \in \<K\>$ and $p \leq x_1$, then $q \leq \alpha_N(b)$.} 
%, \; \text{ for all $(p, q) \in \<K\>\cap h^{-1}(y_0)^2$.}
\end{equation}
Since $\alpha_n(b)$ is an infinite ascending chain, 
%and since $K$ is a finite set, %say, $K = \{(p_1, q_1), \dots, (p_m, q_m)\}$,
we can certainly find $N < \omega$ such that~(\ref{eq:dcc}) holds for all 
of the (finitely many) pairs in $K$.

Now suppose $(p, q) = (p_1, q_1) \wedge (p_2, q_2)$ and $h(p) = b = h(q)$. 
Then $p_1$, $p_2$, $q_1$, $q_2$ must all belong to $h^{-1}\{b\}$ as well. 
Indeed, $b = h(p) = h(p_1 \wedge p_2)= h(p_1) \wedge h(p_2)$, and
$b = 0\vee (1 \wedge 2)$ is meet irreducible, so 
$h(p_1) = b = h(p_2)$. Similarly for $q_1$ and $q_2$.

Assume $(p, q) \in \<K\>$ and $p = p_1\wedge p_2 \leq x_1$.  
We must show $q_1 \wedge q_2 \leq \alpha_N(b)$.
Assume the induction hypothesis that $(p_i, q_i)$ satisfies~(\ref{eq:dcc}) for both $i = 1, 2$.

If we knew that either $p_1\leq x_1$ or $p_2\leq x_1$, the we would have that either 
$q_1 \leq \alpha_N(b)$ or 
$q_2 \leq \alpha_N(b)$, by the induction hypothesis; in either case, we could 
conclude that $q_1 \wedge q_2 \leq \alpha_N(b)$, as desired. Unfortunately, we do 
not know that either $p_1\leq x_1$ or $p_2\leq x_1$, since $x_1$ is not meet prime.

To say that $x_1$ is meet prime means that \emph{for all} $p_1$ and $p_2$ with 
$p_1 \wedge p_2 \leq x_1$, we have either $p_1 \leq x_1$ or $p_2 \leq x_1$.  
But this is actually more that we need.
For our purposes, we only require that the condition
\begin{equation}
p_1 \wedge p_2 \leq x_1 \quad \Longrightarrow \quad p_1 \leq x_1 \text{ or } p_2 \leq x_1,
\end{equation}
is satisfied for some special elements $p_1$ and $p_2$, namely...?

\vfill

Maybe we should try to exploit a $D^d$-cycle here, involving
\[p_1 \wedge (p_1 \to x_1) \leq x_1\]
We know that the maximum element (denoted here by $p_1 \to x_1$), 
among those that meet with $p_1$ below $x_1$, exists by the existence of a 
$D^d$-cycle. 


% \begin{claim}
% In $\mathbf F(x,y,z)$ the element $x \vee (y \wedge z)$ is meet prime.
% \end{claim}
% \begin{proof}
%   Suppose \[u \wedge v \leq x \vee (y \wedge z).\]
% We will prove: 
% \begin{equation}\label{eq:goal}
%   u \leq x \vee (y \wedge z) \quad \text{ or } \quad v \leq x \vee (y \wedge z).
% \end{equation}
% 
% By Theorem~\ref{thm:wordprob} (6), at least one of the following must hold:
% \begin{enumerate}
% \item $u\leq x \vee (y \wedge z)$;
% \item $v \leq x \vee (y \wedge z)$;
% \item $u\wedge v \leq x$; 
% \item $u\wedge v \leq  y \wedge z$.
% \end{enumerate}
% If either of the first two cases hold, then we are done, so consider Case (3).
% Since $x$ is a generator, Theorem~\ref{thm:wordprob} asserts that either $u\leq x$ 
% or $v \leq x$; \textit{a fortiori},~(\ref{eq:goal}) holds.  It remains to consider Case (4): 
% assume $u\wedge v \leq  y \wedge z$. Then $u\wedge v \leq  y$, so 
% (again by Theorem~\ref{thm:wordprob}) either $u\leq y$
% or $v \leq y$.  Similarly $u\wedge v \leq  z$, so either $u\leq z$ or $v \leq z$. 
% If $u \leq y$ and $u\leq z$, then $u\leq y \wedge z$ and~(\ref{eq:goal}) holds.  
% Similarly for $v$.  Assume (wlog) that $u \leq y$ and $v \leq z$.
% 
% \end{proof}
% 





%%%%%%%%%%%%%%%%%%%%%%%%%%%%%%%%%%%%%%%%%%%%%%%%%%%%%%%%%%%%%%%%%%%%%%%%%%%%%%%%%%55
\newpage
%%%%%%%%%%%%%%%%%%%%%%%%%%%%%%%%%%%%%%%%%%%%%%%%%%%%%%%%%%%%%%%%%%%%%%%%%%%%%%%%%%55



  \appendix


\section{Proof of Conjecture under special assumptions}\label{app:conjecture}
As usual, let $X$ be a finite set and let $\mathbf F := \mathbf F (X)$ be the free 
lattice generated by $X$.
\begin{proposition}\label{prop:conjecture}
Suppose $\mathbf L = \langle L, \wedge, \vee\rangle$ is a finite lattice and 
$h\colon \mathbf{F} \twoheadrightarrow \mathbf{L}$ a lattice epimorphism.
Suppose also that there is a class of $\ker h$ containing an infinite descending 
chain as well as a join prime element of $\mathbf F$. 
Then the kernel of $h$ is not a finitely generated sublattice 
of $\mathbf F \times \mathbf F$.
\end{proposition}
\begin{proof} Let $y_0\in L$ and suppose 
  $x_0 \in h^{-1}\{y_0\}$ is a join prime element of $\mathbf F$. 
  Suppose also that the class $h^{-1}\{y_0\}$ contains an 
  infinite descending chain, $\beta_0(y_0) > \beta_1(y_0) > \cdots$.
  Let $K$ be a finite subset of $\ker h$, say, 
  $K = \{(p_1, q_1), \dots, (p_{m}, q_{m})\} \subseteq \ker h$.
  We prove $\langle K \rangle \neq \ker h$. 
  Since $K$ is an arbitrary finite subset of $\ker h$, this will prove 
  that $\ker h$ is not finitely generated.

\begin{claim}
  \label{claim:1}
There exists $N<\omega$ such that for all $(p_i, q_i)$ in $K$, if $p_i \geqslant x_0$, then $q_i \geqslant \beta_N (y_0)$.
\end{claim}
\begin{proof}
Fix $i$ and $(p_i, q_i) \in K$ (so, $h(p_i) = h(q_i)$). Define $N_i$ according 
to which of the following cases holds:
\begin{enumerate}
\item If $p_i \ngeqslant x_0$, let $N_i = 0$.  
\item If $p_i\geqslant x_0$, then $x_0 = x_0\wedge p_i$, so 
$y_0 = h(x_0) = h(x_0) \wedge h(p_i) \leqslant h(p_i)$, so $y_0\leqslant h(q_i)$. 
Also, $h(x_0 \wedge q_i) = h(x_0) \wedge h(q_i) = y_0$, so $x_0\wedge q_i \in h^{-1}\{y_0\}$. 
Therefore (since $\{\beta_i(y_0)\}$ is an infinite descending chain),
there exists $n_i>0$ such that $x_0 \wedge q_i \geqslant\beta_{n_i}(y_0)$. 
Let $N_i = n_i$ in this case (so $q_i \geqslant \beta_{N_i}(y_0)$).
\end{enumerate}
After defining $N_i$ this way for each pair $(p_i,q_i)\in K$, 
let $N := \max_i N_i$. % : 1 \leqslant i \leqslant m\}$. 
Then the desired implication holds for all $1\leqslant i \leqslant m$, that is,
\begin{equation}
\label{eq:star3}    
p_i \geqslant x_0 \quad \Longrightarrow \quad q_i \geqslant \beta_N(y_0),
\end{equation}
so Claim~\ref{claim:1} is proved.
\end{proof}

\begin{claim}
  \label{claim:2}
  There exists $N < \omega$ such that, for all $(p, q) \in \langle K \rangle$,
\begin{equation}
  \label{eq:claim2}
p \geqslant x_0 \quad \Longrightarrow \quad q \geqslant \beta_N(y_0).
\end{equation}
\end{claim}
\begin{proof} Define $N$ as in the proof of Claim~\ref{claim:1}, so that 
for all $(p_i,q_i) \in K$ the implication~(\ref{eq:star3}) holds. Fix $(p, q) \in \langle K \rangle$. We prove~(\ref{eq:claim2}) by induction on the complexity of $(p, q)$.  If $(p, q) \in K$, then there's nothing to prove.
We split the induction step into two cases.
\begin{enumerate}
\item Assume $(p, q) = (p_1, q_1) \wedge (p_2, q_2)$, where $p_i$, $q_i$ ($i = 1, 2$) satisfy~(\ref{eq:claim2}).  Assume $p\geqslant x_0$. %% (We show $q \geqslant \beta_N(y_0)$.)
  Then $p = p_1 \wedge p_2 \geqslant x_0$, so $p_1 \geqslant x_0$ and $p_2 \geqslant x_0$, so (by the induction hypothesis) $q_1\geqslant \beta_N(y_0)$ and $q_2\geqslant \beta_N(y_0).$ Therefore, $q = q_1 \wedge q_2 \geqslant \beta_N(y_0),$ as desired.
\item Assume $(p, q) = (p_1, q_1) \vee (p_2, q_2)$, where $p_i$, $q_i$ ($i = 1, 2$) 
satisfy~(\ref{eq:claim2}). Assume $p\geqslant x_0$. Then $p = p_1 \vee p_2 \geqslant x_0$.  
Since $x_0$ is join prime, either $p_1 \geqslant x_0$ or $p_2 \geqslant x_0$.  
Assume (wlog) $p_1 \geqslant x_0$. Then, by the induction hypothesis, $q_1\geqslant \beta_N(y_0)$.
Therefore, $q = q_1 \vee q_2 \geqslant q_1 \geqslant \beta_N(y_0),$ as desired.
\end{enumerate}
This completes the proof of Claim~\ref{claim:2}.
\end{proof}
We can now see that $K$ does not generate $\ker h$. Indeed, 
% $\beta_0(y_0) > \beta_1(y_0) > \cdots$ is an infinite descending chain, and
if $N$ is as in the proof of Claim~\ref{claim:2}, then $\beta_{N}(y_0) > \beta_{N+1}(y_0)$, 
so the pair $(p, q) = (x_0, \beta_{N+1}(y_0))$ does not 
satisfy condition~\ref{eq:claim2}, so does not belong to $\langle K\rangle$.
Yet, $(x_0, \beta_{N+1}(y_0)) \in \ker h$. Thus, $K$ does not generate $\ker h$.  
% Since $K$ was an arbitrary finite subset of $\ker h$, this completes the proof of 
% Proposition~\ref{prop:conjecture}.
\end{proof}

%%%%%%%%%%%%%%%%%%%%%%%%%%%%%%%%%%%%%%%%%%%%%%%%%%%%%%%%%%%%%%%%%%%%%%%%%%%%%%%%%%%%%%%%%%%%%%
\newpage
%%%%%%%%%%%%%%%%%%%%%%%%%%%%%%%%%%%%%%%%%%%%%%%%%%%%%%%%%%%%%%%%%%%%%%%%%%%%%%%%%%%%%%%%%%%%%%
\section{Examples}

Let $\mathbf{M_3} = \langle \{0, a, b, c, 1\}, \wedge, \vee\rangle$, where $a \wedge b = a \wedge c = b \wedge c = 0$ and $a \vee b = a \vee c = b \vee c = 1.$ Let $\mathbf F := \mathbf F(x, y, z)$ denote the free lattice generated by $\{x, y, z\}$.

\begin{proposition}
Let $h\colon \mathbf{F} \twoheadrightarrow \mathbf{M_3}$ be the epimorphism that acts on the generators as follows: $x\mapsto a$, $y\mapsto b$, $z\mapsto c.$ Then $\operatorname{ker} h$ is not finitely generated.
%% If $K = \operatorname{ker}\left(\mathbf{F}\{x, y, z\} \twoheadrightarrow \mathbf{M_3}\right)$, then $K$ is not finitely generated.
\end{proposition}
\begin{proof}
Let $K := \operatorname{ker} h$, and for $u \in \{x, y, z\}$ let $C_u := u/K := \{v \in F : h(v) = h(u)\}$.
Define sequences of elements in these classes as follows:
let 
\[x_0 := x, \quad y_0 := y, \quad z_0 := z, \quad \text{ and for $i< \omega$, }\]
\[
x_{i+1} := x\vee (y_i \wedge z_i), \quad y_{i+1} := y\vee (x_i \wedge z_i), \quad z_{i+1} := z\vee (x_i \wedge y_i),
\]
\[
m_i := (x_i \wedge y_i) \vee (x_i \wedge z_i)\vee (y_i \wedge z_i).
\]
Summarizing these observations,
\begin{align*}
m_0 &= (x\wedge y) \vee (x\wedge z)\vee (y\wedge z),\\
x_1 &= x \vee (y \wedge z), \quad y_1 = y \vee (x \wedge z),  \quad z_1 = z \vee (x \wedge y),\\  
m_1 &= 
 (x_1 \wedge y_1) \vee (x_1 \wedge z_1)\vee (y_1 \wedge z_1),\\
x_2 &= x \vee \bigl\{\bigl[y \vee (x \wedge z)\bigr] \wedge \bigl[z \vee (x \wedge y)\bigr]\bigr\},\\  
y_2 &= y \vee \bigl\{\bigl[x \vee (y \wedge z)\bigr] \wedge \bigl[z \vee (x \wedge y)\bigr]\bigr\},\\  
z_3 &= z \vee (x_2 \wedge y_2)\\ 
& \vdots
\end{align*}

Let $X$ be a finite subset of $K$.  We will prove there exists $(p,q) \in K \setminus \langle X \rangle$.  
Fix $u\in \{x, y, z, m\}$ and let $\{u_i\}$ be the corresponding sequence defined above. 
Since $X$ is finite, Lemma~\ref{lem:2} implies that there exists 
$M \in \mathbb{N}$ such that for every $(p, q) \in X$ with $p, q \in C_u$, we have 
$p, q \leqslant u_M$.

\begin{subclaim}\label{claim:2.1}
   For $(p,q) \in \langle X \rangle$ and $u \in \{x, y, z\}$, the following implication holds:
\begin{equation}
  \label{eq:star}
q \leqslant u \quad \Longrightarrow \quad p\leqslant u_M.
\end{equation}
\end{subclaim}
We prove the subclaim by induction on the complexity of terms.
Fix $(p,q) \in \langle X \rangle$. Then $p, q \in C_u$ for some $u\in \{x, y, z\}$.
\begin{itemize}
\item {\bf Case 0.} Suppose $(p, q) \in X$. Then by definition of $M$ we have $p, q \leqslant u_M$.

\item {\bf Case 1.} Suppose $(p, q) = (p_1, q_1) \wedge (p_2, q_2)$, where 
  $(p_i, q_i)$ satisfies~(\ref{eq:star}) for $i = 1, 2$.    
  If $q= q_1 \wedge q_2 \leqslant u$, then, since generators in the free lattice are meet-prime (see Theorem~\ref{thm:wordprob} below), we have $q_1\leqslant u$ or $q_2\leqslant u$. Assume $q_1\leqslant u$.  Then, by the induction hypothesis, $p_1\leqslant u_M$.  Therefore, $p = p_1\wedge p_2 \leqslant u_M$, as desired.

\item {\bf Case 2.} Suppose $(p, q) = (p_1, q_1) \vee (p_2, q_2)$, where 
  $(p_i, q_i)$ satisfies~(\ref{eq:star}) for $i = 1, 2$.    
  If $q= q_1 \vee q_2 \leqslant u$, then 
  $q_i \leqslant u$ for $i = 1, 2$.
  It now follows from the induction hypothesis that $p_i\leqslant u_M$ for $i = 1, 2$, so 
  $p = p_1 \vee p_2 \leqslant u_M$, as desired.
\end{itemize}
This completes the proof of Subclaim~\ref{claim:2.1}. It now follows from Lemma~\ref{lem:1} that 
$(x, x_{M+1})\in K \setminus \langle X \rangle$, so proposition is proved.
\end{proof}

\begin{lemma}%##### Lemma 1
  \label{lem:1}
For each $u \in \{x, y, z, m\}$, the sequence $\{u_i\}$ is a strictly ascending chain; that is, $u_0 < u_1 < u_2 < \cdots$.  
\end{lemma}
\begin{proof}  We split the proof up into cases: either $u \in \{x, y, z\}$, or $u = m$.
\begin{itemize}
  \item  \textbf{Case 1.} $u \in \{x, y, z\}$.\\
  For simplicity, assume $u = x$ for the remainder of the proof of this case.  
  (Of course, the same argument applies to the case when $u$ is $y$ or $z$.) 
  Fix $n <\omega$.  We prove $x_n < x_{n+1}$.

  % \noindent {\it Subclaim.} For all $n\in \mathbb N$,
  \begin{subclaim}\label{claim:3.2.1}For all $n< \omega$,
    \begin{enumerate}
      \item 
      $x_n\in C_x$, 
      \item $x_n\ngeq y$, and $x_n\ngeq z$.
    \end{enumerate}
  \end{subclaim}
  {\it Proof of Subclaim~\ref{claim:3.2.1}.}    The first item is obvious; 
  for the second, if $x_n\geqslant y$, then $x_n\wedge y = y$, and 
    then $0 = h(x_n\wedge y) = h(y) = b$.  A similar contradiction is reached 
    if we assume $x_n\geqslant z$, so the subclaim is proved.
  
    Recall, $x_{n+1} = x_n \vee (y_n \wedge z_n)$, so $x_{n+1} > x_n$
    holds as long as $x_n \ngeqslant y_n \wedge z_n$.
    So, by way of contradiction, suppose 
    \begin{equation}\label{eq:doubledaggar}
      x_n \geqslant y_n \wedge z_n.
    \end{equation}

    Now, $y_n = y \vee (x \wedge z) \vee \cdots$, so clearly $y_n \geqslant y$.  
    Similarly, $z_n \geqslant z$.  This, together with~(\ref{eq:doubledaggar}), 
    implies $x_n \geqslant y_n \wedge z_n \geqslant y \wedge z$.  
    But then Theorem~\ref{thm:wordprob} below implies that either 
    $x_n \geqslant y$ or $x_n \geqslant z$, which contradicts Subclaim~\ref{claim:3.2.1}.

\item \textbf{Case 2.} $u = m$.\\
  We first prove that $m_0 = (x\wedge y) \vee (x\wedge z)\vee (y\wedge z)$ is strictly below 
  $m_1 = (x_1\wedge y_1) \vee (x_1\wedge z_1) \vee (y_1\wedge z_1)$.
  By symmetry, it suffices to show $x\wedge y < x_1\wedge y_1$;
  that is, $x\wedge y < \bigl[x \vee (y\wedge z)\bigr]\wedge \bigl[y \vee (x\wedge z)\bigr]$.

  Clearly 
  $x\wedge y \leqslant \bigl[x \vee (y\wedge z)\bigr]\wedge \bigl[y \vee (x\wedge z)\bigr]$. 
  Suppose 
  $x\wedge y = \bigl[x \vee (y\wedge z)\bigr]\wedge \bigl[y \vee (x\wedge z)\bigr]$. 
  Then $\bigl[x \vee (y\wedge z)\bigr]\wedge \bigl[y \vee (x\wedge z)\bigr]\leqslant x$.
  By Theorem~\ref{thm:wordprob}, the latter holds iff
  $x \vee (y\wedge z)\leqslant x$ or
  $y \vee (x\wedge z)\leqslant x$
  The first of these inequalities is clearly false, so it must be the case that 
  $y \vee (x\wedge z)\leqslant x$.  But then $y \leqslant x$, which is obviously false.  
  We conclude that
  $x\wedge y < \bigl[x \vee (y\wedge z)\bigr]\wedge \bigl[y \vee (x\wedge z)\bigr]$.
  This proves $m_0 < m_1$.

  Now fix $n < \omega$ and assume $m_n < m_{n+1}$.
  We show $m_{n+1} < m_{n+2}$.
\wjd{Complete the proof of this case.}

\medskip
  $\downarrow$ \textit{begin scratch work} $\downarrow$

  $m_{n} := (x_n \wedge y_n) \vee (x_n \wedge z_n)\vee (y_n \wedge z_n)$,

  $m_{n+1} := (x_{n+1} \wedge y_{n+1}) \vee (x_{n+1} \wedge z_{n+1})\vee (y_{n+1} \wedge z_{n+1})$,

  By the first Case above, $u_n < u_{n+1}$.
  
  $\uparrow$ \textit{end scratch work} $\uparrow$
\end{itemize}
\end{proof}


\begin{lemma}\label{lem:2} %##### Lemma 2 
For all $u \in \{x, y, z\}$ and $p \in C_u \cup C_0$ there exists $n \in \mathbb N$ such that $p\leqslant m_{u,n}$.  
\end{lemma}

\begin{proof} We prove this by induction on the complexity of $p$. 

\begin{itemize}
  \item {\bf Case 0.} $p\in \{x, y, z\}$. Then $u = p = m_{p,0}$.\\
  For the remaining cases assume $u = x$, without loss of generality.

  \item {\bf Case 1.} $p = p_1 \vee p_2$.  \\
  If $p \in C_x\cup C_0$, then $p_i \in C_x \cup C_0$ for $i = 1, 2$, and 
  the induction hypothesis yields $i$ and $j$ for which $p_1 \leqslant x_i$ 
  and $p_2 \leqslant x_j$. Letting $n = \max\{i, j\}$, we have 
  $p_1, p_2 \leqslant x_n$, from which $p = p_1 \vee p_2 \leqslant x_n$, as desired.

  \item {\bf Case 2.} $p = p_1 \wedge p_2$.  \\
  If $p \in C_x$, then we may assume $p_1 \in C_x$ and $p_2 \in C_x \cup C_0$. 
  By the induction hypothesis, there exists $n\in \mathbb N$ such that 
  $p_1 \leqslant x_n$, whence $p \leqslant p_1 \leqslant x_n$.
  If $p \in C_0$, then each $p_i$ belongs to $C_u \cup C_0$ for some 
  $u\in \{x, y, z\}$.  If $p_1 \in C_x \cup C_0$, then $p_1 \leqslant x_n$, as 
  above and we're done.  Similarly, if $p_2 \in C_x \cup C_0$.  So assume 
  $p_1 \in C_y \cup C_0$ and $p_2 \in C_z \cup C_0$. Then the induction 
  hypothesis implies that there exist $i$ and $j$ such that 
  $p_1 \leqslant y_i$ and $p_2 \leqslant z_j$. If $n = \max\{i, j\}$, then
  $p_1 \leqslant y_n$ and $p_2 \leqslant z_n$.  Then, by the above definition 
  of the sequences, we have $p_1 \wedge p_2 \leqslant y_n \wedge z_n \leqslant 
  m_n \leqslant x_{n+1}$.
\end{itemize}
\end{proof}

\subsection{Other Examples}
In each of the propositions in this section, $X$ is a finite set and 
$\mathbf{F} = \mathbf{F}(X)$ is the free lattice generated by $X$.
The symbol $F$ denotes the universe of $\mathbf{F}$.  
The proof in each case is straightforward, but tedious;
we omit proofs of the first two, and give a detailed proof of the third.
%Ex 1.
\begin{prop}\label{prop:1}
Let $X = \{x,y,z\}$, and let $\mathbf{L} = \mathbf{2}$ be the 2-element chain.    
Then the kernel of an epimorphism $h\colon \mathbf{F} \twoheadrightarrow \mathbf{L}$ 
is a finitely generated sublattice of $\mathbf{F} \times \mathbf{F}$.
\end{prop}

\begin{prop}\label{prop:2} 
Let $X = \{x, y, z\}$ and let $\mathbf{L} = \mathbf{3}$ be the 3-element chain.    
Then the kernel of an epimorphism $h\colon \mathbf{F} \twoheadrightarrow \mathbf{L}$ 
is finitely generated.
\end{prop}

\begin{prop}\label{prop:3} 
Let $n > 2$, $X = \{x_0, x_1,\dots, x_{n-1}\}$, and $\mathbf{L} = \mathbf{2} \times \mathbf{2}$.  
Let $h\colon \mathbf{F} \twoheadrightarrow \mathbf{L}$ be an epimorphism. 
Then $K = \operatorname{ker}h$ is finitely generated.  
\end{prop}
\begin{proof} 
Let the universe of $\mathbf{L} = \mathbf{2} \times \mathbf{2}$ be $\{0, a, b, 1\}$, 
where $a\vee b = 1$ and $a\wedge b = 0$. For each $y \in L$, denote by 
$X_y = X \cap h^{-1}\{y\}$ the set of generators mapped by $h$ to $y$. 
Denote the least and greatest elements of $h^{-1}\{y\}$ (if they exist) by $\ell_y$ 
and $g_y$, respectively.  For example, 
$\ell_a = \bigwedge h^{-1}\{a\} = \bigwedge \{x\in F: h(x) = a\},\;$ 
$\quad g_b = \bigvee \{x\in F : h(x) = b\},\quad$ etc. 
In the present example, the least and greatest elements exist is each case, as we now show.

% Claim 3.1.
\begin{subclaim}\label{claim:3.1}
  $h^{-1}\{a\}$ has least and greatest elements, namely 
  $\ell_a = \bigwedge (X_a \cup X_1)$ and $g_a = \bigvee (X_a\cup X_0)$.  
  (Similarly, $h^{-1}\{b\}$ has least and greatest elements, $\ell_b$ and $g_b$.)
\end{subclaim}
\noindent {\it Proof of Subclaim~\ref{claim:3.1}.}
Let $M(a):=\bigwedge (X_a\cup X_1)$ and $J(a):=\bigvee (X_a\cup X_0)$ and 
note that these values exist in $F$, since the sets involved are finite. Also, 
Then $h(M(a)) = a = h(J(a))$. Fix $r \in h^{-1}\{a\}$.  
\begin{itemize}
\item If $r \in X_a$, then $r\geqslant \bigwedge X_a \geqslant \bigwedge (X_a\cup X_1) = M(a)$.

\item If $r = s \vee t$, where $h(s) = a$ and $h(t) \in \{a, 0\}$, then    
assume (the induction hypothesis) that $s \geqslant M(a)$, and we have
$r = s\vee t \geqslant M(a)$.

\item If $r = s \wedge t$, where $h(s) = a$ and $h(t) \in \{a, 1\}$, then 
assume (the induction hypothesis) that $s, t \geqslant M(a)$, and we have 
$s \wedge t \geqslant M(a)$.
This proves that for each $r \in h^{-1}\{a\}$ we have $r \geqslant M(a)$, and 
as we noted at the outset, $M(a)\in h^{-1}\{a\}$. Therefore, $\ell_a = M(a)$ is 
the least element of $h^{-1}\{a\}$. Similarly, every $r \in h^{-1}\{a\}$ is 
below $J(a)$, so $g_a = J(a)$.  The proofs of $\ell_b = M(b)$ and $g_b = J(b)$ 
are similar.
\end{itemize}
This proves Subclaim~\ref{claim:3.1}.


\begin{subclaim}\label{claim:3.2}
$h^{-1}\{0\}$ has least and greatest elements, 
namely, $\ell_0 = \bigwedge X$ and $g_0 = g_a \wedge g_b$.
\end{subclaim}
\noindent {\it Proof of Subclaim~\ref{claim:3.2}.}
$\ell_0 = \bigwedge X$ is obvious, so we need only verify that $g_0 = g_a \wedge g_b$. 
Observe that $h(g_a \wedge g_b) = h(g_a) \wedge h(g_b) = a \wedge b = 0$, so 
$g_a \wedge g_b \in h^{-1}\{0\}$. It remains to prove that $r \leqslant g_a \wedge g_b$ holds for all $r \in h^{-1}\{0\}$.
Fix $r \in h^{-1}\{0\}$. Then $h(r \vee g_a) = h(r) \vee h(g_a) = 0 \vee a = a$, which places $r \vee g_a$ in $h^{-1}\{a\}$.  Therefore, 
by maximality of $g_a$, we have $r \vee g_a  \leqslant g_a$, whence 
$r\leqslant g_a$.  Similarly, $r\leqslant g_b$.
This proves Subclaim~\ref{claim:3.2}.

\begin{subclaim}\label{claim:3.3}
$h^{-1}\{1\}$ has least and greatest elements, namely $\ell_1 = \ell_a \vee \ell_b$ and $g_1 = \bigvee X$.
\end{subclaim}
\noindent {\it Proof of Subclaim~\ref{claim:3.3}.}
  $g_1 = \bigvee X$ is obvious, so we need only verify that $\ell_1 = \ell_a \vee \ell_b$. 
Observe that $h(\ell_a \vee \ell_b) = h(\ell_a) \vee h(\ell_b) = a \vee b = 1$, so 
$\ell_a \vee \ell_b \in h^{-1}\{1\}$. It remains to prove that $r \geqslant \ell_a \vee \ell_b$ holds for all $r \in h^{-1}\{1\}$.
Fix $r \in h^{-1}\{1\}$. Then $h(r \wedge \ell_a) = h(r) \wedge h(\ell_a) = 1 \wedge a = a$, which places $r \wedge \ell_a$ in $h^{-1}\{a\}$.  Therefore, 
by minimality of $\ell_a$, we have $r \wedge \ell_a  \geqslant \ell_a$, whence 
$r\geqslant \ell_a$.  Similarly, $r\geqslant \ell_b$.
Now let $Y = \{(x, g_p), (g_p, x), (x, \ell_p),(\ell_p, x) : p \in \{0, a, b, 1\}, x \in X_p\}$.
This proves Subclaim~\ref{claim:3.3}.

% Claim 3.4.
\begin{subclaim}\label{claim:3.4} 
  If $r \in F$ and $h(r) = p$, then $(r, \ell_p), (r, g_p) \in \langle Y \rangle$.
\end{subclaim}
\noindent {\it Proof of Subclaim~\ref{claim:3.4}.}
Either $r \in X_p$ or $r = s \wedge t$ or $r = s \vee t$.
If $r \in X_p$, then the pair belongs to $Y$ and the claim is trivial.\\[6pt]
Suppose $r = s \wedge t$.
\begin{itemize}
\item If $h(r) = 1$, then $h(s) = h(t) = 1$.  Assume 
(the induction hypothesis) that 
$\{(s, \ell_1), (s, g_1), (t, \ell_1), (t, g_1)\} \subseteq \langle Y \rangle$.
Then $(r, \ell_1) = (s \wedge t, \ell_1) =  (s, \ell_1) \wedge (t, \ell_1) \in \langle Y \rangle$. 
\item If $h(r) = a$, then (wlog) $h(s) = a$ and $h(t) \in \{a, 1\}$.  Assume 
(the induction hypothesis) that 
$\{(s, \ell_a), (s, g_a), (t, \ell_p), (t, g_p)\} \subseteq \langle Y \rangle$. 
By Claim 1, $\ell_a \leqslant \ell_1$, so $\ell_a = \ell_a \wedge \ell_1$.\\
If $h(t) = 1$, then 
\[(r, \ell_a) = (s \wedge t, \ell_a \wedge \ell_1) =  (s, \ell_a) \wedge (t, \ell_1) \in \langle Y \rangle,\]
while If $h(t) = a$, then $(r, \ell_a) = (s \wedge t, \ell_a \wedge \ell_a) =  (s, \ell_a) \wedge (t, \ell_a) \in \langle Y \rangle$.
  
      \item If $h(r) = 0$, then (wlog) that either (i) $h(s) = 0$, or (ii) $h(s) = a$, $h(t)=b$.  
      If $h(s) = 0$, then $(s, \ell_0) \in \langle Y \rangle$ implies 
      $(r, \ell_0) = (s \wedge t, \ell_0) =  (s, \ell_0) \wedge (t, \ell_p) \in \langle Y \rangle$. If 
      If $h(s) = a$, $h(t) = b$, and
      $(s, \ell_a), (t, \ell_b) \in \langle Y \rangle$, then
      $(r, \ell_0) = (s \wedge t, \ell_0) =  (s, \ell_a) \wedge (t, \ell_b) \in \langle Y \rangle$.
  \end{itemize}
  This proves that $(r, \ell_1) \in \langle Y \rangle$ if $r = s \wedge t$.
  A similar argument shows that $(r, g_p) \in \langle Y \rangle$ in each of the three subcases. 
  We have thus proved that $\{(r, \ell_1), (r, g_p)\}\subseteq \langle Y \rangle$, if $r = s \wedge t$.\\[6pt]
  Suppose $r = s \vee t$.
  \begin{itemize}
  \item If $h(r) = 0$, then $h(s) = h(t) = 0$. Assume (the induction hypothesis) that 
  $\{(s, \ell_p), (s, g_p), (t, \ell_p), (t, g_p)\} \subseteq \langle Y \rangle$. Then 
  $(r, \ell_p) = (s \vee t, \ell_p) =  (s, \ell_p) \vee (t, \ell_p) \in \langle Y \rangle$.

  \item If $h(r) = a$, then (wlog) $h(s) = a$ and $h(t) \in \{a, 0\}$.  If we assume (the induction hypothesis) that $(s, \ell_p), (s, g_p), (t, \ell_p), (t, g_p)$ belong to $\langle Y \rangle$, then $(r, \ell_p) = (s \vee t, \ell_p) =  (s, \ell_p) \vee (t, \ell_p) \in \langle Y \rangle$. 

  \item If $h(r) = 1$, then (wlog) that either (i) $h(s) = 1$, or (ii) $h(s) = a$, $h(t)=b$.\\
  In the first case, $(s, \ell_1) \in \langle Y \rangle$ implies 
  $(r, \ell_1) = (s \vee t, \ell_1) =  (s, \ell_1) \vee (t, \ell_p) \in \langle Y \rangle$.
  In the second case $h(s) = a$, $h(t) = b$, and
  $(s, \ell_a), (t, \ell_b) \in \langle Y \rangle$. Then
  $(r, \ell_1) = (s \vee t, \ell_1) =  (s, \ell_a) \vee (t, \ell_b) \in \langle Y \rangle$.

  \end{itemize}
Similarly, in each of these three subcases, we have $(r, g_p) \in \langle Y \rangle$.
This proves Subclaim~\ref{claim:3.4}, and completes the proof of Prop~\ref{prop:3}.
\end{proof}

%%%%%%%%%%%%%%%%%%%%%%%%%%%%%%%%%%%%%5
\newpage
%%%%%%%%%%%%%%%%%%%%%%%%%%%%%%%%%%%%%%%%%%%%%%%%%%%%%%%%%%%%%%%%%%%%%%%%%%%%%%%%%%%%%

\section{Background}
The notation, definitions, ideas presented below are based on those 
we learned from the book by Freese, Jezek, Nation~\cite{MR1319815}, 

% Here are some useful definitions and results from the Free Lattices book by Freese, Jezek, and Nation~\cite{MR1319815}.

\begin{definition}[length of a term] Let $X$ be a set. Each element of $X$ is a 
  term of length 1, also known as a \textbf{variable}. 
  If $t_1, \dots, t_n$ are terms of lengths $k_1, \dots, k_n$, 
  then $t_1 \vee \cdots \vee t_n$ and $t_1 \wedge \cdots \wedge t_n$ are 
  both terms of length $1+ k_1 + \cdots + k_n$.
\end{definition}

\noindent \textbf{Examples.} By the above definition, the terms 
\[x \vee y \vee z \qquad x \vee (y \vee z) \qquad (x \vee y) \vee z\]
have lengths 4, 5, and 5, respectively. Reason: variables have length 1, 
so $x \vee y \vee z$ has length $1 + 1 + 1 + 1$.  On the other hand, 
$x\vee y$ is a term of length $3$, so $(x \vee y) \vee z$ has length 
$1 + 3 + 1$. Similarly, $x \vee (y \vee z)$ has length $1 + 1 + 3$.

\begin{lemma}[\protect{\cite[Lem.~1.2]{MR1319815}}]
Let $\mathcal{V}$ be a nontrivial variety of lattices and let $\mathbf{F}_{\mathcal V}(X)$ be the relatively free lattice in $\mathcal V$ over $X$.  Then,
\begin{equation}
  \label{eq:star2}
\bigwedge S \leqslant \bigvee T \text{ implies }S \cap T \neq \emptyset
\text{ for each pair of finite subsets $S, T \subseteq X$.}
\end{equation}
\end{lemma}

\begin{lemma}[\protect{\cite[Lem.~1.4]{MR1319815}}]\label{lem:prime}
Let $\mathbf L$ be a lattice generated by a set $X$ and let $a \in L$.
\begin{enumerate}
\item 
If $a$ is join prime, then $a = \bigwedge S$ for some finite subset $S \subseteq X$.
\item 
If $a$ is meet prime, then $a = \bigvee S$ for some finite subset $S \subseteq X$.

If $X$ satisfies condition~(\ref{eq:star2}) above, then 
\item for every finite, nonempty subset $S \subset X$, $\bigwedge S$ is join prime and $\bigvee S$ is meet prime.
\end{enumerate}
\end{lemma}

\begin{corollary}[\protect{\cite[Cor.~1.5]{MR1319815}}]
Let $\mathcal V$ be a nontrivial variety of lattices and let $\mathbf{F}_{\mathcal V}(X)$ be the relatively free lattice in $\mathcal V$ over $X$.  For each finite nonempty subset $S \subseteq X$, $\bigwedge S$ is join prime and $\bigvee S$ is meet prime. In particular, every $x\in X$ is both join and meet prime.  Moreover, if $x\leqslant y$ for $x, y \in X$, then $x = y$.
\end{corollary}

\begin{theorem}[Whitman's Condition, ver.~1]  
The free lattice $\mathbf{F}(X)$ satisfies the following condition:

(W)  If $v = v_1 \wedge \cdots \wedge v_r \leqslant u_1 \vee \cdots \vee u_s = u$, 
then either $v_i \leqslant u$ for some $i$, or $v \leqslant u_j$ for some $j$. 
\end{theorem}

\begin{corollary}[\protect{\cite[Cor.~1.9]{MR1319815}}]
Every sublattice of a free lattice satisfies (W). Every element of a lattice satisfying (W) is either join or meet irreducible.
\end{corollary}

\begin{theorem}[Whitman's Condition, ver.~2]  
The free lattice $\mathbf{F}(X)$ satisfies the following condition:

(W+)  If $v = v_1 \wedge \cdots \wedge v_r \wedge x_1 \wedge \cdots
\wedge x_n \leqslant u_1 \vee \cdots \vee u_s \vee
y_1 \vee \cdots \vee y_m = u$, where $x_i, y_j\in X$, then either 
$x_i = y_j$ for some $i$ and $j$, or $v_i \leqslant u$ for some $i$, or
$v \leqslant u_j$ for some $j$. 
\end{theorem}

\begin{theorem}[\protect{\cite[Thm.~1.11]{MR1319815}}]
  \label{thm:wordprob}
If $s = s(x_1, \dots, x_n)$ and $t = t(x_1, \dots, x_n)$ are terms and $x_1, \dots, x_n \in X$, then the truth of 
\begin{equation}
  \label{eq:ast}
s^{\mathbf{F}(X)} \leqslant t^{\mathbf{F}(X)}
\end{equation}
can be determined by applying the following rules.
\begin{enumerate}
\item If $s=x_i$ and $t=x_j$, then (\ref{eq:ast}) holds iff $x_i = x_j$.
\item If $s = s_1 \vee \dots \vee s_k$ is a formal join, then (\ref{eq:ast}) holds iff $s_i^{\mathbf{F}(X)} \leqslant t^{\mathbf{F}(X)}$ for all $i$.
\item If $t = t_1 \wedge \dots \wedge t_k$ is a formal meet, then (\ref{eq:ast}) holds iff 
$s^{\mathbf{F}(X)} \leqslant t_i^{\mathbf{F}(X)}$ for all $i$.
\item If $s = x_i$ and $t = t_1 \vee \dots \vee t_k$ is a formal join, 
   then (\ref{eq:ast}) holds iff $x_i \leqslant t_j^{\mathbf{F}(X)}$ for some $j$.
\item If $s = s_1 \wedge \dots \wedge s_k$ is a formal meet and $t = x_i$, then (\ref{eq:ast}) holds iff $s_j^{\mathbf{F}(X)} \leqslant x_i$ for some $j$.
\item If $s = s_1 \wedge \dots \wedge s_k$ is a formal meet and 
and $t = t_1 \vee \dots \vee t_m$ is a formal join, then (\ref{eq:ast}) holds iff 
$s_i^{\mathbf{F}(X)} \leqslant t^{\mathbf{F}(X)}$ for some $i$
or $s^{\mathbf{F}(X)} \leqslant t_j^{\mathbf{F}(X)}$ for some $j$
\end{enumerate}
\end{theorem}

\begin{theorem}[\protect{\cite[Thm.~1.17]{MR1319815}}] For each $w \in \mathbf F (X)$ 
  there is a term of minimal length representing $w$, unique up to commutativity. 
  This term is called the \textbf{canonical form} of $w$.
\end{theorem}


Let $w \in \mathbf F(X)$ be join reducible and suppose 
$t = t_1 \vee \cdots \vee t_n$ (with $n > 1$) is the
canonical form of $w$. 
Let $w_i = t_i^{\mathbf F(X)}$.  Then $\{w_1, \dots, w_n\}$ are called the 
\textbf{canonical joinands} of $w$. We also say $w = w_1 \vee \dots \vee w_n$
\textit{canonically} and that  $w = w_1 \vee \dots \vee w_n$
is the \textbf{canonical join representation} of $w$. 
If $w$ is join irreducible, we define the canonical joinands of $w$ to be the set 
$\{w\}$. Of course the \textbf{canonical meet representation} and 
\textbf{canonical meetands} of an element in a free lattice are defined dually. 

A join representation $a = a_1 \vee \cdots \vee a_n$ in an arbitrary lattice is 
said to be a \textbf{minimal join representation} if 
$a = b_1 \vee \cdots \vee b_m$  and $\{b_1, \dots, b_m\} \ll \{a_1, \dots, a_n\}$
imply $\{a_1, \dots, a_n\} \subseteq \{b_1, \dots, b_m\}$. Equivalently, 
a join representation is minimal if it is an antichain and nonrefinable.

\begin{theorem}[\protect{\cite[Thm.~1.19]{MR1319815}}] \label{thm:1.19}
Let $w = w_1 \vee \dots \vee w_n$ canonically in $\mathbf F(X)$.
If also $w = u_1 \vee \dots \vee u_m$, then 
$\{w_1, \dots, w_n\} \ll \{u_1, \dots, u_m\}$.
Thus $w = w_1 \vee \cdots \vee w_n$ is the unique minimal join representation of $w$.
\end{theorem}

\begin{theorem}[\protect{\cite[Thm.~1.20]{MR1319815}}] 
Let $w \in \mathbf F (X)$ and let $u$ be a join irreducible element in
$\mathbf F(X)$. Then $u$ is a canonical joinand of $w$ if and only if there 
is an element $a$ such that $w = u \vee a$ and $w > v \vee a$ for every $v < u$.
\end{theorem}

\begin{definition}[up directed, continuous] A subset $A$ of a lattice $L$ is said 
  to be \textbf{up directed} if every finite subset of $A$ has an upper bound in $A$.
  It suffices to check this for pairs.  $A$ is up directed iff for all $a, b \in A$
  there exists $c\in A$ such that $a\leqslant c$ and $b\leqslant c$.  
  A lattice is \textbf{upper continuous} if whenever $A\subseteq L$ is an up 
  directed set having a least upper bound $u = \bigvee A$, then for every $b$,
  \[\bigvee_{a\in A} (a \wedge b) = 
  \bigvee_{a\in A} a \wedge b =  u \wedge b.\]
\noindent  \textbf{Down directed} and \textbf{down continuous} are defined dually.  A lattice that is 
  both up and down continuous is called \textbf{continuous}.
\end{definition}

\begin{theorem}[\protect{\cite[Thm.~1.22]{MR1319815}}]
  Free lattices are continuous.
\end{theorem}


\subsection{Bounded Homomorphisms}
\label{sec:introduction}
We continue to follow~\cite{MR1319815} very closely,
although the authors of that book indicate that the ideas in this 
subsection have their roots in Ralph McKenzie's work on nonmodular 
lattice varieties~\cite{MR0313141}, and Bjarni J\'onsson's work on 
sublattices of free lattices~\cite{MR0472614}.

If $x, y$ are elements of a lattice $L$, and if $x \leq y$, then we write
$\llbracket x, y \rrbracket$ to denote the sublattice of elements between $x$ and $y$.
That is, 
\[
\llb x, y \rrb := \{z \in L \mid x \leq z \leq y \}.
\]

Let $\mathbf K$ and $\mathbf L$ be lattices and suppose $\mathbf L$ has bottom and top elements, 
$0_{\mathbf L}$ and $1_{\mathbf L}$, resp.  
If $h \colon \mathbf K \to \mathbf L$ is a lattice homomorphism, then for each
$a \in L$ we consider the sets $h^{-1}\llb a, 1 \rrb = \{x \in K \mid h(x) \geq a\}$ and
$h^{-1}\llb 0, a \rrb = \{x \in K \mid h(x) \leq a \}$. 
When $h^{-1}\llb a, 1 \rrb$ is nonempty,
it is a filter of $\mathbf K$; dually a nonempty $h^{-1}\llb 0, a \rrb$ is an ideal. 
If $K$ is infinite, then 
$h^{-1}\llb a, 1 \rrb$ need not have a least element, nor 
$h^{-1}\llb 0, a \rrb$ a greatest element. However, 
considering when such extrema exist leads to the notion of
bounded homomorphism, which in turn helps us understand the structure of free lattices.


A lattice homomorphism $h \colon \mathbf K \to \mathbf L$ is \textbf{lower bounded} if for every 
$a \in L$, the set $h^{-1} \llb a, 1 \rrb$ is either empty or has a least element. 
The least element of a nonempty $h^{-1}\llb a, 1\rrb$ is denoted by $\beta_h(a)$, or  
by $\beta(a)$ when $h$ is clear from context.
Thus, if $h$ is a lower bounded homomorphism, then 
$\beta_h \colon \mathbf L \rightharpoonup \mathbf K$ 
is a partial mapping whose domain is an ideal of $\mathbf L$. 

Dually, $h$ is an \textbf{upper bounded} homomorphism if, whenever the set
$h^{-1}\llb 0, a \rrb$ is nonempty, it has a greatest element, denoted by 
$\alpha_h(a)$, or $\alpha (a)$.
For an upper bounded homomorphism, the domain of 
$\alpha_h \colon \mathbf L \rightharpoonup \mathbf K$ 
is clearly a filter of $\mathbf L$.  
A \textbf{bounded} homomorphism is one that is both upper and lower bounded.

These definitions simplify when $h$ is an epimorphism.
In that case $h$ is lower bounded if and only if each preimage 
$h^{-1}\{ a \}$ has a least element.
Likewise, if $L$ is finite, then $h \colon \mathbf K \to \mathbf L$ is lower 
bounded if and only if $h^{-1}\{a\}$ has a least element whenever it is nonempty. 
On the other hand, every homomorphism $h$ from a finite lattice $\mathbf K$ is bounded.

Note that $\beta$  is monotonic and a left adjoint for $h$, i.e., $a \leq h(x)$ iff
$\beta (a) ≤ x$. It then follows from a standard argument that $\beta$  is a join
preserving map on its domain: if $h^{-1}\llb a, 1 \rrb \neq \emptyset$ and
$h^{-1}\llb b, 1 \rrb \neq \emptyset$, then
$\beta (a \vee b) = \beta (a) \vee \beta (b)$.
Similarly, $\alpha$  is a right adjoint for $h$, so that $h(y) \leq a$ iff
$y \leq \alpha (a)$, and for $a, b \in \dom \alpha$,
$\alpha (a \wedge b) = \alpha (a) \wedge \alpha (b)$.
In particular, if $h$ is an epimorphism, then $\alpha$  and $\beta$  are respectively
meet and join homomorphisms of $\mathbf L$ into $\mathbf K$.
For future reference, we note that $\alpha$  and $\beta$  behave correctly with
respect to composition.

\begin{theorem}[\protect{\cite[Thm.~2.1]{MR1319815}}]
  Let $f \colon  \mathbf K \to \mathbf L$ and $g \colon \mathbf L \to \mathbf M$ be
  homomorphisms. If $f, g$ are lower bounded, then $gf \colon  \mathbf K \to  \mathbf M$ is lower
  bounded and $\beta_{gf} = \beta_f \beta_g$. Similarly, if $f, g$
  are upper bounded, then $\alpha_{gf} = \alpha_f \alpha_g$.
\end{theorem}
\begin{proof}  For $x \in K$ and $a \in M$, we have
  \[ a \leq gf (x) \quad \text{ iff } \quad \beta_g(a) \leq f (x)
  \quad \text{ iff } \quad \beta_f \beta_g (a) \leq x.
  \]
  The upper bounded case is dual.
\end{proof}

We need a way to determine whether a lattice homomorphism
$h\colon \mathbf K \to \mathbf L$
is upper or lower bounded. The most natural setting for this is
when the lattice $\mathbf K$ is finitely generated, so from now on 
we assume $\mathbf K$ is generated by a finite set $X$.
% There are no special assumptions about $\mathbf L$, nor do we assume that
% $h$ is upper or lower bounded. 
We want to analyze the sets $h^{-1}\llb a, 1 \rrb$
for $a \in L$, with the possibility of lower boundedness in mind.
(The corresponding results for
$h^{-1}\llb 0, a\rrb$
are obtained by duality.) Note that $\mathbf K$ has a greatest element 
$1_{\mathbf K} = \bigvee X$, and that $h^{-1}\llb a, 1 \rrb$ is nonempty 
if and only if $a \leq h(1_{\mathbf K})$.

Define a pair of closure operators, 
denoted by $^\wedge$ and $^\vee$, on  subsets of an arbitrary 
lattice %$\mathbf L = \langle L, \vee, \wedge\rangle$ as follows: for each 
$L$ as follows: for each $A\subseteq L$,
\[
A^\wedge := \{\bigwedge B \mid B \text{ is a finite subset of } A\}.
\]
We adopt the following convention: if 
$\mathbf L$ has a greatest element 
$1_{\mathbf L}$, then $\bigwedge \emptyset = 1_{\mathbf L}$, and we include 
this in $A^\wedge$ for every $A \subseteq L$; otherwise, $\bigwedge \emptyset$ is undefined.  
The set $A^\vee$ is defined dually.
% If $\mathbf K = \langle K, \vee, \wedge\rangle$ is a lattice generated by a 
% finite set $X$, then 

We can write $K$ as the union of a chain of subsets 
$H_0\subseteq H_1 \subseteq \cdots$ defined inductively by setting 
$H_0 := X^\wedge$ and $H_{k+1} := (H_k)^{\vee \wedge}$, for all $k\geqslant 0$. 
By induction, each $H_n = X^{\wedge(\vee\wedge)^n}$ is a finite meet-closed 
subset of $K$, and $\bigcup H_n = K$, since $X$  generates $K$.

Let $h \colon \mathbf K \to \mathbf L$ be an epimorphism and, 
for each $y \in L$ and $k< \omega$, define
\[
\beta_k(y) = \bigwedge \{w \in H_k : h(w) \geqslant a\}.
\]

\begin{theorem}[\protect{\cite[Thm.~2.2]{MR1319815}}]\label{thm:2.2}
   Let $\mathbf K$ be finitely generated, and let $h \colon \mathbf K \to \mathbf L$ be a lattice
  homomorphism. If $a \leq h(1_{\mathbf K})$, then
\begin{enumerate}
\item $j \leq k$ implies $\beta_j(a) \geq \beta_k(a)$,
\item  $\beta_k(a)$ is the least element of $H_k \cap h^{-1}\llb a, 1 \rrb$,
\item  $h^{-1}\llb a, 1 \rrb = \bigcup_{k\in \omega}\llb \beta_k(a), 1 \rrb$.
\end{enumerate}  

\end{theorem}  

%%%%%%%%%%%%%%%%%%%%

\subsubsection{Minimal join covers and refinement}

A \textbf{join cover} of the element $a \in L$ is a finite
subset $S \subseteq L$ such that $a \leq \bigvee S$. 
A join cover $S$ of $a$ is \textit{nontrivial} if $a \nleq s$ for all
$s \in S$. Let $\mathcal C (a)$ be the set of all nontrivial join 
covers of $a$ in $\mathbf L$.

\begin{theorem}[\protect{\cite[Thm.~2.3]{MR1319815}}]\label{thm:2.3}
Suppose $\mathbf K$ is generated by a finite set $X$,
$h \colon \mathbf K \to \mathbf L$ is a homomorphism, $a \leq h(1_{\mathbf K})$, and $k\in \omega$. Then,
\begin{align*}
    \beta_0(a) &= \bigwedge \{x \in X \mid h (x) \geq a\},\\
    \beta_{k+1}(a) &= \beta_0(a) \wedge 
    \bigwedge_{\stackrel{S \in \mathcal C(a)}{\bigvee S \leq h(1_{\mathbf K})}}
    \bigvee_{s \in S} \beta_k(s).
\end{align*}
\end{theorem}
In general, the expression for $\beta_{k+1}(a)$ has some redundant terms, which we can 
exclude if $\mathbf L$ satisfies a weak finiteness condition that we now define.
For finite subsets $A$, $B \subseteq L$, we say $A$ \textit{join refines
$B$} and write $A \ll B$ if for every $a \in A$ there exists $b \in B$ 
with $a \leq b$. Theorem~\ref{thm:1.19} states that if $w \in \mathbf F(X)$ and 
$w = \bigvee B$, then the set of canonical joinands of $w$ join
refines $B$. 

Define a \textbf{minimal nontrivial join cover} of $a \in L$ to be a 
nontrivial join cover $S$ with the property that whenever 
$a\leq \bigvee T$ and $T \ll S$, then
$S \subseteq T$. This formulation is equivalent to our more 
intuitive notion of what minimality ought to mean: \textit{a nontrivial join cover $S$ of a is minimal 
if and only if}
\begin{enumerate}
\item \textit{$S$ is an antichain of join irreducible elements of $L$, and}
\item \textit{if an element of $S$ is deleted or replaced by a (finite) 
  set of strictly smaller elements, then the resulting set is no longer 
  a join cover of $a$.}
\end{enumerate}

Let $\mathcal M(a)$ denote the set of minimal nontrivial join covers of $a \in L$.
Let us say that $\mathbf L$ has the \textbf{minimal join cover refinement property} if for each
$a \in L$, $\mathcal M(a)$ is finite and every nontrivial join cover of $a$ refines 
to a minimal one. Clearly every finite lattice has the minimal join cover refinement 
property, but so do free lattices.  The following reformulation of Theorem~\ref{thm:2.3} 
simplifies the calculation of $\beta_k$ whenever the minimal join cover refinement property holds.

\begin{theorem}[\protect{\cite[Thm.~2.4]{MR1319815}}]\label{thm:2.4}
Let $\mathbf K$ be generated by the finite set $X$, and let 
$h \colon \mathbf K \to \mathbf L$ be a lattice homomorphism. 
If $\mathbf L$ has the minimal join cover refinement property, then
for each $a \in L$ with $a \leq h(1_{\mathbf K})$ and $k\in \omega$, we have
\begin{align*}
    \beta_0(a) &= \bigwedge \{x \in X \mid h (x) \geq a\},\\
    \beta_{k+1}(a) &= \beta_0(a) \wedge 
    \bigwedge_{\stackrel{S \in \mathcal M (a)}{\bigvee S \leq h(1_{\mathbf K})}}
    \bigvee_{s \in S} \beta_k(s).
\end{align*}
\end{theorem}

We now look for a condition on the $\mathbf L$ that will insure the 
homomorphism $h \colon \mathbf K \to \mathbf L$ is lower bounded. From 
Theorem~\ref{thm:2.2}, this will happen iff for each 
$a \leq h(1_{\mathbf K})$ there exists $N \in \omega$ such
that $\beta_n(a) = \beta_N(a)$ for all $n \geq N$. In this case, 
$\beta(a) = \beta_N(a)$ for all $a\in \dom \beta = \llb 0, h(1_{\mathbf K})\rrb$, 
where $N$ depends upon $a$.

\begin{fact}
\label{fact:1}
The following are equivalent:
\wjd{To do: verify this!}
\begin{enumerate}
  \item $h$ is not lower bounded;
  \item $(\exists y_0 \in L)(\forall N)(\exists n> N)\, \beta_n(a) \neq \beta_N(a)$;
  \item $(\exists y_0 \in L)(\exists N)(\forall n> N) \beta_n(a) \neq \beta_N(a)$.
\end{enumerate}
\end{fact}

Let $D_0(\mathbf L)$ be the set of all join prime elements of $\mathbf L$, 
i.e., the set of elements that have no nontrivial join cover. 
Given $D_k(\mathbf L)$, define $D_{k+1}(\mathbf L)$ to be the set of $p \in L$ 
such that every nontrivial join cover of $p$ refines to a join cover 
contained in $D_k(\mathbf L)$, i.e., $p \leq \bigvee S$ nontrivially implies 
there exists $T \ll S$ with $p \leq \bigvee T$ and $T \subseteq D_k(\mathbf L)$. 
Note that if $L$ has the minimal join cover refinement property, then 
$p \in D_{k+1}(\mathbf L)$ iff every minimal nontrivial join
cover of $p$ is contained in $D_k(\mathbf L)$.

The definition clearly implies 
$D_0(\mathbf L) \subseteq D_1(\mathbf L) \subseteq D_2(\mathbf L) \subseteq \cdots$.  
Let $D(\mathbf L) = \bigcup D_i$. For $a \in D(\mathbf L)$, define the $D$-\textbf{rank}, $\rho(a)$, 
to be the least integer $N$ such that $a \in D_N(\mathbf L)$; for $a \notin D(L)$, 
$\rho (a)$ is undefined. 
The duals of $D_k(\mathbf L)$, $D(\mathbf L)$, and $\rho(a)$ are denoted by $D^d_k(\mathbf L)$, 
$D^d(\mathbf L)$, and $\rho^d(a)$, respectively.

% We are interested in the property D(L) = L, i.e., when every element is in
% Dfc(L) for some k. If A is a finite subset of Dk(L), then \J A G D/c+i(L). Indeed, if
% V A < \/ S for some set 5, then, for a G A, S is a join cover of a. Since a G D/C(L),
% if this is a nontrivial join cover, there is a set Ua C D/c_i(L) with Ua <^ S and
% a < \jUa- For those a's for which 5 is a trivial join cover, set Ua = {a}. Let
% U = \jUa- Then U C D/C(L) and is a join cover of V-^ and U <C 5, which
% shows that \J A is in D/c+i(L). Hence D(L) is a join subsemilattice of L. On the
% other hand, it is easy to see that in a lattice with the minimal join cover refinement
% property, every element is a join of join irreducibles. Combining these observations,
% we obtain the following useful equivalence.
% Lemma 2.5. For a lattice with the minimal join cover refinement property,
% D(L) = L if and only if J(L) C D(L).
% For example, if K is a finite distributive lattice, then D(K) = Di(K) = K
% because every join irreducible element is join prime. Likewise, it is easy to see that
% for the pentagon N5 we have D(N5) = Di(Ns) = AT5. On the other hand, for the
% diamond we have D(M3) = {0} ^ M3.
% We need to analyze the property D(L) = L at length, but first let us show that
% it does what we need it to do.
% Theorem 2.6. Let K be a finitely generated lattice and let h : K —► L be a
% homomorphism. If D(L) = L, then h is lower bounded.
% The proof of this theorem yields a slightly stronger statement which, because
% we will refer to it a couple of times, we give as a separate lemma.
% Lemma 2.7. Let K be a finitely generated lattice and let h : K —► L be a
% homomorphism. For all w G K and a G Dn(L),
% (*) h(w) > a if and only if w > (3n(a)>
% Thus in the case when D(L) = L, we obtain that h is lower bounded with
% /3(a) = /?p(a)(a) for all a G L.
% Proof of Lemma 2.7. We of course use Theorem 2.3. In order to prove (*)
% by induction, for each nGw define
% Wn = {w G K : for all a G Dn(L), h(w) > a iff w > 0n(a)}.
% We need to show that Wn — K for all n. Clearly the generating set X for K
% is contained in Wn, and Wn is closed under meets for each n. So it remains to
% show that Wn is closed under joins. For n = 0 this follows immediately from the
% definition of D0(L) as the set of all join prime elements.
%%%%%%%%%%%%%%%%%%%%%%%%%%%%%%%%%%%%%%%%%%%%%%%%%%%%%%%%%%%%%%%%%%%%%%%%%%%%%%%%%%%%%%%%%%%%%%
\newpage
%%%%%%%%%%%%%%%%%%%%%%%%%%%%%%%%%%%%%%%%%%%%%%%%%%%%%%%%%%%%%%%%%%%%%%%%%%%%%%%%%%%%%%%%%%%%%%

%%%%%%%%%%%%%%%%%%%%%%%%%%%%%%%%%%%%%%%%%%%%%%%%%%%%%%%%%%%%%%%%%%%%%%%%%%%%%%%%%%%%%%%%%%%%%%
\newpage
%%%%%%%%%%%%%%%%%%%%%%%%%%%%%%%%%%%%%%%%%%%%%%%%%%%%%%%%%%%%%%%%%%%%%%%%%%%%%%%%%%%%%%%%%%%%%%

\section{Miscellaneous Notes}
\noindent Let $K$ be a finite subset of $\ker h$.  Since $K$ is finite, we can find an $N < \omega$ such that for all $\binom{p}{q} \in K$, the following implications are satisfied:
\begin{align}
p &\leqslant x \quad \Longrightarrow \quad q \leqslant x_N\nonumber\\
p &\leqslant y \quad \Longrightarrow \quad q \leqslant y_N\label{eq:imp1}\\
p &\leqslant z \quad \Longrightarrow \quad q \leqslant z_N\nonumber
\end{align}
\begin{align}
p &\leqslant x \vee (y \wedge z) \quad \Longrightarrow \quad q \leqslant x_{N+1}\nonumber\\
p &\leqslant y \vee (x \wedge z) \quad \Longrightarrow \quad q \leqslant y_{N+1}\label{eq:imp2}\\
p &\leqslant z \vee (x \wedge y) \quad \Longrightarrow \quad q \leqslant z_{N+1}\nonumber
\end{align}

\medskip

\noindent {\bf Claim 4.3} If $N$ is chosen as just described, and if $\binom{p}{q} \in \langle K \rangle$ then the implications~\ref{eq:imp1} and~\ref{eq:imp2} hold.

\begin{proof}
  As usual, we proceed by induction on term complexity.
If $\binom{p}{q} \in K$, then by choice of $N$, there is nothing to prove.

\medskip
\noindent {\it Case 1.} Suppose $\binom{p}{q} = \binom{p_1}{q_1} \vee \binom{p_2}{q_2}$, where $\binom{p_1}{q_1}$ and $\binom{p_2}{q_2}$ satisfy~(\ref{eq:imp1}) and~(\ref{eq:imp2}). We show that $\binom{p}{q}$ satisfies these two implications as well.
Recall, in the notation above, $x_1:=x \vee (y \wedge z)$.


Assume $p\leqslant x_1$. We show $q \leq x_{N+1}$.
Since $p = p_1 \vee p_2 \leq x_1$, we have 
$p_1 \leqslant x_1$ and 
$p_2 \leqslant x_1$, so by the induction hypothesis,  
$q_1 \leqslant x_{N+1}$ and 
$q_2 \leqslant x_{N+1}$.  Therefore, $q = q_1 \vee q_2 \leq x_{N+1}$, as desired.

Now assume $p\leqslant x$. We show $q\leqslant x_N$. 
Since $p = p_1 \vee p_2 \leq x$, we have 
$p_1 \leqslant x$ and 
$p_2 \leqslant x$, so by the induction hypothesis,  
$q_1 \leqslant x_{N}$ and 
$q_2 \leqslant x_{N}$.  Therefore, $q = q_1 \vee q_2 \leq x_{N}$, as desired.


\medskip
\noindent {\it Case 2.} Suppose $\binom{p}{q} = \binom{p_1}{q_1} \wedge \binom{p_2}{q_2}$, where $\binom{p_1}{q_1}$ and $\binom{p_2}{q_2}$ satisfy~(\ref{eq:imp1}) and~(\ref{eq:imp2}). 

Assume $p\leqslant x_1 = x \vee (y \wedge z)$. 
We must show $q\leqslant x_{N+1}$. 
Since $p_1 \wedge p_2 \leqslant x_1$, then according to Theorem~\ref{thm:wordprob}, at least one of the following inequalities must hold:
\begin{enumerate}
  \item $p_1 \leqslant x_1$;
  \item $p_2 \leqslant x_1$;
  \item $p_1 \wedge p_2 \leqslant x$;
  \item $p_1 \wedge p_2 \leqslant y \wedge z$.
\end{enumerate}
By the induction hypothesis, (1) implies $q_1 \leq x_{N+1}$  and (2) implies $q_2 \leq x_{N+1}$.  In either case, $q = q_1 \wedge q_2 \leq x_{N+1}$, as desired.  In case (3), Theorem~\ref{thm:wordprob} implies that either $p_1 \leq x$ or $p_2 \leq x$, since $x$ is a generator. Therefore, $q_1 \leq x_N$ or $q_2 \leq x_N$ and we conclude that 
$q \leq x_N \leq x_{N+1}$, as desired.  It remains to prove $q \leq x_{N+1}$ for the final case in which $p_1 \wedge p_2 \leq y \wedge z$. 

If $p_1 \wedge p_2 \leq y \wedge z$, then 
$p_1 \wedge p_2 \leq y$ and $p_1 \wedge p_2 \leq z$. Therefore, both of the following disjunctions hold:
\begin{itemize}
  \item $p_1 \leq y$ or $p_2 \leq y$, and 
  \item $p_1 \leq z$ or $p_2 \leq z$. 
\end{itemize}
If $p_1 \leq y$ and $p_1 \leq z$, then $p_1 \leq x \vee (y\wedge z) = x_1$, so $q_1 \leq x_{N+1}$, so $q = q_1\wedge q_2 \leq x_{N+1}$, as desired.
Similarly, if $p_2 \leq y$ and $p_2 \leq z$, the desired conclusion holds.
Finally, consider the case in which $p_1 \leq y$ and $p_2 \leq z$. In this case $q_1 \leq y_N$ and $q_2 \leq z_N$.  Therefore, $q = q_1 \wedge q_2 \leq y_N \wedge z_N \leq x_N \vee (y_N\wedge z_N) = x_{N+1}$, as desired.
\end{proof}





\bibliographystyle{alphaurl}
\bibliography{inputs/refs}


%%%%%%%%%%%%%%%%%%%%%%%%%%%%%%%%%%%%%%%%%%%%%%%%%%%%%%%%%%%%%%%%%%%%%%%%%%
%%%%%%%%%%%%%%%%%%%%%%%%%%%%%%%%%%%%%%%%%%%%%%%%%%%%%%%%%%%%%%%%%%%%%%%%%%
%%%%%%%%%%%%%%%%%%%  END OF DOCUMENT %%%%%%%%%%%%%%%%%%%%%%%%%%%%%%%%%%%%%
%%%%%%%%%%%%%%%%%%%%%%%%%%%%%%%%%%%%%%%%%%%%%%%%%%%%%%%%%%%%%%%%%%%%%%%%%%
%%%%%%%%%%%%%%%%%%%%%%%%%%%%%%%%%%%%%%%%%%%%%%%%%%%%%%%%%%%%%%%%%%%%%%%%%%

\end{document}

%%%%%%%%%%%%%%%%%%%%%%%%%%%%%%%%%%%%%%%%%%%%%%%%%%%%%%%%%%%%%%%%%%%%%%%%%%
%%%%%%%%%%%%%%%%%%%%%%%%%%%%%%%%%%%%%%%%%%%%%%%%%%%%%%%%%%%%%%%%%%%%%%%%%%
%%%%%%%%%%%%%%%%%%%  END OF DOCUMENT %%%%%%%%%%%%%%%%%%%%%%%%%%%%%%%%%%%%%
%%%%%%%%%%%%%%%%%%%%%%%%%%%%%%%%%%%%%%%%%%%%%%%%%%%%%%%%%%%%%%%%%%%%%%%%%%
%%%%%%%%%%%%%%%%%%%%%%%%%%%%%%%%%%%%%%%%%%%%%%%%%%%%%%%%%%%%%%%%%%%%%%%%%%
































\subsection{Bounded Homomorphisms}
\label{sec:bound-homom}

Following Freese, Jezek, Nation~\cite{MR1319815}, we define a pair of closure operators, denoted by superscripts $^\wedge$ and $^\vee$, on  subsets of an arbitrary lattice
$\mathbf L = \langle L, \vee, \wedge\rangle$ as follows: For each $A\subseteq L$, let
\[
A^\wedge = \{\bigwedge B : B \text{ is a finite subset of } A\}.
\]
We adopt the convention that if $\mathbf L$ has a greatest element $1_{\mathbf L}$, then 
$\bigwedge \emptyset = 1_{\mathbf L}$, and we include this in $A^\wedge$ for every $A \subseteq L$.
(For lattices without a greatest element, $\bigwedge \emptyset$ is undefined.)  The set $A^\vee$ is defined dually.

If $\mathbf K = \langle K, \vee, \wedge\rangle$ is a lattice generated by a finite set $X$, then we can write $K$ as a union of a chain of subsets $H_0\subseteq H_1 \subseteq \cdots$ defined inductively by setting $H_0 = X^\wedge$ and $H_{k+1} = (H_k)^{\vee \wedge}$, for all $k\geqslant 0$. By induction, each $H_n = X^{\wedge(\vee\wedge)^n}$ is a finite meet-closed subset of $K$, and $\bigcup H_n = K$, since $X$  generates $\mathbf X$.

Let $h \colon \mathbf K \to \mathbf L$ be a lattice epimorphism and, for each $y \in L$ and $k< \omega$, define
\[
\beta_k(y) = \bigwedge \{w \in H_k : h(w) \geqslant a\}.
\]

On page 30 of~\cite{MR1319815}, immediately after Theorem 2.4, the authors make the following remark, which is a crucial ingredient of our proof:

``...[$h$ is lower bounded] if and only if for each $a \leqslant h(1_{\mathbf K})$ there exists $N\in \omega$ such that $\beta_n(a) = \beta_N(a)$ for all $n\geqslant N$.''

Equivalently, 
\begin{align}
h \text{ is not lower bounded } &\iff (\exists y_0 \in L)(\forall N)(\exists n> N)\, \beta_n(a) \neq \beta_N(a) \nonumber \\
&\iff (\exists y_0 \in L)(\exists N)(\forall n> N) \beta_n(a) \neq \beta_N(a).\nonumber
\end{align}
