%% FILE: diffTerm.tex
%% AUTHOR: William DeMeo, Peter Mayr, Nik Ruskuc
%% DATE: 14 May 2018
%% COPYRIGHT: (C) 2018 DeMeo, Mayr, Ruskuk

%%%%%%%%%%%%%%%%%%%%%%%%%%%%%%%%%%%%%%%%%%%%%%%%%%%%%%%%%%
%%                         BIBLIOGRAPHY FILE            %%
%%%%%%%%%%%%%%%%%%%%%%%%%%%%%%%%%%%%%%%%%%%%%%%%%%%%%%%%%%
%% The `filecontents` command will crete a file in the inputs directory called
%% refs.bib containing the references in the document, in case this file does
%% not exist already.
%% If you want to add a BibTeX entry, please don't add it directly to the
%% refs.bib file.  Instead, add it in this file between the
%% \begin{filecontents*}{refs.bib} and \end{filecontents*} lines
%% then delete the existing refs.bib file so it will be automatically generated
%% again with your new entry the next time you run pdfaltex.
\begin{filecontents*}{inputs/refs.bib}
@book {MR1319815,
    AUTHOR = {Freese, Ralph and Je{\v{z}}ek, Jaroslav and Nation, J. B.},
     TITLE = {Free lattices},
    SERIES = {Mathematical Surveys and Monographs},
    VOLUME = {42},
 PUBLISHER = {American Mathematical Society, Providence, RI},
      YEAR = {1995},
     PAGES = {viii+293},
      ISBN = {0-8218-0389-1},
   MRCLASS = {06B25 (06-02 06-04 06B20 68Q25)},
  MRNUMBER = {1319815 (96c:06013)},
MRREVIEWER = {T. S. Blyth},
       DOI = {10.1090/surv/042},
       URL = {http://dx.doi.org/10.1090/surv/042},
}
@article {MR3239624,
    AUTHOR = {Valeriote, M. and Willard, R.},
     TITLE = {Idempotent {$n$}-permutable varieties},
   JOURNAL = {Bull. Lond. Math. Soc.},
  FJOURNAL = {Bulletin of the London Mathematical Society},
    VOLUME = {46},
      YEAR = {2014},
    NUMBER = {4},
     PAGES = {870--880},
      ISSN = {0024-6093},
   MRCLASS = {08A05 (06F99 68Q25)},
  MRNUMBER = {3239624},
       DOI = {10.1112/blms/bdu044},
       URL = {http://dx.doi.org/10.1112/blms/bdu044},
}
@book {MR2839398,
    AUTHOR = {Bergman, Clifford},
     TITLE = {Universal algebra},
    SERIES = {Pure and Applied Mathematics (Boca Raton)},
    VOLUME = {301},
      NOTE = {Fundamentals and selected topics},
 PUBLISHER = {CRC Press, Boca Raton, FL},
      YEAR = {2012},
     PAGES = {xii+308},
      ISBN = {978-1-4398-5129-6},
   MRCLASS = {08-02 (06-02 08A40 08B05 08B10 08B26)},
  MRNUMBER = {2839398 (2012k:08001)},
MRREVIEWER = {Konrad P. Pi{\'o}ro},
}
@article{Freese:2009,
    AUTHOR = {Freese, Ralph and Valeriote, Matthew A.},
    TITLE = {On the complexity of some {M}altsev conditions},
    JOURNAL = {Internat. J. Algebra Comput.},
    FJOURNAL = {International Journal of Algebra and Computation},
    VOLUME = {19},
    YEAR = {2009},
    NUMBER = {1},
    PAGES = {41--77},
    ISSN = {0218-1967},
    MRCLASS = {08B05 (03C05 08B10 68Q25)},
    MRNUMBER = {2494469 (2010a:08008)},
    MRREVIEWER = {Clifford H. Bergman},
    DOI = {10.1142/S0218196709004956},
    URL = {http://dx.doi.org/10.1142/S0218196709004956}
  }
@article {MR3076179,
    AUTHOR = {Kearnes, Keith A. and Kiss, Emil W.},
     TITLE = {The shape of congruence lattices},
   JOURNAL = {Mem. Amer. Math. Soc.},
  FJOURNAL = {Memoirs of the American Mathematical Society},
    VOLUME = {222},
      YEAR = {2013},
    NUMBER = {1046},
     PAGES = {viii+169},
      ISSN = {0065-9266},
      ISBN = {978-0-8218-8323-5},
   MRCLASS = {08B05 (08B10)},
  MRNUMBER = {3076179},
MRREVIEWER = {James B. Nation},
       DOI = {10.1090/S0065-9266-2012-00667-8},
       URL = {http://dx.doi.org/10.1090/S0065-9266-2012-00667-8},
}
@misc{william_demeo_2016_53936,
  author       = {DeMeo, William and Freese, Ralph},
  title        = {AlgebraFiles v1.0.1},
  month        = May,
  year         = 2016,
  doi          = {10.5281/zenodo.53936},
  url          = {http://dx.doi.org/10.5281/zenodo.53936}
}
@article{FreeseMcKenzie2016,
	Author = {Freese, Ralph and McKenzie, Ralph},
	Date-Added = {2016-08-22 19:43:56 +0000},
	Date-Modified = {2016-08-22 19:45:50 +0000},
	Journal = {Algebra Universalis},
	Title = {Mal'tsev families of varieties closed under join or Mal'tsev product},
	Year = {to appear}
}
@misc{UACalc,
	Author = {Ralph Freese and Emil Kiss and Matthew Valeriote},
	Date-Added = {2014-11-20 01:52:20 +0000},
	Date-Modified = {2014-11-20 01:52:20 +0000},
	Note = {Available at: {\verb+www.uacalc.org+}},
	Title = {Universal {A}lgebra {C}alculator},
	Year = {2011}
}
\end{filecontents*}
%:biblio
%\documentclass[12pt]{amsart}
\documentclass[12pt,reqno]{amsart}

%%%%%%% wjd: added these packages vvvvvvvvvvvvvvvvvvvvvvvvv
% PAGE GEOMETRY
% These settings are for letter format
\def\OPTpagesize{8.5in,11in}     % Page size
\def\OPTtopmargin{0.75in}     % Margin at the top of the page
\def\OPTbottommargin{0.75in}  % Margin at the bottom of the page
%% \def\OPTinnermargin{0.5in}    % Margin on the inner side of the page
\def\OPTinnermargin{1in}    % Margin on the inner side of the page
\def\OPTbindingoffset{0.35in} % Extra offset on the inner side
%% \def\OPToutermargin{0.75in}   % Margin on the outer side of the page
\def\OPToutermargin{1in}   % Margin on the outer side of the page
% \usepackage[papersize={\OPTpagesize},
%             twoside,
%             includehead,
%             top=\OPTtopmargin,
%             bottom=\OPTbottommargin,
%             inner=\OPTinnermargin,
%             outer=\OPToutermargin,
%             bindingoffset=\OPTbindingoffset]{geometry}
\usepackage{url,amssymb,enumerate,tikz,scalefnt}
\usepackage[normalem]{ulem} % for \sout (strikeout)   wjd: could remove this in final draft
\usepackage[colorlinks=true,urlcolor=blue,linkcolor=blue,citecolor=blue]{hyperref}
\usepackage{algorithm2e}

\newcommand{\mysetminus}{\ensuremath{-}}
%% uncomment the next line if we want to revert to the "set" minus notation
%% \renewcommand{\mysetminus}{\ensuremath{\setminus}}

\usepackage[yyyymmdd,hhmmss]{datetime}
\usepackage{background}
\backgroundsetup{
  position=current page.east,
  angle=-90,
  nodeanchor=east,
  vshift=-1cm,
  hshift=8cm,
  opacity=1,
  scale=1,
  contents={\textcolor{gray!80}{WORK IN PROGRESS.  DO NOT DISTRIBUTE. (compiled on \today\ at \currenttime)}}
}
%%%%%%  (end wjd addition of packages)


\usepackage{pdfcomment}
\usepackage{color}
\usepackage{amsmath}
\usepackage{amssymb}
\usepackage{amsfonts}
\usepackage{mathtools}
\usepackage{amscd}
%% \usepackage{exers}
\usepackage{inputs/wjdlatexmacs}

\usepackage[mathcal]{euscript}
\usepackage{comment}
\usepackage{tikz}
\usetikzlibrary{math} %needed tikz library

\renewcommand{\th}[2]{#1\mathrel{\theta}#2}
\newcommand{\infixrel}[3]{#2\mathrel{#1}#3}


\newtheorem{theorem}{Theorem}
\newtheorem{lemma}[theorem]{Lemma}
\newtheorem{corollary}[theorem]{Corollary}
\newtheorem{prop}[theorem]{Proposition}
\newtheorem{conjecture}[theorem]{Conjecture}
\theoremstyle{definition}
\newtheorem{definition}[theorem]{Definition}
\newtheorem{example}[theorem]{Example}
\newtheorem{fact}[theorem]{Fact}
\newtheorem{remark}{Remark}
\newtheorem*{remarks}{Remarks}
\newtheorem*{rem}{Remark}
\newtheorem{prob}{Problem}

% \title[A test for a difference term]{A polynomial time test for a
% difference term in an idempotent variety}
% \author[DeMeo]{William DeMeo}
% \address[William DeMeo]{
% Department of Mathematics\\
% University of Hawaii\\
% Honolulu, Hawaii\\
% 96822 USA}
% \email[William DeMeo]{demeo@math.hawaii.edu}
% \author[Freese]{Ralph Freese}
% \address[Ralph Freese]{
% Department of Mathematics\\
% University of Hawaii\\
% Honolulu, Hawaii\\
% 96822 USA}
% \email[Ralph Freese]{ralph@math.hawaii.edu}


\title[Kernels of Lattice Epimorphisms]{Kernels of epimorphisms of finitely generated free lattices}
\author[W.~DeMeo]{William DeMeo}
\email{williamdemeo@gmail.com}
%% \urladdr{http://williamdemeo.github.io}
%% \address{University of Colorado\\Mathematics Dept\\Boulder 80309\\USA}

\author[P.~Mayr]{Peter Mayr}
%% \email{}\urladdr{}
%% \address{University of Colorado\\Mathematics Dept\\Boulder 80309\\USA}

\author[N.~Ruskuc]{Nik Ruskuc}
%% \email{}\urladdr{}
%% \address{University of St. Andrews\\Mathematics Dept\\St. Andrews, Scottland}

%% \thanks{The first and second authors were supported by the National
%% Science Foundation under Grant No...}

\date{\today}

\begin{document}

\maketitle

\section{Main Theorem}
\label{sec:introduction}

Let $X$ be a finite set and $\mathbf F := \mathbf F(X)$ the free lattice generated by $X$.

\begin{theorem}
Suppose $\mathbf L = \langle L, \wedge, \vee\rangle$ is a finite lattice and $h\colon \mathbf{F} \twoheadrightarrow \mathbf{L}$ a lattice epimorphism.
Then $h$ is bounded if and only if $\ker h$ is finitely generated.
\end{theorem}

\begin{proof}
($\Rightarrow$) Assume $h$ is bounded.  That is, the preimage of each $y\in L$ under $h$ is bounded.  For each $y\in L$, let $\alpha y= \bigvee h^{-1}\{y\}$ and $\beta y = \bigwedge h^{-1}\{y\}$ denote the greatest and least elements of $h^{-1}\{y\}$, respectively (both of which exist by the boundedness assumption).  Observe that $h \alpha h = h$, and $h \beta h = h$. In fact, $\alpha$ and $\beta$ are adjoint to $h$. Indeed, it is easy to see that
\[
h x \leqslant y \quad \Leftrightarrow \quad x \leqslant \alpha y,
\]
\[
y \leqslant h x \quad \Leftrightarrow \quad \beta y \leqslant x.
\]

For each $y \in L$, let $X_y := X\cap h^{-1}\{y\}$, the set of generators that lie in the inverse image of $y$ under $h$.
Let $G$ be the (finite) set of pairs in $\mathbf F \times \mathbf F$ defined as follows:
\[
G = \bigcup_{y \in L}\{(x, \alpha y), (\alpha y, x), (x, \beta y), (\beta y, x), (\alpha y, \beta y), (\beta y, \alpha y) : x \in X_y\}.
\]
We claim that $G$ generates $\ker h$.  To prove this, we first show, by induction on term complexity, that for every $y \in L$, for every $r \in h^{-1}\{y\}$, the pairs $(r,\alpha y)$ and $(r,\beta y)$ belong to the sublattice $\langle G \rangle \leqslant \mathbf F \times \mathbf F$ generated by $G$.

\noindent {\it Case 0.} If $r \in X$, then $(r,\alpha y)$ and $(r,\beta y)$ belong to $G$ itself, so there's nothing to prove.  

\noindent {\it Case 1.} Suppose $r = s \vee t$  and assume (the induction hypothesis) that
$(s, \alpha {h(s)})$, $(s, \beta{h(s)})$, $(t, \alpha {h(t)})$, and $(t, \beta{h(t)})$ belong to $\langle G \rangle$. Then $y = h (r) = h(s\vee t) = h (s)\vee h(t)$, so 
\[
h(\alpha {h(s)} \vee \alpha {h(t)})= h\alpha h(s) \vee h\alpha h(t)=
h(s) \vee h(t) = y.
\]
Similarly, $h(\beta{h(s)} \vee \beta {h(t)})= h(s) \vee h(t) = y$.
Therefore, 
\[
\beta y \leqslant \beta h(s) \vee \beta h(t) \leqslant \alpha {h(s)} \vee \alpha {h(t)} \leqslant \alpha y.
\]


Also, $r \leqslant \alpha y$, so $r = \alpha y \wedge (s\vee t)$.  Taken together, these observations yield
\begin{align}
\left(\begin{array}{c} r \\ \beta y\end{array}\right) &= 
\left(\begin{array}{c} \alpha y \wedge (s\vee t) \\ \beta y\end{array}\right) = 
  \left(\begin{array}{c} \alpha y \wedge (s\vee t) \\ \beta y \wedge (\beta {h(s)} \vee \beta {h(t)}) \end{array}\right)\nonumber\\
  &= 
\left(\begin{array}{c} \alpha y\\ \beta y\end{array}\right) \wedge 
  \left[\left(\begin{array}{c}s \\ \beta {h(s)}\end{array}\right) \vee \left(\begin{array}{c}t \\ \beta {h(t)} \end{array}\right)\right], \nonumber
\end{align}
and each term in the last expression belongs to $\langle G \rangle$, so $(r, \beta y) \in \langle G \rangle$, as desired.

Similarly, $(r, \alpha y) \in \langle G \rangle$.  Indeed, $\beta y \leqslant r$ implies $r = \beta y \vee s\vee t$, and $\beta {h(s)} \vee \beta {h(t)} \leqslant \alpha y$ implies $\alpha y = \alpha y \vee \beta {h(s)} \vee \beta {h(t)}$. Therefore,
\[\left(\begin{array}{c} r \\ \alpha y\end{array}\right) = 
\left(\begin{array}{c} \beta y \vee s\vee t \\ \alpha y \vee \beta {h(s)} \vee \beta {h(t)} \end{array}\right) = 
\left(\begin{array}{c} \beta y\\ \alpha y\end{array}\right) \vee 
\left(\begin{array}{c}s \\ \beta {h(s)}\end{array}\right) \vee \left(\begin{array}{c}t \\ \beta {h(t)} \end{array}\right).\]

{\it Case 2.} Suppose $r = s \wedge t$ and assume 
$(s, \alpha {h(s)})$, $(s, \beta{h(s)})$, $(t, \alpha {h(t)})$, and $(t, \beta{h(t)})$ belong to $\langle G \rangle$. Then $h(s\wedge t) = h(r) = y$, so 
$h(\alpha {h(s)} \wedge\alpha {h(t)}) = y = h(\beta {h(s)} \wedge\beta {h(t)})$, so
\[
\beta y \leqslant \beta h(s) \wedge \beta h(t) \leqslant \alpha {h(s)} \wedge \alpha {h(t)} \leqslant \alpha y.
\]
Also, $\beta y \leqslant r \leqslant \alpha y$ so $r = \alpha y \wedge s\wedge t$
and $r = \beta y \vee (s\wedge t)$. Taken together, these 
observations yield
\[
\left(\begin{array}{c} r \\ \alpha a\end{array}\right) = 
\left(\begin{array}{c} \beta y \vee (s\wedge t) \\ \alpha y \vee (\alpha {h(s)} \wedge \alpha {h(t)}) \end{array}\right) = 
\left(\begin{array}{c} \beta y\\ \alpha y\end{array}\right) \vee
\left[\left(\begin{array}{c}s \\ \alpha {h(s)}\end{array}\right) \wedge \left(\begin{array}{c}t \\ \alpha {h(t)} \end{array}\right)\right],\]
and each term in the last expression belongs to $\langle Y \rangle$.

Note, we could have used $\beta$'s instead:
\[\left(\begin{array}{c} r \\ \alpha y\end{array}\right) = 
\left(\begin{array}{c} \beta y \vee (s\wedge t) \\ \alpha y \vee (\beta {h(s)} \wedge \beta {h(t)}) \end{array}\right) = 
\left(\begin{array}{c} \beta y\\ \alpha y\end{array}\right) \vee
\left[\left(\begin{array}{c}s \\ \beta {h(s)}\end{array}\right) \wedge \left(\begin{array}{c}t \\ \beta {h(t)} \end{array}\right)\right].\]

Similarly,
\[\left(\begin{array}{c} r \\ \beta y\end{array}\right) = 
\left(\begin{array}{c} \alpha y \wedge s\wedge t \\ \beta y \wedge \alpha {h(s)} \wedge \alpha {h(t)} \end{array}\right) = 
\left(\begin{array}{c} \alpha y\\ \beta y\end{array}\right) \wedge 
\left(\begin{array}{c}s \\ \alpha {h(s)}\end{array}\right) \wedge \left(\begin{array}{c}t \\ \alpha {h(t)} \end{array}\right).\]

Again, we could have used $\beta$'s instead:
\[\left(\begin{array}{c} r \\ \beta y\end{array}\right) = 
\left(\begin{array}{c} \alpha y \wedge s\wedge t \\ \beta y \wedge \beta {h(s)} \wedge \beta {h(t)} \end{array}\right) = 
\left(\begin{array}{c} \alpha y\\ \beta y\end{array}\right) \wedge 
\left(\begin{array}{c}s \\ \beta {h(s)}\end{array}\right) \wedge \left(\begin{array}{c}t \\ \beta {h(t)} \end{array}\right).\]

In each case, we end up with an expression involving terms from $\langle G \rangle$, and this proves that $(r, \alpha y)$ and $(r, \beta y)$ belong to $\langle G \rangle$.

\medskip

\noindent ($\Leftarrow$) Suppose $h$ is not lower bounded. Then there exists an element $y_0\in L$ such that $\beta_0(y_0) > \beta_1(y_0) > \cdots$ is an infinite descending chain. This is a consequence of the definitions and remarks excerpted from Freese, Jezek, Nation~\cite{MR1319815} and paraphrased in the appendix below. (Specifically, see Section~\ref{sec:bound-homom} on \emph{bounded homomorphisms}.)

Let $K$ be a finite subset of $\ker h$, say, $K = \{(p_1, q_1), \dots, (p_{m}, q_{m})\} \subseteq \ker h$.
We prove $\langle K \rangle \neq \ker h$. (Since $K$ is an arbitrary finite subset of $\ker h$, this will prove $\ker h$ is not finitely generated.)

Let $x_0\in X$ be a generator of $\mathbf F$ that belongs to the class $h^{-1}\{y_0\}$ (so, $h(x_0) = y_0$).

\medskip

\noindent {\it Claim 1.1.} There exists $N<\omega$ such that for all $(p_i, q_i)$ in $K$, if $p_i \geqslant x_0$, then $q_i \geqslant \beta_N (y_0)$.

\noindent {\it Proof.}
Fix $i$ and $(p_i, q_i) \in K$ (so, $h(p_i) = h(q_i)$). Define $N_i$ as follows:
\begin{itemize}
\item[{\it Case 0.}] If $p_i \ngeqslant x_0$, let $N_i = 0$.  
\item[{\it Case 1.}] If $p_i\geqslant x_0$, then $x_0 = x_0\wedge p_i$, so $y_0 = h(x_0) = h(x_0) \wedge h(p_i) \leqslant h(p_i)$, so $y_0\leqslant h(q_i)$. Also, $h(x_0 \wedge q_i) = h(x_0) \wedge h(q_i) = y_0$, so $x_0\wedge q_i \in h^{-1}\{y_0\}$. Therefore (since $\{\beta_i(y_0)\}$ is an infinite descending chain in $h^{-1}\{y_0\}$) there exists $n_i>0$ such that $x_0 \wedge q_i \geqslant\beta_{n}(y_0)$. Let $N_i = n_i$ in this case (so $q_i \geqslant \beta_{N_i}(y_0)$).

\noindent Since $K$ is finite, we can find such $N_i$ for each $(p_i,q_i) \in K$. Let $N = \max\{N_i : 1 \leqslant i \leqslant m\}$.
Then for all $1\leqslant i \leqslant m$ the following implication holds:
\begin{equation}
\label{eq:star3}    
p_i \geqslant x_0 \quad \Longrightarrow \quad q_i \geqslant \beta_N(y_0).
\end{equation}
\end{itemize}

\medskip

\noindent {\it Claim 1.2.} There exists $N < \omega$ such that, for all $(p, q) \in \langle K \rangle$,
\begin{equation}
  \label{eq:claim2}
p \geqslant x_0 \quad \Longrightarrow \quad q \geqslant \beta_N(y_0).
\end{equation}

\noindent {\it Proof.}  Choose $N$ as described in the proof of Claim 1.1 above so that 
for all $(p_i,q_i) \in K$ the implication~(\ref{eq:star3}) holds. Fix $(p, q) \in \langle K \rangle$. We prove~(\ref{eq:claim2}) by induction on the complexity of $(p, q)$.  If $(p, q) \in K$, then there's nothing to prove.
\begin{itemize}
\item[{\it Case 1.}] Assume $(p, q) = (p_1, q_1) \wedge (p_2, q_2)$, where $p_i$, $q_i$ ($i = 1, 2$) satisfy~(\ref{eq:claim2}).  Assume $p\geqslant x_0$. %% (We show $q \geqslant \beta_N(y_0)$.)
  Then $p = p_1 \wedge p_2 \geqslant x_0$, so $p_1 \geqslant x_0$ and $p_2 \geqslant x_0$, so (by the induction hypothesis) $q_1\geqslant \beta_N(y_0)$ and $q_2\geqslant \beta_N(y_0).$ Therefore, $q = q_1 \wedge q_2 \geqslant \beta_N(y_0),$ as desired.

\item[{\it Case 2.}] Assume $(p, q) = (p_1, q_1) \vee (p_2, q_2)$, where $p_i$, $q_i$ ($i = 1, 2$) satisfy~(\ref{eq:claim2}). Assume $p\geqslant x_0$. Then $p = p_1 \vee p_2 \geqslant x$.  Since $x_0$ is a generator, it is join prime in $\mathbf{F}(X)$, so either $p_1 \geqslant x_0$ or $p_2 \geqslant x_0$.  Assume (wlog) $p_1 \geqslant x_0$. Then, (by induction hypothesis) $q_1\geqslant \beta_N(y_0).$
Therefore, $q = q_1 \vee q_2 \geqslant q_1 \geqslant \beta_N(y_0),$ as desired.
\end{itemize}

\medskip

\noindent {\it Claim 1.3.}  $K$ does not generate $\ker h$.

\noindent {\it Proof.} Let $N$ be chosen as in the proof of Claim 1.2 above.  Since $\beta_0(y_0) > \beta_1(y_0) > \cdots$ is an infinite descending chain, $\beta_{N}(y_0) > \beta_{N+1}(y_0)$. The pair $(p, q) = (x_0, \beta_{N+1}(y_0))$ does not belong to $\langle K\rangle$, however it does belong to the kernel of $h$.  This proves that the finite subset $K$ does not generate $\ker h$.  Since $K$ was an arbitrary finite subset of $\ker h$, we have proved that $\ker h$ is not finitely generated.

\end{proof}

\section{Examples}

Let $\mathbf{M_3} = \langle \{0, a, b, c, 1\}, \wedge, \vee\rangle$, where $a \wedge b = a \wedge c = b \wedge c = 0$ and $a \vee b = a \vee c = b \vee c = 1.$ Let $\mathbf F := \mathbf F(x, y, z)$ denote the free lattice generated by $\{x, y, z\}$.

\begin{theorem}
Let $h\colon \mathbf{F} \twoheadrightarrow \mathbf{M_3}$ be the epimorphism that acts on the generators as follows: $x\mapsto a$, $y\mapsto b$, $z\mapsto c.$ Then $\operatorname{ker} h$ is not finitely generated.
%% If $K = \operatorname{ker}\left(\mathbf{F}\{x, y, z\} \twoheadrightarrow \mathbf{M_3}\right)$, then $K$ is not finitely generated.
\end{theorem}
\begin{proof}
Let $K := \operatorname{ker} h$, and for $u \in \{x, y, z\}$ let $C_u := u/K := \{v \in F : h(v) = h(u)\}$.
Define sequences of elements in these classes by the following mutual recursions:
\begin{itemize}
\item for $i \in \mathbb N$,  
  \[m_{0, i} = (m_{x,i} \wedge m_{y,i}) \vee (m_{x,i} \wedge m_{z,i})\vee (m_{y,i} \wedge m_{z,i});\]

\item for $u \in \{x, y, z\}$,   
  \begin{align*}
    m_{u,0} &= u,\\  
    m_{u, i+1} &= m_{u, i}\vee m_{0,i}.
  \end{align*}
\end{itemize}

\noindent Notice that
$m_{0, 0} = (x\wedge y) \vee (x\wedge z)\vee (y\wedge z)$ and $m_{x, i+1} = m_{x,i} \vee (m_{y,i} \wedge m_{z,i})$.  

Let $X$ be a finite subset of $K$.  We will prove there exists $(p,q) \in K \setminus \langle X \rangle$.  
Fix $u\in \{0, x, y, z\}$. Since $X$ is finite, Lemma~\ref{lem:2} implies that there exists $M \in \mathbb{N}$ such that for every $(p, q) \in X$ with $p, q \in C_u$, we have $p, q \leqslant m_{u, M}$.\\[6pt]
{\bf Claim 2.1.} For $(p,q) \in \langle X \rangle$ and $u \in \{x, y, z\}$, the following implication holds:
\begin{equation}
  \label{eq:star}
q \leqslant u \quad \Longrightarrow \quad p\leqslant m_{u, M}.
\end{equation}

We prove the claim by induction on the complexity of terms.
Fix $(p,q) \in \langle X \rangle$. Then $p, q \in C_u$ for some $u\in \{x, y, z\}$.
\begin{itemize}
\item[{\it Case 0.}] Suppose $(p, q) \in X$. Then by definition of $M$ we have $p, q \leqslant m_{u, M}$.

\item[{\it Case 1.}] Suppose $(p, q) = (p_1, q_1) \wedge (p_2, q_2)$, where 
  $(p_i, q_i)$ satisfies~(\ref{eq:star}) for $i = 1, 2$.    
  If $q= q_1 \wedge q_2 \leqslant u$, then, since generators in the free lattice are meet-prime (see Theorem 1.11 below), we have $q_1\leqslant u$ or $q_2\leqslant u$. Assume $q_1\leqslant u$.  Then, by the induction hypothesis, $p_1\leqslant m_{u, M}$.  Therefore, $p = p_1\wedge p_2 \leqslant m_{u, M}$, as desired.

\item[{\it Case 2.}] Suppose $(p, q) = (p_1, q_1) \vee (p_2, q_2)$, where 
  $(p_i, q_i)$ satisfies~(\ref{eq:star}) for $i = 1, 2$.    
  If $q= q_1 \vee q_2 \leqslant u$, then 
  $q_i \leqslant u$ for $i = 1, 2$.
  It now follows from the induction hypothesis that $p_i\leqslant m_{u, M}$ for $i = 1, 2$, so 
  $p = p_1 \vee p_2 \leqslant m_{u, M}$, as desired.
\end{itemize}

From Claim 2.1, and Lemma~\ref{lem:1}, it follows that $(m_{x, M+1}, x)\in K \setminus \langle X \rangle$, so the proof of the theorem is complete.
\end{proof}

\begin{lemma}%##### Lemma 1
  \label{lem:1}
For each $u \in \{0, x, y, z\}$, the sequence $\{m_{u,n} : n \in \mathbb N\}$ is a strictly ascending chain; that is, $m_{u,0} < m_{u,1} < m_{u,2} < \cdots$.  
\end{lemma}

\begin{proof}\
  
\noindent {\it Case.} $u \in \{x, y, z\}$.
    
  For simplicity, assume $u = x$ for the remainder of the proof of this case.  (Of course, the same argument goes through when $u$ is $y$ or $z$.) Fix $n \in \mathbb N$.  We prove $m_{x,n} < m_{x,n+1}$.

\noindent {\it Subclaim.} For all $n\in \mathbb N$,
  \begin{enumerate}
  \item 
  $m_{x,n}\in C_x$, 
  \item $m_{x,n}\ngeq y$, and $m_{x,n}\ngeq z$.
  \end{enumerate}
  The first subclaim is obvious. For the second, if $m_{x,n}\geqslant y$, then $m_{x,n}\wedge y = y$, and then $0 = h(m_{x,n}\wedge y) = h(y) = b$.  A similar contradiction is reach if we assume $m_{x,n}\geqslant z$, so the subclaim is proved.

  Recall, $m_{x, n} = m_{x,n} \vee (m_{y,n} \wedge m_{z,n})$, so our desired conclusion, $m_{x,n} < m_{x,n+1}$, holds unless 
  $m_{x,n} \geqslant m_{y,n} \wedge m_{z,n}$.
  So, by way of contradiction, suppose 
  \begin{equation}\label{eq:doubledaggar}
  m_{x,n} \geqslant m_{y,n} \wedge m_{z,n}.
  \end{equation}

  Now, $m_{y,n} = y \vee (x \wedge z) \vee \cdots$, so clearly $m_{y,n} \geqslant y$.  Similarly, $m_{z,n} \geqslant z$.
  This, together with~(\ref{eq:doubledaggar}), implies $m_{x,n} \geqslant m_{y,n} \wedge m_{z,n} \geqslant y \wedge z$.  But then Theorem 1.11 below implies that either $m_{x,n} \geqslant y$ or $m_{x,n} \geqslant z$, which contradicts Subclaim 2 above.

  \medskip

\noindent {\it Case.} $u = 0$.
  
  We first prove that $m_{0, 0} = (x\wedge y) \vee (x\wedge z)\vee (y\wedge z)$ is strictly below $m_{0, 1} = 
  (m_{x,1}\wedge m_{y,1}) \vee (m_{x,1}\wedge m_{z,1}) \vee (m_{y,1}\wedge m_{z,1})$.

  By symmetry, it suffices to show $x\wedge y < m_{x,1}\wedge m_{y,1}$;
  that is, $x\wedge y < (x \vee (y\wedge z))\wedge (y \vee (x\wedge z))$.

  Clearly 
  $x\wedge y \leqslant (x \vee (y\wedge z))\wedge (y \vee (x\wedge z))$. Suppose $x\wedge y = (x \vee (y\wedge z))\wedge (y \vee (x\wedge z))$. Then $(x \vee (y\wedge z))\wedge (y \vee (x\wedge z))\leqslant x$.
  By Theorem 1.11, the latter holds iff
  $x \vee (y\wedge z)\leqslant x$ or
  $y \vee (x\wedge z)\leqslant x$
  The first of these inequalities is clearly false, so it must be the case that $y \vee (x\wedge z)\leqslant x$.  But then $y \leqslant x$, which is obviously false.  We conclude that
  $x\wedge y <(x \vee (y\wedge z))\wedge (y \vee (x\wedge z))$.
  This proves $m_{0,0} < m_{0,1}$.

  Now fix $n\in \mathbb{N}$ and assume $m_{0,n} < m_{0,n+1}$.
  We show $m_{0,n+1} < m_{0,n+2}$.

  ({\bf To do:} complete the proof in this case; i.e., for $u = 0$.)

  --scratch work--

  $m_{0, n} := (m_{x,n} \wedge m_{y,n}) \vee (m_{x,n} \wedge m_{z,n})\vee (m_{y,n} \wedge m_{z,n})$,

  $m_{0, n+1} := (m_{x,n+1} \wedge m_{y,n+1}) \vee (m_{x,n+1} \wedge m_{z,n+1})\vee (m_{y,n+1} \wedge m_{z,n+1})$,

  By the first Case above, $m_{u,n} < m_{u,n+1}$.
\end{proof}


\begin{lemma}\label{lem:2} %##### Lemma 2 
For all $u \in \{x, y, z\}$ and $p \in C_u \cup C_0$ there exists $n \in \mathbb N$ such that $p\leqslant m_{u,n}$.  
\end{lemma}

\begin{proof}(By induction on the complexity of $p$.) 

\noindent {\it Case 0.} $p\in \{x, y, z\}$. Then $u = p = m_{p,0}$.

For the remaining cases assume $u = x$, without loss of generality.

\noindent {\it Case 1.} $p = p_1 \vee p_2$.  
  If $p \in C_x\cup C_0$, then $p_i \in C_x \cup C_0$ for $i = 1, 2$, and 
  the induction hypothesis yields $i$ and $j$ for which $p_1 \leqslant m_{x, i}$ and $p_2 \leqslant m_{x, j}$. Letting $n = \max\{i, j\}$, we have 
  $p_1, p_2 \leqslant m_{x, n}$, from which 
  $p = p_1 \vee p_2 \leqslant m_{x, n}$, as desired.

\noindent {\it Case 2.} $p = p_1 \wedge p_2$.  
  If $p \in C_x$, then we may assume $p_1 \in C_x$ and $p_2 \in C_x \cup C_0$. By the induction hypothesis, there exists $n\in \mathbb N$ such that $p_1 \leqslant m_{x, n}$, whence $p \leqslant p_1 \leqslant m_{x,n}$.
If $p \in C_0$, then each $p_i$ belongs to $C_u \cup C_0$ for some $u\in \{x, y, z\}$.  If $p_1 \in C_x \cup C_0$, then $p_1 \leqslant m_{x, n}$, as above and we're done.  Similarly, if $p_2 \in C_x \cup C_0$.  So assume $p_1 \in C_y \cup C_0$ and $p_2 \in C_z \cup C_0$. Then the induction hypothesis implies that there exist $i$ and $j$ such that 
  $p_1 \leqslant m_{y, i}$ and $p_2 \leqslant m_{z, j}$. If $n = \max\{i, j\}$, then
  $p_1 \leqslant m_{y, n}$ and $p_2 \leqslant m_{z, n}$.  Then, by the above definition of the sequences, we have
  $p_1 \wedge p_2 \leqslant m_{y, n} \wedge m_{z, n} \leqslant m_{0,n} \leqslant m_{x,n+1}$.
\end{proof}

\subsection{Other Examples}
In each of the examples below, $X$ is a finite set and 
$\mathbf{F} = \mathbf{F}(X)$ is the free lattice generated by $X$.
The symbol $F$ denotes the universe of $\mathbf{F}$. 
\begin{itemize}
\item[{\bf Ex 1.}] Let $X = \{x,y,z\}$, and let $\mathbf{L} = \mathbf{2}$ be the 2-element chain.    
Then the kernel of an epimorphism $h\colon \mathbf{F} \twoheadrightarrow \mathbf{L}$ is a finitely generated sublattice of $\mathbf{F} \times \mathbf{F}$.
\item[{\bf Ex 2.}] Let $X = \{x, y, z\}$ and let $\mathbf{L} = \mathbf{3}$ be the 3-element chain.    
Then the kernel of an epimorphism $h\colon \mathbf{F} \twoheadrightarrow \mathbf{L}$ is finitely generated.
\item[{\bf Ex 3.}] Let $n > 2$, $X = \{x_0, x_1,\dots, x_{n-1}\}$, and $\mathbf{L} = \mathbf{2} \times \mathbf{2}$.  
Let $h\colon \mathbf{F} \twoheadrightarrow \mathbf{L}$ be an epimorphism. Then $K = \operatorname{ker}h$ is finitely generated.  
\item[{\bf Ex 4.}] Let $\mathbf{F} = \mathbf{F}(x,y,z)$, and let $\mathbf{L} = \mathbf{F}_{\mathbf{M}_3}(3)$ (see Figure~\ref{fig:1}).  
Let $h\colon \mathbf{F} \twoheadrightarrow \mathbf{L}$ be an epimorphism. Then $K = \operatorname{ker}h$ is not finitely generated.  
\end{itemize}

\newcommand{\dotsize}{0.8pt}
\tikzstyle{lat} = [circle,draw,inner sep=\dotsize]
\newcommand{\figscale}{1}
\begin{figure}
\begin{tikzpicture}[scale=\figscale]
  \foreach \j in {0,...,8} {
    \node[lat] (0\j) at (0,\j) {};
  }
  \foreach \j in {1,2,4,6,7} {
    \node[lat] (n1\j) at (-1,\j) {};
    \node[lat] (1\j) at (1,\j) {};
  }
  \foreach \j in {3,5} {
    \node[lat] (n2\j) at (-2,\j) {};
    \node[lat] (2\j) at (2,\j) {};
  }
  \node[lat] (n34) at (-3,4) {};
  \node[lat] (34) at (3,4) {};

  \draw[semithick] (00) -- (01);
  \draw[semithick] (02) -- (03) -- (04) -- (05) -- (06);
  \draw[semithick] (07) -- (08);
  \draw[semithick] (n16) -- (n17);
  \draw[semithick] (16) -- (17);
  \draw[semithick] (n11) -- (n12);
  \draw[semithick] (11) -- (12);

  \draw[semithick] (00) -- (n11) -- (02) -- (11) -- (00);
  \draw[semithick] (06) -- (n17) -- (08) -- (17) -- (06);

  \draw[semithick] (01) -- (n12) -- (n23) -- (n34);
  \draw[semithick] (01) -- (12) -- (23) -- (34);
  \draw[semithick] (12) -- (03) -- (n14) -- (n25);
  \draw[semithick] (n12) -- (03) -- (14) -- (25);
  \draw[semithick] (23) -- (14) -- (05) -- (n16);
  \draw[semithick] (n23) -- (n14) -- (05) -- (16);
  \draw[semithick] (34) -- (25) -- (16) -- (07);
  \draw[semithick] (n34) -- (n25) -- (n16) -- (07);

  \node[lat] (n13) at (-1,3) {};
  \node[lat] (n15) at (-1,5) {};
  \node[lat] (n24) at (-2,4) {};
  \draw[thick,red] (02) -- (n13) -- (n24) -- (n15) -- (06);
  \draw[thick,red] (n13) -- (04) -- (n15);

  \node (x) at (-3,4) [circle,fill,inner sep=\dotsize]{};
  \node (y) at (3,4) [circle,fill,inner sep=\dotsize]{};
  \node (z) at (-2,4) [circle,fill,inner sep=\dotsize]{};
  \draw (x) node[left] {$x$};
  \draw (y) node[right] {$y$};
  \draw (z) node[left] {$z$};
  \end{tikzpicture}
  \caption{The free lattice over $M_3$.}
  \label{fig:1}
\end{figure}

\bigskip
\hrule
\bigskip

\noindent {\bf Proof of the claim in Ex 3.} Let the universe of $\mathbf{L} = \mathbf{2} \times \mathbf{2}$ be $\{0, a, b, 1\}$, where $a\vee b = 1$ and $a\wedge b = 0$.  
For each $y \in L$, denote by $X_y = X \cap h^{-1}\{y\}$ the set of generators mapped by $h$ to $y$. 
Denote the least and greatest elements of $h^{-1}\{y\}$ (if they exist) by $\ell_y$ and $g_y$, respectively.  For example, 
$\ell_a = \bigwedge h^{-1}\{a\} = \bigwedge \{x\in F: h(x) = a\},\;$ $\quad g_b = \bigvee \{x\in F : h(x) = b\},\quad$ etc. 
In the present example, the least and greatest elements exist is each case, as we now show.

\medskip

\noindent {\it Claim 3.1.} $h^{-1}\{a\}$ has least and greatest elements, namely $\ell_a = \bigwedge (X_a \cup X_1)$ and $g_a = \bigvee (X_a\cup X_0)$.  (Similarly, 
$h^{-1}\{b\}$ has least and greatest elements, $\ell_b$ and $g_b$.)

\bigskip


\noindent {\it Claim 3.2.} $h^{-1}\{0\}$ has least and greatest elements, 
namely, $\ell_0 = \bigwedge X$ and $g_0 = g_a \wedge g_b$.

\bigskip

\noindent {\it Claim 3.3.} $h^{-1}\{1\}$ has least and greatest elements, namely $\ell_1 = \ell_a \vee \ell_b$ and $g_1 = \bigvee X$.


\noindent {\it Proof of Claim 3.1.}
Let $M(a):=\bigwedge (X_a\cup X_1)$ and $J(a):=\bigvee (X_a\cup X_0)$ and 
note that these values exist in $F$, since the sets involved are finite. Also, 
Then $h(M(a)) = a = h(J(a))$. Fix $r \in h^{-1}\{a\}$.  
\begin{itemize}
\item[{\it Case 0.}] $r \in X_a$.  
   Then $r\geqslant \bigwedge X_a \geqslant \bigwedge (X_a\cup X_1) = M(a)$.

\item[{\it Case 1.}] $r = s \vee t$, where $h(s) = a$ and $h(t) \in \{a, 0\}$.  
Assume (the induction hypothesis) that $s \geqslant M(a)$. Then 
$r = s\vee t \geqslant M(a)$.

\item[{\it Case 2.}] $r = s \wedge t$, where $h(s) = a$ and $h(t) \in \{a, 1\}$.  
Assume (the induction hypothesis) that $s, t \geqslant M(a)$. 
Then $s \wedge t \geqslant M(a)$.
This proves that for each $r \in h^{-1}\{a\}$ we have $r \geqslant M(a)$, and 
as we noted at the outset, $M(a)\in h^{-1}\{a\}$. Therefore, $\ell_a = M(a)$ is the least element of $h^{-1}\{a\}$.
Similarly, every $r \in h^{-1}\{a\}$ is below $J(a)$, so $g_a = J(a)$.  The proofs of $\ell_b = M(b)$ and $g_b = J(b)$ are similar.
\end{itemize}

\medskip

\noindent {\it Proof of Claim 3.2.}
$\ell_0 = \bigwedge X$ is obvious, so we need only verify that $g_0 = g_a \wedge g_b$. 
Observe that $h(g_a \wedge g_b) = h(g_a) \wedge h(g_b) = a \wedge b = 0$, so 
$g_a \wedge g_b \in h^{-1}\{0\}$. It remains to prove that $r \leqslant g_a \wedge g_b$ holds for all $r \in h^{-1}\{0\}$.
Fix $r \in h^{-1}\{0\}$. Then $h(r \vee g_a) = h(r) \vee h(g_a) = 0 \vee a = a$, which places $r \vee g_a$ in $h^{-1}\{a\}$.  Therefore, 
by maximality of $g_a$, we have $r \vee g_a  \leqslant g_a$, whence 
$r\leqslant g_a$.  Similarly, $r\leqslant g_b$.

\medskip

\noindent {\it Proof of Claim 3.3.}
$g_1 = \bigvee X$ is obvious, so we need only verify that $\ell_1 = \ell_a \vee \ell_b$. 
Observe that $h(\ell_a \vee \ell_b) = h(\ell_a) \vee h(\ell_b) = a \vee b = 1$, so 
$\ell_a \vee \ell_b \in h^{-1}\{1\}$. It remains to prove that $r \geqslant \ell_a \vee \ell_b$ holds for all $r \in h^{-1}\{1\}$.
Fix $r \in h^{-1}\{1\}$. Then $h(r \wedge \ell_a) = h(r) \wedge h(\ell_a) = 1 \wedge a = a$, which places $r \wedge \ell_a$ in $h^{-1}\{a\}$.  Therefore, 
by minimality of $\ell_a$, we have $r \wedge \ell_a  \geqslant \ell_a$, whence 
$r\geqslant \ell_a$.  Similarly, $r\geqslant \ell_b$.
Now let $Y = \{(x, g_p), (g_p, x), (x, \ell_p),(\ell_p, x) : p \in \{0, a, b, 1\}, x \in X_p\}$.

\medskip
\noindent {\it Claim 3.4.} If $r \in F$ and $h(r) = p$, then $(r, \ell_p), (r, g_p) \in \langle Y \rangle$.

\noindent {{\it Proof.}\\
  \noindent {\it Case 0.} If $r \in X_p$, then the pair belongs to $Y$ and the claim is trivial.

  \medskip

  \noindent {\it Case 1.} Suppose $r = s \wedge t$.

  \begin{itemize}
  \item If $h(r) = 1$, then $h(s) = h(t) = 1$.  If we assume (the induction hypothesis) that $(s, \ell_1), (s, g_1), (t, \ell_1), (t, g_1)$ belong to $\langle Y \rangle$, then $(r, \ell_1) = (s \wedge t, \ell_1) =  (s, \ell_1) \wedge (t, \ell_1) \in \langle Y \rangle$. 
  
\item If $h(r) = a$, then (wlog) $h(s) = a$ and $h(t) \in \{a, 1\}$.  Assume (the induction hypothesis) that $(s, \ell_a), (s, g_a), (t, \ell_p), (t, g_p)$ belong to $\langle Y \rangle$. By Claim 1, $\ell_a \leqslant \ell_1$, so $\ell_a = \ell_a \wedge \ell_1$.\\
  - If $h(t) = 1$, then $(r, \ell_a) = (s \wedge t, \ell_a \wedge \ell_1) =  (s, \ell_a) \wedge (t, \ell_1) \in \langle Y \rangle$.\\
  - If $h(t) = a$, then $(r, \ell_a) = (s \wedge t, \ell_a \wedge \ell_a) =  (s, \ell_a) \wedge (t, \ell_a) \in \langle Y \rangle$.
  
\item If $h(r) = 0$, then (wlog) that either (i) $h(s) = 0$, or (ii) $h(s) = a$, $h(t)=b$.  
  If $h(s) = 0$, then $(s, \ell_0) \in \langle Y \rangle$ implies 
  $(r, \ell_0) = (s \wedge t, \ell_0) =  (s, \ell_0) \wedge (t, \ell_p) \in \langle Y \rangle$. If 
  If $h(s) = a$, $h(t) = b$, and
  $(s, \ell_a), (t, \ell_b) \in \langle Y \rangle$, then
  $(r, \ell_0) = (s \wedge t, \ell_0) =  (s, \ell_a) \wedge (t, \ell_b) \in \langle Y \rangle$.
  
  \end{itemize}
Similarly, in each of these three subcases we have $(r, g_p) \in \langle Y \rangle$.

  \medskip
\noindent {\it Case 2.} Suppose $r = s \vee t$.
  \begin{itemize}
  \item If $h(r) = 0$, then $h(s) = h(t) = 0$.  If we assume (the induction hypothesis) that $(s, \ell_p), (s, g_p), (t, \ell_p), (t, g_p)$ belong to $\langle Y \rangle$, then $(r, \ell_p) = (s \vee t, \ell_p) =  (s, \ell_p) \vee (t, \ell_p) \in \langle Y \rangle$. 

  \item If $h(r) = a$, then (wlog) $h(s) = a$ and $h(t) \in \{a, 0\}$.  If we assume (the induction hypothesis) that $(s, \ell_p), (s, g_p), (t, \ell_p), (t, g_p)$ belong to $\langle Y \rangle$, then $(r, \ell_p) = (s \vee t, \ell_p) =  (s, \ell_p) \vee (t, \ell_p) \in \langle Y \rangle$. 

  \item If $h(r) = 1$, then (wlog) that either (i) $h(s) = 1$, or (ii) $h(s) = a$, $h(t)=b$.\\
  - If $h(s) = 1$, then $(s, \ell_1) \in \langle Y \rangle$ implies 
  $(r, \ell_1) = (s \vee t, \ell_1) =  (s, \ell_1) \vee (t, \ell_p) \in \langle Y \rangle$.\\
  - If $h(s) = a$, $h(t) = b$, and
  $(s, \ell_a), (t, \ell_b) \in \langle Y \rangle$, then
  $(r, \ell_1) = (s \vee t, \ell_1) =  (s, \ell_a) \vee (t, \ell_b) \in \langle Y \rangle$.

  \end{itemize}
Similarly, in each of these three subcases, we have $(r, g_p) \in \langle Y \rangle$.


\bigskip

\hrule


\bigskip


\noindent {\bf Proof of the claim in Ex 4.} 
Define the sequences $\{m_n\}$, $\{x_n\}$, $\{y_n\}$, $\{z_n\}$ ($n< \omega$) of elements of $\mathbf{F}(X)$ as follows: 

Let  $x_0 = x$, $y_0 = y$, $z_0 = z$, and for $n\geqslant 0$, 
\begin{align*}
  m_n &= (x_n \wedge y_n) \vee (x_n \wedge z_n) \vee (y_n \wedge z_n);\\
  % m_0 &= (m_{x0}\wedge m_{y0}) \vee (m_{x0} \wedge m_{z0}) \vee (m_{y0} \wedge m_{z0})= (x\wedge y) \vee (x \wedge z) \vee (y \wedge z);\\
  % m_{x1} &= x\vee m_0 = x \vee (y \wedge z), \; \text{ Define $m_{y1}$ and $m_{z1}$ similarly;}\\
  x_{n+1} &= x_n\vee m_n = x_n \vee (y_n \wedge z_n).
\end{align*}
Define $y_{n+1}$ and $z_{n+1}$ similarly.

\bigskip

\noindent \underline{Claim 4.1} If $\{s_n\}$ is any one of the four sequences just defined, then for $n>0$, we have $s_{n+1} > s_n$ and $h(s_{n+1}) = h(s_n)$.

\medskip

\wjd{fill in proof}
%\noindent {\bf To Do:} Prove Claim 4.1.

\bigskip

\noindent \underline{Claim 4.2} If $\{s_n\}$ is any one of the four sequences just defined, then for $n>0$, we have $s_{n+1} > s_n$ and $h(s_{n+1}) = h(s_n)$.

\medskip

\wjd{fill in proof}
%\noindent {\bf To Do:} Prove Claim 4.2.

\bigskip

\noindent For now, let's assume Claims 4.1 and 4.2 are true.

\newpage

\noindent Let $K$ be a finite subset of $\ker h$.  Since $K$ is finite, we can find an $N < \omega$ such that for all $\binom{p}{q} \in K$, the following implications are satisfied:
\begin{align}
p &\leqslant x \quad \Longrightarrow \quad q \leqslant x_N\nonumber\\
p &\leqslant y \quad \Longrightarrow \quad q \leqslant y_N\label{eq:imp1}\\
p &\leqslant z \quad \Longrightarrow \quad q \leqslant z_N\nonumber
\end{align}
\begin{align}
p &\leqslant x \vee (y \wedge z) \quad \Longrightarrow \quad q \leqslant x_{N+1}\nonumber\\
p &\leqslant y \vee (x \wedge z) \quad \Longrightarrow \quad q \leqslant y_{N+1}\label{eq:imp2}\\
p &\leqslant z \vee (x \wedge y) \quad \Longrightarrow \quad q \leqslant z_{N+1}\nonumber
\end{align}

\medskip

\noindent {\bf Claim 4.3} If $N$ is chosen as just described, and if $\binom{p}{q} \in \langle K \rangle$ then the implications~\ref{eq:imp1} and~\ref{eq:imp2} hold.

\begin{proof}
  As usual, we proceed by induction on term complexity.
If $\binom{p}{q} \in K$, then by choice of $N$, there is nothing to prove.

\medskip
\noindent {\it Case 1.} Suppose $\binom{p}{q} = \binom{p_1}{q_1} \vee \binom{p_2}{q_2}$, where $\binom{p_1}{q_1}$ and $\binom{p_2}{q_2}$ satisfy~(\ref{eq:imp1}) and~(\ref{eq:imp2}). We show that $\binom{p}{q}$ satisfies these two implications as well.
Recall, in the notation above, $x_1:=x \vee (y \wedge z)$.


Assume $p\leqslant x_1$. We show $q \leq x_{N+1}$.
Since $p = p_1 \vee p_2 \leq x_1$, we have 
$p_1 \leqslant x_1$ and 
$p_2 \leqslant x_1$, so by the induction hypothesis,  
$q_1 \leqslant x_{N+1}$ and 
$q_2 \leqslant x_{N+1}$.  Therefore, $q = q_1 \vee q_2 \leq x_{N+1}$, as desired.

Now assume $p\leqslant x$. We show $q\leqslant x_N$. 
Since $p = p_1 \vee p_2 \leq x$, we have 
$p_1 \leqslant x$ and 
$p_2 \leqslant x$, so by the induction hypothesis,  
$q_1 \leqslant x_{N}$ and 
$q_2 \leqslant x_{N}$.  Therefore, $q = q_1 \vee q_2 \leq x_{N}$, as desired.


\medskip
\noindent {\it Case 2.} Suppose $\binom{p}{q} = \binom{p_1}{q_1} \wedge \binom{p_2}{q_2}$, where $\binom{p_1}{q_1}$ and $\binom{p_2}{q_2}$ satisfy~(\ref{eq:imp1}) and~(\ref{eq:imp2}). 

Assume $p\leqslant x_1 = x \vee (y \wedge z)$. 
We must show $q\leqslant x_{N+1}$. 
Since $p_1 \wedge p_2 \leqslant x_1$, then according to Theorem~\ref{thm:wordprob}, at least one of the following inequalities must hold:
\begin{enumerate}
  \item $p_1 \leqslant x_1$;
  \item $p_2 \leqslant x_1$;
  \item $p_1 \wedge p_2 \leqslant x$;
  \item $p_1 \wedge p_2 \leqslant y \wedge z$.
\end{enumerate}
By the induction hypothesis, (1) implies $q_1 \leq x_{N+1}$  and (2) implies $q_2 \leq x_{N+1}$.  In either case, $q = q_1 \wedge q_2 \leq x_{N+1}$, as desired.  In case (3), Theorem~\ref{thm:wordprob} implies that either $p_1 \leq x$ or $p_2 \leq x$, since $x$ is a generator. Therefore, $q_1 \leq x_N$ or $q_2 \leq x_N$ and we conclude that 
$q \leq x_N \leq x_{N+1}$, as desired.  It remains to prove $q \leq x_{N+1}$ for the final case in which $p_1 \wedge p_2 \leq y \wedge z$. 

If $p_1 \wedge p_2 \leq y \wedge z$, then 
$p_1 \wedge p_2 \leq y$ and $p_1 \wedge p_2 \leq z$. Therefore, both of the following disjunctions hold:
\begin{itemize}
  \item $p_1 \leq y$ or $p_2 \leq y$, and 
  \item $p_1 \leq z$ or $p_2 \leq z$. 
\end{itemize}
If $p_1 \leq y$ and $p_1 \leq z$, then $p_1 \leq x \vee (y\wedge z) = x_1$, so $q_1 \leq x_{N+1}$, so $q = q_1\wedge q_2 \leq x_{N+1}$, as desired.
Similarly, if $p_2 \leq y$ and $p_2 \leq z$, the desired conclusion holds.
Finally, consider the case in which $p_1 \leq y$ and $p_2 \leq z$. In this case $q_1 \leq y_N$ and $q_2 \leq z_N$.  Therefore, $q = q_1 \wedge q_2 \leq y_N \wedge z_N \leq x_N \vee (y_N\wedge z_N) = x_{N+1}$, as desired.


\end{proof}

\bigskip


\bigskip
\hrule
\bigskip

  \appendix

\section{Background}
Here are some useful definitions and results from the Free Lattices book by Freese, Jezek, and Nation~\cite{MR1319815}.

\begin{definition}[length of a term] Let $X$ be a set. Each element of $X$ is a term of length 1, also known as a \emph{variable}. If $t_1, \dots, t_n$ are terms of 
lengths $k_1, \dots, k_n$, then $t_1 \vee \cdots \vee t_n$ and 
$t_1 \wedge \cdots \wedge t_n$ are both terms of length $1+ k_1 + \cdots + k_n$.
\end{definition}

{\bf Examples.} By the above definition, the terms 

$x \vee y \vee z \qquad x \vee (y \vee z) \qquad (x \vee y) \vee z$

have lengths 4, 5, and 5, respectively. Reason: variables have length 1, so $x \vee y \vee z$ has length $1 + 1 + 1 + 1$.  On the other hand, $x\vee y$ is a term of length $3$, so $(x \vee y) \vee z$ has length $1 + 3 + 1$. Similarly, 
$x \vee (y \vee z)$ has length $1 + 1 + 3$.

\begin{lemma}[\protect{\cite[Lem.~1.2]{MR1319815}}]
Let $\mathcal{V}$ be a nontrivial variety of lattices and let $\mathbf{F}_{\mathcal V}(X)$ be the relatively free lattice in $\mathcal V$ over $X$.  Then,
\begin{equation}
  \label{eq:star2}
\bigwedge S \leqslant \bigvee T \text{ implies }S \cap T \neq \emptyset
\text{ for each pair of finite subsets $S, T \subseteq X$.}
\end{equation}
\end{lemma}

\begin{lemma}[\protect{\cite[Lem.~1.4]{MR1319815}}]
Let $\mathbf L$ be a lattice generated by a set $X$ and let $a \in L$.  Then 
\begin{enumerate}
\item 
if $a$ is join prime, then $a = \bigwedge S$ for some finite subset $S \subseteq X$.
\item if $a$ is meet prime, then $a = \bigvee S$ for some finite subset $S \subseteq X$.

If $X$ satisfies condition~(\ref{eq:star2}) above, then 
\item for every finite, nonempty subset $S \subset X$, $\bigwedge S$ is join prime and $\bigvee S$ is meet prime.
\end{enumerate}
\end{lemma}

\begin{corollary}[\protect{\cite[Cor.~1.5]{MR1319815}}]
Let $\mathcal V$ be a nontrivial variety of lattices and let $\mathbf{F}_{\mathcal V}(X)$ be the relatively free lattice in $\mathcal V$ over $X$.  For each finite nonempty subset $S \subseteq X$, $\bigwedge S$ is join prime and $\bigvee S$ is meet prime. In particular, every $x\in X$ is both join and meet prime.  Moreover, if $x\leqslant y$ for $x, y \in X$, then $x = y$.
\end{corollary}

\begin{theorem}[Whitman's Condition, ver. 1]  
The free lattice $\mathbf{F}(X)$ satisfies the following condition:

(W)  If $v = v_1 \wedge \cdots \wedge v_r \leqslant u_1 \vee \cdots \vee u_s = u$, 
then either $v_i \leqslant u$ for some $i$, or $v \leqslant u_j$ for some $j$. 
\end{theorem}

\begin{corollary}[\protect{\cite[Cor.~1.9]{MR1319815}}]
Every sublattice of a free lattice satisfies (W). Every element of a lattice satisfying (W) is either join or meet irreducible.
\end{corollary}

\begin{theorem}[Whitman's Condition, ver. 2]  
The free lattice $\mathbf{F}(X)$ satisfies the following condition:

(W+)  If $v = v_1 \wedge \cdots \wedge v_r \wedge x_1 \wedge \cdots
\wedge x_n \leqslant u_1 \vee \cdots \vee u_s \vee
y_1 \vee \cdots \vee y_m = u$, where $x_i, y_j\in X$, then either 
$x_i = y_j$ for some $i$ and $j$, or $v_i \leqslant u$ for some $i$, or
$v \leqslant u_j$ for some $j$. 
\end{theorem}

\begin{theorem}[\protect{\cite[Thm.~1.11]{MR1319815}}]
  \label{thm:wordprob}
If $s = s(x_1, \dots, x_n)$ and $t = t(x_1, \dots, x_n)$ are terms and $x_1, \dots, x_n \in X$, then the truth of 
\begin{equation}
  \label{eq:ast}
s^{\mathbf{F}(X)} \leqslant t^{\mathbf{F}(X)}
\end{equation}
can be determined by applying the following rules.
\begin{enumerate}
\item If $s=x_i$ and $t=x_j$, then (\ref{eq:ast}) holds iff $x_i = x_j$.
\item If $s = s_1 \vee \dots \vee s_k$ is a formal join, then (\ref{eq:ast}) holds iff $s_i^{\mathbf{F}(X)} \leqslant t^{\mathbf{F}(X)}$ for all $i$.
\item If $t = t_1 \wedge \dots \wedge t_k$ is a formal meet, then (\ref{eq:ast}) holds iff 
$s^{\mathbf{F}(X)} \leqslant t_i^{\mathbf{F}(X)}$ for all $i$.
\item If $s = x_i$ and $t = t_1 \vee \dots \vee t_k$ is a formal join, 
   then (\ref{eq:ast}) holds iff $x_i \leqslant t_j^{\mathbf{F}(X)}$ for some $j$.
\item If $s = s_1 \wedge \dots \wedge s_k$ is a formal meet and $t = x_i$, then (\ref{eq:ast}) holds iff $s_j^{\mathbf{F}(X)} \leqslant x_i$ for some $j$.
\item If $s = s_1 \wedge \dots \wedge s_k$ is a formal meet and 
and $t = t_1 \vee \dots \vee t_m$ is a formal join, then (\ref{eq:ast}) holds iff 
$s_i^{\mathbf{F}(X)} \leqslant t^{\mathbf{F}(X)}$ for some $i$
or $s^{\mathbf{F}(X)} \leqslant t_j^{\mathbf{F}(X)}$ for some $j$
\end{enumerate}

\end{theorem}

\begin{definition}[up directed, continuous] A subset $A$ of a lattice $L$ is said to be \emph{up directed} if every finite subset of $A$ has an upper bound in $A$.
It suffices to check this for pairs.  $A$ is up directed iff for all $a, b \in A$
there exists $c\in A$ such that $a\leqslant c$ and $b\leqslant c$.  A lattice is \emph{upper continuous} if whenever $A\subseteq L$ is an up directed set having a least upper
bound $u = \bigvee A$, then for every $b$,

\[\bigvee_{a\in A} (a \wedge b) = 
\bigvee_{a\in A} a \wedge b =  u \wedge b.\]

\emph{Down directed} and \emph{down continuous} are defined dually.  A lattice that is 
both up and down continuous is called \emph{continuous}.
\end{definition}

\begin{theorem}[\protect{\cite[Thm.~1.22]{MR1319815}}]
  Free lattices are continuous.
\end{theorem}

\subsection{Bounded Homomorphisms}
\label{sec:bound-homom}

Following Freese, Jezek, Nation~\cite{MR1319815}, we define a pair of closure operators, denoted by superscripts $^\wedge$ and $^\vee$, on  subsets of an arbitrary lattice
$\mathbf L = \langle L, \vee, \wedge\rangle$ as follows: For each $A\subseteq L$, let
\[
A^\wedge = \{\bigwedge B : B \text{ is a finite subset of } A\}.
\]
We adopt the convention that if $\mathbf L$ has a greatest element $1_{\mathbf L}$, then 
$\bigwedge \emptyset = 1_{\mathbf L}$, and we include this in $A^\wedge$ for every $A \subseteq L$.
(For lattices without a greatest element, $\bigwedge \emptyset$ is undefined.)  The set $A^\vee$ is defined dually.

If $\mathbf K = \langle K, \vee, \wedge\rangle$ is a lattice generated by a finite set $X$, then we can write $K$ as a union of a chain of subsets $H_0\subseteq H_1 \subseteq \cdots$ defined inductively by setting $H_0 = X^\wedge$ and $H_{k+1} = (H_k)^{\vee \wedge}$, for all $k\geqslant 0$. By induction, each $H_n = X^{\wedge(\vee\wedge)^n}$ is a finite meet-closed subset of $K$, and $\bigcup H_n = K$, since $X$  generates $\mathbf X$.

Let $h \colon \mathbf K \to \mathbf L$ be a lattice epimorphism and, for each $y \in L$ and $k< \omega$, define
\[
\beta_k(y) = \bigwedge \{w \in H_k : h(w) \geqslant a\}.
\]

On page 30 of~\cite{MR1319815}, immediately after Theorem 2.4, the authors make the following remark, which is a crucial ingredient of our proof:

"...[$h$ is lower bounded] if and only if for each $a \leqslant h(1_{\mathbf K})$ there exists $N\in \omega$ such that $\beta_n(a) = \beta_N(a)$ for all $n\geqslant N$."

Equivalently, 
\begin{align}
h \text{ is not lower bounded } &\iff (\exists y_0 \in L)(\forall N)(\exists n> N)\, \beta_n(a) \neq \beta_N(a) \nonumber \\
&\iff (\exists y_0 \in L)(\exists N)(\forall n> N) \beta_n(a) \neq \beta_N(a).\nonumber
\end{align}

\bibliographystyle{alphaurl}
\bibliography{inputs/refs}

\end{document}
