%% FILE: diffTerm.tex
%% AUTHOR: William DeMeo, Peter Mayr, Nik Ruskuc
%% DATE: 14 May 2018
%% COPYRIGHT: (C) 2018 DeMeo, Mayr, Ruskuk
%%%%%%%%%%%%%%%%%%%%%%%%%%%%%%%%%%%%%%%%%%%%%%%%%%%%%%%%%%
%%                         BIBLIOGRAPHY FILE            %%
%%%%%%%%%%%%%%%%%%%%%%%%%%%%%%%%%%%%%%%%%%%%%%%%%%%%%%%%%%
%% The `filecontents` command will crete a file in the inputs directory called
%% refs.bib containing the references in the document, in case this file does
%% not exist already.
%% If you want to add a BibTeX entry, please don't add it directly to the
%% refs.bib file.  Instead, add it in this file between the
%% \begin{filecontents*}{refs.bib} and \end{filecontents*} lines
%% then delete the existing refs.bib file so it will be automatically generated
%% again with your new entry the next time you run pdfaltex.
\begin{filecontents*}{inputs/refs.bib}
  @book {MR1319815,
      AUTHOR = {Freese, Ralph and Je{\v{z}}ek, Jaroslav and Nation, J. B.},
       TITLE = {Free lattices},
      SERIES = {Mathematical Surveys and Monographs},
      VOLUME = {42},
   PUBLISHER = {American Mathematical Society, Providence, RI},
        YEAR = {1995},
       PAGES = {viii+293},
        ISBN = {0-8218-0389-1},
     MRCLASS = {06B25 (06-02 06-04 06B20 68Q25)},
    MRNUMBER = {1319815 (96c:06013)},
  MRREVIEWER = {T. S. Blyth},
         DOI = {10.1090/surv/042},
         URL = {http://dx.doi.org/10.1090/surv/042},
  }
  @article {MR3239624,
      AUTHOR = {Valeriote, M. and Willard, R.},
       TITLE = {Idempotent {$n$}-permutable varieties},
     JOURNAL = {Bull. Lond. Math. Soc.},
    FJOURNAL = {Bulletin of the London Mathematical Society},
      VOLUME = {46},
        YEAR = {2014},
      NUMBER = {4},
       PAGES = {870--880},
        ISSN = {0024-6093},
     MRCLASS = {08A05 (06F99 68Q25)},
    MRNUMBER = {3239624},
         DOI = {10.1112/blms/bdu044},
         URL = {http://dx.doi.org/10.1112/blms/bdu044},
  }
  @book {MR2839398,
      AUTHOR = {Bergman, Clifford},
       TITLE = {Universal algebra},
      SERIES = {Pure and Applied Mathematics (Boca Raton)},
      VOLUME = {301},
        NOTE = {Fundamentals and selected topics},
   PUBLISHER = {CRC Press, Boca Raton, FL},
        YEAR = {2012},
       PAGES = {xii+308},
        ISBN = {978-1-4398-5129-6},
     MRCLASS = {08-02 (06-02 08A40 08B05 08B10 08B26)},
    MRNUMBER = {2839398 (2012k:08001)},
  MRREVIEWER = {Konrad P. Pi{\'o}ro},
  }
  @article{Freese:2009,
      AUTHOR = {Freese, Ralph and Valeriote, Matthew A.},
      TITLE = {On the complexity of some {M}altsev conditions},
      JOURNAL = {Internat. J. Algebra Comput.},
      FJOURNAL = {International Journal of Algebra and Computation},
      VOLUME = {19},
      YEAR = {2009},
      NUMBER = {1},
      PAGES = {41--77},
      ISSN = {0218-1967},
      MRCLASS = {08B05 (03C05 08B10 68Q25)},
      MRNUMBER = {2494469 (2010a:08008)},
      MRREVIEWER = {Clifford H. Bergman},
      DOI = {10.1142/S0218196709004956},
      URL = {http://dx.doi.org/10.1142/S0218196709004956}
    }
  @article {MR3076179,
      AUTHOR = {Kearnes, Keith A. and Kiss, Emil W.},
       TITLE = {The shape of congruence lattices},
     JOURNAL = {Mem. Amer. Math. Soc.},
    FJOURNAL = {Memoirs of the American Mathematical Society},
      VOLUME = {222},
        YEAR = {2013},
      NUMBER = {1046},
       PAGES = {viii+169},
        ISSN = {0065-9266},
        ISBN = {978-0-8218-8323-5},
     MRCLASS = {08B05 (08B10)},
    MRNUMBER = {3076179},
  MRREVIEWER = {James B. Nation},
         DOI = {10.1090/S0065-9266-2012-00667-8},
         URL = {http://dx.doi.org/10.1090/S0065-9266-2012-00667-8},
  }
  @misc{william_demeo_2016_53936,
    author       = {DeMeo, William and Freese, Ralph},
    title        = {AlgebraFiles v1.0.1},
    month        = May,
    year         = 2016,
    doi          = {10.5281/zenodo.53936},
    url          = {http://dx.doi.org/10.5281/zenodo.53936}
  }
  @article{FreeseMcKenzie2016,
    Author = {Freese, Ralph and McKenzie, Ralph},
    Date-Added = {2016-08-22 19:43:56 +0000},
    Date-Modified = {2016-08-22 19:45:50 +0000},
    Journal = {Algebra Universalis},
    Title = {Mal'tsev families of varieties closed under join or Mal'tsev product},
    Year = {to appear}
  }
  @misc{UACalc,
    Author = {Ralph Freese and Emil Kiss and Matthew Valeriote},
    Date-Added = {2014-11-20 01:52:20 +0000},
    Date-Modified = {2014-11-20 01:52:20 +0000},
    Note = {Available at: {\verb+www.uacalc.org+}},
    Title = {Universal {A}lgebra {C}alculator},
    Year = {2011}
  }
  @incollection {MR0472614,
      AUTHOR = {J{\'o}nsson, B. and Nation, J. B.},
       TITLE = {A report on sublattices of a free lattice},
   BOOKTITLE = {Contributions to universal algebra ({C}olloq., {J}\'ozsef
                {A}ttila {U}niv., {S}zeged, 1975)},
       PAGES = {223--257. Colloq. Math. Soc. J\'anos Bolyai, Vol. 17},
   PUBLISHER = {North-Holland, Amsterdam},
        YEAR = {1977},
     MRCLASS = {06A20},
    MRNUMBER = {0472614},
  MRREVIEWER = {Ralph Freese},
  }
  @article {MR0313141,
      AUTHOR = {McKenzie, Ralph},
       TITLE = {Equational bases and nonmodular lattice varieties},
     JOURNAL = {Trans. Amer. Math. Soc.},
    FJOURNAL = {Transactions of the American Mathematical Society},
      VOLUME = {174},
        YEAR = {1972},
       PAGES = {1--43},
        ISSN = {0002-9947},
     MRCLASS = {06A20 (08A15)},
    MRNUMBER = {0313141},
  MRREVIEWER = {G. Gr{\"a}tzer},
  }
  \end{filecontents*}
  %:biblio
  %\documentclass[12pt]{amsart}
  \documentclass[12pt,reqno]{amsart}
  
  %%%%%%% wjd: added these packages vvvvvvvvvvvvvvvvvvvvvvvvv
  % PAGE GEOMETRY
  % These settings are for letter format
  \def\OPTpagesize{8.5in,11in}     % Page size
  \def\OPTtopmargin{1in}     % Margin at the top of the page
  \def\OPTbottommargin{1in}  % Margin at the bottom of the page
  %% \def\OPTinnermargin{0.5in}    % Margin on the inner side of the page
  \def\OPTinnermargin{1.5in}    % Margin on the inner side of the page
  \def\OPTbindingoffset{0in} % Extra offset on the inner side
  %% \def\OPToutermargin{0.75in}   % Margin on the outer side of the page
  \def\OPToutermargin{1.5in}   % Margin on the outer side of the page
  \usepackage[papersize={\OPTpagesize},
               twoside,
               includehead,
               top=\OPTtopmargin,
               bottom=\OPTbottommargin,
               inner=\OPTinnermargin,
               outer=\OPToutermargin,
               bindingoffset=\OPTbindingoffset]{geometry}
  \usepackage{url,amssymb,enumerate,tikz,scalefnt}
  \usepackage[normalem]{ulem} % for \sout (strikeout)   wjd: could remove this in final draft
  \usepackage[colorlinks=true,urlcolor=blue,linkcolor=blue,citecolor=blue]{hyperref}
  \usepackage{algorithm2e}
  \usepackage{stmaryrd}
  
  \newcommand\alg[1]{\ensuremath{\mathbf{#1}}}
  %% uncomment the next line if we want to revert to the "set" minus notation
  %% \renewcommand{\mysetminus}{\ensuremath{\setminus}} 
  
  \usepackage[yyyymmdd,hhmmss]{datetime}
  \usepackage{background}
  \backgroundsetup{
    position=current page.east,
    angle=-90,
    nodeanchor=east,
    vshift=-1cm,
    hshift=8cm,
    opacity=1,
    scale=1,
    contents={\textcolor{gray!80}{WORK IN PROGRESS.  DO NOT DISTRIBUTE. (compiled on \today\ at \currenttime)}}
  }
  %%%%%%  (end wjd addition of packages)
  
  
  \usepackage{pdfcomment}
  \usepackage{color}
  \usepackage{amsmath}
  \usepackage{amssymb}
  \usepackage{amsfonts}
  \usepackage{mathtools}
  \usepackage{amscd}
  %% \usepackage{exers}
  \usepackage{inputs/wjdlatexmacs}
  
  \usepackage[mathcal]{euscript}
  \usepackage{comment}
  \usepackage{tikz}
  \usetikzlibrary{math} %needed tikz library
  
  \renewcommand{\th}[2]{#1\mathrel{\theta}#2}
  \newcommand{\infixrel}[3]{#2\mathrel{#1}#3}
  \newcommand\llb{\ensuremath{\llbracket}}
  \newcommand\rrb{\ensuremath{\rrbracket}}
  
  %%////////////////////////////////////////////////////////////////////////////////
  %% Theorem styles
  \numberwithin{equation}{section}
  \theoremstyle{plain}
  \newtheorem{theorem}{Theorem}[section]
  \newtheorem{lemma}[theorem]{Lemma}
  \newtheorem{proposition}[theorem]{Proposition}
  \newtheorem{prop}[theorem]{Prop.}
  \theoremstyle{definition}
  \newtheorem{conjecture}{Conjecture}
  \newtheorem{claim}[theorem]{Claim}
  \newtheorem{subclaim}{Subclaim}
  \newtheorem{corollary}[theorem]{Corollary}
  \newtheorem{definition}[theorem]{Definition}
  \newtheorem{notation}[theorem]{Notation}
  \newtheorem{Fact}[theorem]{Fact}
  \newtheorem*{fact}{Fact}
  \newtheorem{example}[theorem]{Example}
  \newtheorem{examples}[theorem]{Examples}
  \newtheorem{exercise}{Exercise}
  \newtheorem*{lem}{Lemma}
  \newtheorem*{cor}{Corollary}
  \newtheorem*{remark}{Remark}
  \newtheorem*{remarks}{Remarks}
  \newtheorem*{obs}{Observation}
  
  \title[Kernel of $\alg F(x,y,z) \twoheadrightarrow \alg F_{\alg M_3}(0,1,2)$]{Kernel of the homomorphism from $\alg F (x,y,z)$ onto the
  free modular lattice on three generators}
  % \author[W.~DeMeo]{William DeMeo}
  \email{williamdemeo@gmail.com}
  %% \urladdr{http://williamdemeo.github.io}
  %% \address{University of Colorado\\Mathematics Dept\\Boulder 80309\\USA}
  
  % \author[P.~Mayr]{Peter Mayr}
  \email{Peter.Mayr@colorado.edu}
  %% \urladdr{}
  %% \address{University of Colorado\\Mathematics Dept\\Boulder 80309\\USA}
  
  % \author[N.~Ru\v{s}kuc]{Nik Ru\v{s}kuc}
  \email{nik.ruskuc@st-andrews.ac.uk}
  %% \urladdr{}
  %% \address{University of St. Andrews\\Mathematics Dept\\St. Andrews, Scottland}
  
  %% \thanks{The first and second authors were supported by the National
  %% Science Foundation under Grant No...}
  
  \date{\today}
  
  \begin{document}
  
  \maketitle
  
  \section{J.B.'s Example}

  \newcommand{\dotsize}{1.5pt}
  \tikzstyle{lat} = [circle,draw,inner sep=\dotsize]
  \newcommand{\figscale}{1}
  
  \begin{figure}[!h]
  \begin{tikzpicture}[scale=\figscale]
    \foreach \j in {0,...,8} {
      \node[lat] (0\j) at (0,\j) {};
    }
    \foreach \j in {1,2,4,6,7} {
      \node[lat] (n1\j) at (-1,\j) {};
      \node[lat] (1\j) at (1,\j) {};
    }
    \foreach \j in {3,5} {
      \node[lat] (n2\j) at (-2,\j) {};
      \node[lat] (2\j) at (2,\j) {};
    }
    \node[lat] (n34) at (-3,4) {};
    \node[lat] (34) at (3,4) {};
  
    \draw[semithick] (00) -- (01);
    \draw[semithick] (02) -- (03) -- (04) -- (05) -- (06);
    \draw[semithick] (07) -- (08);
    \draw[semithick] (n16) -- (n17);
    \draw[semithick] (16) -- (17);
    \draw[semithick] (n11) -- (n12);
    \draw[semithick] (11) -- (12);
  
    \draw[semithick] (00) -- (n11) -- (02) -- (11) -- (00);
    \draw[semithick] (06) -- (n17) -- (08) -- (17) -- (06);
  
    \draw[semithick] (01) -- (n12) -- (n23) -- (n34);
    \draw[semithick] (01) -- (12) -- (23) -- (34);
    \draw[semithick] (12) -- (03) -- (n14) -- (n25);
    \draw[semithick] (n12) -- (03) -- (14) -- (25);
    \draw[semithick] (23) -- (14) -- (05) -- (n16);
    \draw[semithick] (n23) -- (n14) -- (05) -- (16);
    \draw[semithick] (34) -- (25) -- (16) -- (07);
    \draw[semithick] (n34) -- (n25) -- (n16) -- (07);
    \node[lat] (n13) at (-1,3) {};
    \node[lat] (n15) at (-1,5) {};
    \node[lat] (n24) at (-2,4) {};
    \draw[thick,blue] (02) -- (n13) -- (n24) -- (n15) -- (06);
    \draw[thick,blue] (n13) -- (04) -- (n15);
  
    \node (07) at (0,7) [blue,circle,fill,inner sep=\dotsize]{};
    \node (n17) at (-1,7) [blue,circle,fill,inner sep=\dotsize]{};
    \node (17) at (1,7) [blue,circle,fill,inner sep=\dotsize]{};
    \node (x) at (-3,4) [circle,fill,inner sep=\dotsize]{};
    \node (y) at (3,4) [circle,fill,inner sep=\dotsize]{};
    \node (n23) at (-2,3) [red,circle,fill,inner sep=\dotsize]{};
    \node (n13) at (-1,3) [red,circle,fill,inner sep=\dotsize]{};
    \node (23) at (2,3) [red,circle,fill,inner sep=\dotsize]{};
    \node (05) at (0,5) [green,circle,fill,inner sep=\dotsize]{};
    \node (z) at (-2,4) [circle,fill,inner sep=\dotsize]{};
    \draw (x) node[left] {$0$};
    \draw (17) node[right] {$1 \vee 2$};
    \draw (n17) node[left] {$0 \vee 2$};
    \draw (n23) node[left] {$a = 0 \wedge (1 \vee 2)$};
    \draw (23) node[right] {$b = 1 \wedge (0 \vee 2)$};
    \draw (y) node[right] {$1$};
    \draw (z) node[left] {$2$};
    \end{tikzpicture}
    \caption{The free lattice over $M_3$ generated by $\{0, 1, 2\}$.}
    % blue dots: $0 \vee 1$, $0 \vee 2$, $1 \vee 2$;
    % green dot: $(0 \vee 1) \wedge (0 \vee 2)  \wedge (1 \vee 2)$; 
    % red dots: $0 \wedge (1 \vee 2)$, $1 \wedge (0 \vee 2)$, $2 \wedge (0 \vee 1)$.}
    % a yellow dot marks $(0 \wedge 1) \vee (0 \wedge 2)  = (0 \wedge 1) \vee (1 \wedge 2) = (0 \wedge 2)\vee (1 \wedge 2)$; 
    % green dots mark $0 \wedge 1$, $0 \wedge 2$, and $1 \wedge 2$; \\
    \label{fig:1}
  \end{figure}
  
  

J.B.~Nation showed us an example of a finite lattice $\alg L$ 
and an epimorphism $h \colon \alg F(x,y,z) \to \alg L$
for which the unbounded $h$-kernel classes do not contain generators of $\alg{F}(x,y,z)$;
however, these unbounded classes do map onto join prime elements of $\alg L$.

In this note we show how to exploit the existence of join prime elements in the image of 
infinite $h$-kernel classes, and the fact that there are three unbounded 
$h$-kernel classes forming an antichain, to prove that the kernel of $h$ 
is not a finitely generated subalgebra of $\alg{F} \times \alg{F}$.
\begin{proposition}
  \label{prop:2.2}
Let $\mathbf{F} = \mathbf{F}(x,y,z)$, and let $\mathbf{L} = \mathbf{F}_{\mathbf{M}_3}(0,1,2)$ (see Figure~\ref{fig:1}).  
Let $h\colon \mathbf{F} \twoheadrightarrow \mathbf{L}$ be the epimorphism induced by $x\mapsto 0$, $y\mapsto 1$, $z\mapsto 2$. Then $\operatorname{ker}h$ is not finitely generated.  
\end{proposition}

To prove this proposition we will need some notation that, for situations like ours, 
is a standard way to write certain sequences of elements of $\alg{F}$. (See, e.g.,~\cite{MR1319815}.) 
Let $a = 0 \wedge (1 \vee 2)$, $b = 1 \wedge (0 \vee 2)$, $c = 2 \wedge (0 \vee 1)$.
For each $d\in \{a, b, c\}$, define the sequences $\{\alpha_n(d)\}$ and $\{\beta_n(d)\}$ 
by the following recursive formulas:
\begin{alignat*}{3}
  \alpha_0(a) &=& x \vee (y \wedge z) \qquad \alpha_{n+1}(a)   &= &x \vee (\alpha_n(b) \wedge \alpha_n(c))\\
  \alpha_0(b) &=& y \vee (x \wedge z)\qquad \alpha_{n+1}(b)   &= &y \vee (\alpha_n(a) \wedge \alpha_n(c))\\
  \alpha_0(c)   &=& z \vee (x \wedge y)\qquad  \alpha_{n+1}(c)   &= &z \vee (\alpha_n(a) \wedge \alpha_n(b))
\end{alignat*}
and 
\begin{alignat*}{3}
  \beta_0(a)   &= &x \wedge (y \vee z)  \qquad \beta_{n+1}(a)   &= &x \wedge (\beta_n(b) \vee \beta_n(c))\\
  \beta_0(b)   &= & y \wedge (x \vee z)  \qquad  \beta_{n+1}(b)   &=& y \wedge (\beta_n(a) \vee \beta_n(c))\\
  \beta_0(c)   &= &z \wedge (x \vee y) \qquad   \beta_{n+1}(c)   &=& z \wedge (\beta_n(a) \vee \beta_n(b))
\end{alignat*}
It is not hard to show that for each $d \in \{a,b,c\}$, the sequence $\alpha_k(d)$ (resp., $\beta_k(d)$) is
an infinite ascending (resp., descending) chain.

Now, suppose $K \subseteq \alg{F} \times \alg{F}$ is a finite set. 
We wish to prove $\<K\> \neq \ker h$.

\subsection{Exploiting non-upper boundedness}
First, we show there exists $N < \omega$ such that 
for all $(p, q) \in \<K\>$ 
the following implications hold:
\begin{align}
  p \leq \alpha_0(a) &\implies q \leq \alpha_N(a); \label{eq:acc01}\\
  p \leq \alpha_0(b) &\implies q \leq \alpha_N(b); \label{eq:acc02}\\ 
  p \leq \alpha_0(c) &\implies q \leq \alpha_N(c); \label{eq:acc03}
  % p \leq w_0 &\implies q \leq w_N. \label{eq:acc04}\\
  % p \geq \beta_0(a) &\implies q \geq \beta_N(a); \label{eq:dcc01}\\
  % p \geq \beta_0(b) &\implies q \geq \beta_N(b); \label{eq:dcc02}\\ 
  % p \geq \beta_0(c) &\implies q \geq \beta_N(c); \label{eq:dcc03}\\
  % p \geq m_0 &\implies q \geq m_N. \label{eq:dcc04}
\end{align}
Let us give a name to the set of strictly ascending 
sequences involved in implications~(\ref{eq:acc01})--(\ref{eq:acc03}), say, 
$\mathcal A = \{\{\alpha_n(a)\}, \{\alpha_n(b)\}, \{\alpha_n(c)\}\}$. %, \{w_n\}\}.
%   and let $\mathcal D$ denote the set of strictly descending sequences.
% $\mathcal D = \{\{\beta_n(a)\}, \{\beta_n(b)\}, \{\beta_n(c)\}, \{m_n\}\}$.
Obviously we can find, for each sequence $s \in \mathcal A$, % \cup \mathcal D$,
a number $N_s < \omega$ such that the implication involving $s$
holds for $N = N_s$ and for all of the (finitely many) pairs in $K$.  
Therefore, all of these implications will hold for $(p, q) \in K$ if we take 
$N = \max \{N_{s} : s \in \mathcal A\}$. % \cup \mathcal D\}$.
In fact, for this choice of $N$, implications~(\ref{eq:acc01})--(\ref{eq:acc03}) 
are satisfied not only for every $(p, q) \in K$, but also for all $(p, q) \in \<K\>$,
as we now prove.

\begin{claim}\label{claim:main_alpha}
  The implications~(\ref{eq:acc01})--(\ref{eq:acc03}) hold for all $(p,q) \in \<K\>$.
  \end{claim}
  \begin{proof}
    Fix $(p, q) \in  \<K\>$. 
  
    In case $(p,q)$ belongs to the finite set $K$, all eight implications were already 
    established in the paragraph immediately preceding the statement of the claim.
    
    \medskip
    \noindent \textbf{Case 1.}
    Suppose $(p,q) = (p_1, q_1) \wedge (p_2, q_2)$, and assume (the induction hypothesis) 
    that~(\ref{eq:acc01})--(\ref{eq:acc03}) hold for
    $(p_i,q_i)$ $i\in \{1,2\}$.
    We now prove that, under these assumptions, $(p,q)$ 
    satisfies~(\ref{eq:acc01})--(\ref{eq:acc03}).
    
    Assume $p = p_1\wedge p_2 \leq \alpha_0(a)$; we must show 
    $q=q_1\wedge q_2 \leq \alpha_N(a)$.
    The hypothesis $p_1\wedge p_2 \leq \alpha_0(a) = x\vee (y \wedge z)$, along with 
    Whitman's Condition (W+), implies that one of the following holds:
    \begin{enumerate}
      \item   $p_1=x$;  
      \item   $p_2=x$;  
      \item   $p_1\leq \alpha_0(a)$;  
      \item   $p_2\leq \alpha_0(a)$;  
      \item   $p_1\wedge p_2 \leq y \wedge z$;  
    \end{enumerate}
    In the first case we have $p_1 = x \leq x \vee (y \wedge z) = \alpha_0(a)$, so
    the induction hypothesis implies $q_1\leq \alpha_N(a)$, so 
    $q = q_1\wedge q_2\leq \alpha_N(a)$, as desired.  
    Notice that the first four items in the list submit to the same argument,
    so we are left with the fifth case, 
    in which $p_1\wedge p_2 \leq y \wedge z$. 
    This implies in particular that $p_1\wedge p_2 \leq y$, and $y$ is meet prime, so 
    $p_i\leq y$ for some $i\in \{1,2\}$.  
    But then, $p_i \leq \alpha_0(b) = y\vee (x\wedge z)$, and the induction hypothesis
    yields $q_i \leq \alpha_N(b)$.
    Similarly, $p_1\wedge p_2 \leq z$ implies 
    $p_j\leq z$ for some $j\in \{1,2\}$, so $q_j \leq \alpha_N(c)$.
    This yields the desired result,
    \[
    q = q_1\wedge q_2 \leq \alpha_N(b) \wedge \alpha_N(c) 
    \leq x \vee (\alpha_N(b)\wedge \alpha_N(c)) = \alpha_N(a).
    \] 
  
    \medskip
    \noindent \textbf{Case 2.}
    Next, suppose $p = p_1\vee p_2 \leq \alpha_0(a)$; we show 
    $q = q_1 \vee q_2 \leq \alpha_N(a)$. 
    The relation $p_1\vee p_2 \leq \alpha_0(a)$ implies 
    $p_1\leq \alpha_0(a)$ and $p_2\leq \alpha_0(a)$, 
    so by the induction hypothesis we have 
    $q_1\leq \alpha_N(a)$ and $q_2\leq \alpha_N(a)$.  
    Therefore, $q = q_1 \vee q_2 \leq \alpha_N(a)$, as desired.

    We have thus shown that $(p,q)$ satisfies~(\ref{eq:acc01}).
    It is obvious that the same argument may be applied to prove 
    implications~(\ref{eq:acc02}) and~(\ref{eq:acc03}) as well.
  \end{proof}

  
  It follows Claim~\ref{claim:main_alpha} that 
  $\<K\> \neq \ker h$. Indeed, if we take $(p, q)$ to be 
  % since $\{\alpha_n(a)\}$ is a strictly increasing sequence, 
  % $\alpha_N(a) < \alpha_{N+1}(a)$, so 
  the pair $(\alpha_0(a), \alpha_{N+1}(a))$, then  $h(p) = a = h(q)$, 
  so $(p,q) \in \ker h$, but this choice of $(p,q)$ does not satisfy 
  condition~(\ref{eq:acc01}). On 
  the other hand, by Claim~\ref{claim:main_alpha} all pairs in $\<K\>$ 
  satisfy~(\ref{eq:acc01})--(\ref{eq:acc03}) when $K$ is a finite set.   The 
  proof of Proposition~\ref{prop:2.2} is now complete.\qed
  

  \subsection{Exploiting non-lower boundedness}
  It seems non-upper boundedness (and a result like Claim~(\ref{claim:main_alpha}) 
that exploits it) is all we need to prove that
$\ker h$ is not finitely generated.

In this example, $h$ is neither upper nor lower bounded,
and we could have proved an analog of Claim~\ref{claim:main_alpha} 
that exploits the fact that $h$ is not \emph{lower} bounded,
and then used this to prove Proposition~\ref{prop:2.2} (instead of 
using Claim~\ref{claim:main_alpha}).

We now state and prove the analog of Claim~\ref{claim:main_alpha}.
We will show there exists $N < \omega$ such that 
for all $(p, q) \in \<K\>$ 
the following implications hold:
\begin{align}
  p \geq \beta_0(a) &\implies q \geq \beta_N(a); \label{eq:dcc01}\\
  p \geq \beta_0(b) &\implies q \geq \beta_N(b); \label{eq:dcc02}\\ 
  p \geq \beta_0(c) &\implies q \geq \beta_N(c); \label{eq:dcc03}
\end{align}
Let $\mathcal D = \{\{\beta_n(a)\}, \{\beta_n(b)\}, \{\beta_n(c)\}\}$
denote the set of strictly descending 
sequences involved in implications~(\ref{eq:dcc01})--(\ref{eq:dcc03}).

Obviously we can find, for each sequence $s \in \mathcal D$,
a number $N_s < \omega$ such that the implication involving that sequence 
holds for $N = N_s$ and for all of the (finitely many) pairs in $K$.  
Therefore, all three implications will hold for $(p, q) \in K$ if we take 
$N = \max \{N_{s} : s \in \mathcal D\}$.
In fact, for this choice of $N$, all implications~(\ref{eq:dcc01})--(\ref{eq:dcc03}) hold 
not only for every $(p, q) \in K$, but also for all $(p, q) \in \<K\>$,
as we now prove.

\begin{claim}\label{claim:main_beta}
  The implications~(\ref{eq:dcc01})--(\ref{eq:dcc03}) hold for all $(p,q) \in \<K\>$.
  \end{claim}
  \begin{proof}
    Fix $(p, q) \in  \<K\>$. 
  
    In case $(p,q)$ belongs to the finite set $K$, the implications were already 
    justified in the paragraph immediately preceding the statement of the claim.
    
    \medskip
    \noindent \textbf{Case 1.}
    Suppose $(p,q) = (p_1, q_1) \wedge (p_2, q_2)$, and assume (the induction hypothesis) 
    that~(\ref{eq:dcc01})--(\ref{eq:dcc03}) are satisfied 
    by $(p_i,q_i)$ $i\in \{1,2\}$.
    % We now prove that, under these assumptions, $(p,q)$
    % satisfies~(\ref{eq:dcc01})--(\ref{eq:dcc03}).
    We must show $q=q_1\wedge q_2 \geq \beta_N(a)$.

    The relation $p_1\wedge p_2 \geq \beta_0(a)$ implies 
    $p_1\geq \beta_0(a)$ and $p_2\geq \beta_0(a)$, 
    so by the induction hypothesis we have 
    $q_1\geq \beta_N(a)$ and $q_2\geq \beta_N(a)$.  
    Therefore, $q = q_1 \vee q_2 \geq \beta_N(a)$, as desired.
    

    \medskip
    \noindent \textbf{Case 2.}
    Next, suppose $p = p_1\vee p_2 \geq \beta_0(a)$; we show 
    $q = q_1 \vee q_2 \geq \beta_N(a)$. 
    The relation $p_1\vee p_2 \geq \beta_0(a) = x \wedge (y \vee z)$ along with 
    Whitman's Condition (W+), implies that one of the following holds:
    \begin{enumerate}
      \item   $p_1=x$;  
      \item   $p_2=x$;  
      \item   $p_1\geq \beta_0(a)$;  
      \item   $p_2\geq \beta_0(a)$;  
      \item   $p_1\vee p_2 \geq y \vee z$;  
    \end{enumerate}
    In the first case we have $p_1 = x \geq x \wedge (y \vee z) = \beta_0(a)$, so
    the induction hypothesis implies $q_1\geq \beta_N(a)$, so 
    $q = q_1\vee q_2\leq \alpha_N(a)$, as desired.  
    Notice that the first four items in the list submit to the same argument,
    so we are left with the fifth case, 
    in which $p_1\vee p_2 \geq y \vee z$. 
    This implies in particular that $p_1\vee p_2 \geq y$, and, since $y$ is join prime,
    we have $p_i\geq y$ for some $i\in \{1,2\}$.  
    But then, $p_i \geq \beta_0(b) = y\wedge (x\vee z)$, and the induction hypothesis
    yields $q_i \geq \beta_N(b)$.
    Similarly, $p_1\vee p_2 \geq z$ implies 
    $p_j\geq z$ for some $j\in \{1,2\}$, so $q_j \geq \beta_N(c)$.
    This yields the desired result,
    \[
    q = q_1\vee q_2 \geq \beta_N(b) \vee \beta_N(c) 
    \geq x \wedge (\beta_N(b)\vee \beta_N(c)) = \beta_N(a).
    \]                   
    We have thus shown that $(p,q)$ satisfies the implication~(\ref{eq:dcc01}).
    It is obvious that the same argument applies to the 
    implications~(\ref{eq:dcc02})--(\ref{eq:dcc03}).
  \end{proof}
  

\bigskip

\bibliographystyle{alphaurl}
\bibliography{inputs/refs}


%%%%%%%%%%%%%%%%%%%%%%%%%%%%%%%%%%%%%%%%%%%%%%%%%%%%%%%%%%%%%%%%%%%%%%%%%%
%%%%%%%%%%%%%%%%%%%%%%%%%%%%%%%%%%%%%%%%%%%%%%%%%%%%%%%%%%%%%%%%%%%%%%%%%%
%%%%%%%%%%%%%%%%%%%  END OF DOCUMENT %%%%%%%%%%%%%%%%%%%%%%%%%%%%%%%%%%%%%
%%%%%%%%%%%%%%%%%%%%%%%%%%%%%%%%%%%%%%%%%%%%%%%%%%%%%%%%%%%%%%%%%%%%%%%%%%
%%%%%%%%%%%%%%%%%%%%%%%%%%%%%%%%%%%%%%%%%%%%%%%%%%%%%%%%%%%%%%%%%%%%%%%%%%

\bigskip
Please direct comments, questions, or suggestions to:

%%%%%%%%%%%%%%%%%%%%%%%%%%%%%%%%%%%%%%%%%%%%%%%%%%%%%%%%%%%%%%%%%%%%%%%%%%
%%%%%%%%%%%%%%%%%%%%%%%%%%%%%%%%%%%%%%%%%%%%%%%%%%%%%%%%%%%%%%%%%%%%%%%%%%
%%%%%%%%%%%%%%%%%%%  END OF DOCUMENT %%%%%%%%%%%%%%%%%%%%%%%%%%%%%%%%%%%%%
%%%%%%%%%%%%%%%%%%%%%%%%%%%%%%%%%%%%%%%%%%%%%%%%%%%%%%%%%%%%%%%%%%%%%%%%%%
%%%%%%%%%%%%%%%%%%%%%%%%%%%%%%%%%%%%%%%%%%%%%%%%%%%%%%%%%%%%%%%%%%%%%%%%%%

\end{document}
